%\section{FPGA un ASIC tehnoloģijas}
\section{Sintēzes mērķa platformas un tehnoloģijas} \label{sec:synth}
Kā tika apskatīts iepriekš, sintēzes mērķis ir iegūt fizisko shēmas
realizāciju no tās HDL apraksta. Līdz šim tika apskatīts sintēzes process
abstrahējoties no mērķa platformas.

Kad sintēzes rīks ir translējis HDL aprakstu un sagatavojis shēmas 
iekšējo repre\-zen\-tā\-ciju, ir nepieciešams to nokartēt uz konkrētiem
loģiskajiem elementiem \cite[5.~lpp.]{HDL}.
Lai gan tehniski ir iespējama kartēšana uz
atsevišķām loģisko elementu (piem. 7400 sērijas vai 4000 sērijas)
mikroshēmām, praktiski vienmēr mērķis ir realizēt shēmu vienā
mikroshēmā. Šādai realizācijai tiek izmantota FPGA un ASIC tehnoloģijas.

\subsection{FPGA} \label{sec:fpga}
	FPGA (\termEn{field-programmable gate array}) jeb pārprogrammējams
	loģisko elementu masīvs ir integrētā shēma, kura sastāv no
	programmējamu loģisko bloku matricas ar konfigurējamu starpsavienojumu
	tīklu (\termEn{the interconnect}) [\todo].
	Vienkāršoti, FPGA var uzskatīt par ,,tukšu'' mikroshēmu, kurā, ar HDL
	un ražotāja programmatūras palīdzību, var ieprogrammēt vēlamo
	funkcionalitāti.
	
	Konkrēta loģisko bloku un to starpsavienojumu implementācija ir ražotāja un 
	FPGA modeļa	specifiska\cite{SmartFusionFabric}\cite{Xilinx7}, 
	bet parasti piedāvā loģiskos blokus konfigurēt kā:
	\begin{itemize}
		\item kombinacionālos loģikas elementus (ar 2--6 ieejām);
		\item statiskos D tipa trigerus;
		\item dinamiskos D tipa trigerus.
	\end{itemize}
	
	Kombinacionālās shēmas FPGA loģiskajos blokos praktiski vienmēr tiek
	realizētas ar LUT (\termEn{Look-up table}) elementiem%
	\cite{SmartFusionFabric}\cite{Xilinx7}, t.i.~reāli loģiskie elementi
	netiek imple\-men\-tēti, bet tiek uzmeklētas izejas vērtības no LUT
	uzglabātās patiesības tabulas. Var teikt ka LUT ,,emulē'' loģiskos
	elementus (sk.~piem.~\ref{fig:and2lut}~att.),
	bet funkcionāli šī atšķirība ir maznozīmīga, savukārt
	iespēja LUT elementus pielāgot nepieciešamajai loģiskajai funkcijai ir 
	FPGA tehnoloģijas pamatā.
	\begin{figure}[hb]
		\centering
		%\def\svgwidth{7cm}
		%\def\svgscale{1.25}
		{\ttfamily\small\input{img/and2lut.pdf_tex}}
		\caption{Loģiskais elements \texttt{UN}, ar tā ekvivalento LUT.}
		\label{fig:and2lut}
	\end{figure}
	
	Saskarnei ar perifēriju, pa mikroshēmas perimetru tiek izvietotas
	ieejas/izejas pieslēgvietas. Papildus loģiskajiem blokiem un
	pieslēgvietām, uz mikroshēmas specializētus resursus, piem. operatīvo
	atmiņu, \textit{Flash} atmiņu, analogo signālu ciparotājus un 
	atciparotājus, standarta perifērās saskarnes, u.c.,%
	\cite{SmartFusionFabric}\cite{Xilinx7} kuri veiktspējas uzlabošanas 
	un mikroshēmas aizņemtā laukuma samazināšanas nolūkos, implementēti
	tradicionālā stilā ar nekonfigurējamu, ,,cieto'' loģiku.
	
	Lai sintezētu shēmu priekš FPGA, sintēzes rīks nokartē iekšējo shēmas
	reprezentāciju uz elementiem, kuru spēj implementēt FPGA loģiskie bloki.[\todo]
	Šo elementu saraksts ir pieejams FPGA ražotāja bibliotēkā, un ir
	iekļauts ražotāja FPGA projektēšanas rīku programmpakā.
	
	Visai nozīmīgs, bet grūti salīdzināms, FPGA arhitektūras rādītājs ir 
	loģisko bloku granularitāte. Par pamatu šim salīdzinājumam izmantosim
	rupjas granularitātes\linebreak
	Xilinx 7 paaudzes ,,CLB'' arhitektūru\cite{Xilinx7}
	un smalkas granularitātes Actel ,,VersaTile''
	arhitektūru\cite{SmartFusionFabric}.
	
	Rupjas granularitātes loģisko bloku iespējams nokonfigurēt par
	komplicētāku kombinacionālo vai secīgo elementu, pretstatā smalkas
	granularitātes loģiskajam blokam. Piemēram, CLB bloks satur 4 LUT un
	dinamisko trigeru pārus, kuru katru iespējams
	konfigurēt par 6 ieeju (un vienas izejas) kombinacionālo loģisko
	funkciju, kura izejas vērtību iespējams saglabāt trigerī.%
	\cite[6.~lpp.]{Xilinx7}
	Savukārt VersaTile bloku var konfigurēt par 3 ieeju kombinacionālo
	loģisko funkciju vai trigeri (dinamisko vai statisko).%
	\cite[3.~lpp.]{SmartFusionFabric}
	Var visai vienkārši secināt, ka, ignorējot citas uzbūves īpašības, 
	noteiktā laukumā iespējams novietot daudz vairāk smalkās granularitātes
	bloku, nekā rupjās granularitātes bloku (sk.~\ref{fig:tiles}~att.).
	\begin{figure}[hb]
		\centering
		%\def\svgwidth{0.75\textwidth}
		\input{img/tile-density.pdf_tex}
		\caption[Loģisko bloku granularitātes salīdzinājums.]%
			{Loģisko bloku granularitātes salīdzinājums 
				(kur v ir patvaļīga garuma vienība).}
		\label{fig:tiles}
	\end{figure}
	
	Balstoties uz šo secinājumu var spriest, ka realizējot dažādas shēmas
	uz smalkas granularitātes arhitektūras,
	izmantoto loģisko elementu blīvums ir tuvu konstants, savukārt rupjas
	granularitātes arhitektūrai izmantoto loģisko elementu blīvums var
	ievērojami mainīties atkarībā cik liela daļa no katra loģiskā bloka
	resursiem tiek izmantota. Tādējādi, autors uzskata, ka smalkas
	granularitātes arhitektūra ir universālāka, bet rupjas granularitātes 
	arhitektūra potenciāli ir ar lielāku veiktspēju, ja izstrādē tiek ņemta
	vērā konkrētās FPGA arhitektūras uzbūve.
	
	
\subsection{ASIC} \label{sec:asic}
	Ar ASIC (\textit{Application-specific integrated circuit}) apzīmē
	integrētās shēmas, kas ir specializētas un paredzētas konkrētam
	pielietojumam. ASIC izstrāde var notikt dažādos abstrakcijas līmeņos,
	bet to apvienojošā īpašība ir tā, ka projekta rezultāts tiek nodots
	ASIC ražotājam, kurš pēc pasūtījuma 
	masveidā saražo izprojektētās mikroshēmas.
	
	ASIC izstrādei ir ievērojami lielākas vienreizējās izmaksas, tādēļ ASIC
	izstrādi izmanto tikai ja paredzams liels vienību apjoms. Papildus
	apsvērumi varētu būt arī veiktspēja un/vai mikroshēmas lielums, jo,
	pretstatā FPGA, ASIC ir pielāgojami projektētāja vajadzībām.
	
	\subsubsection{Standartšūnu ASIC projektēšana}
		Standartšūna (\termEn{Standard cell}) ir vienkāršas komponentes
		modelis, ko ASIC ražotājs izstrādājis tranzistoru līmenī
		izvietošanai uz tā ražotajiem ASIC. ASIC ražotājs projektu
		izstrādei piedāvā standartšūnu bibliotēku, kura satur vismaz pamata
		loģiskos elementus un parasti arī trigerus un citas bieži
		izmantojamas komponentes. Standartšūnas mikroshēmas izstrādē ir
		uzskatāms par analoģisku standarta loģiskajām (piem. 7400 sērijas)
		komponentēm spiesto elektronisko plašu izstrādē.
	
		ASIC projektēšana izmantojot standartšūnu bibliotēku, ir visai
		līdzīga FPGA projektēšanai. Sintēzes rīka shēmas kartēšana uz
		bibliotēkas elementiem notiek identiski. Būtiskākā atšķirība
		ir ,,izvietošanas un savienošanas'' (\termEn{Place and Route}) solis,
		jo ierobežojošais faktors ir mikroshēmas laukums, nevis
		loģisko bloku skaits, [\todo{}?] un rezultāts ir 
		mikroshēmu izstrādes fotomaskas, nevis
		pārprogrammēšanas datu fails.
	
	\subsubsection{Specializētā ASIC projektēšana}
		Specializētā ASIC (\termEn{full custom ASIC}) projektēšanas gadījumā
		izstrāde notiek tranzistoru līmenī jeb zemākajā abstrakcijas
		līmenī. \todo
		
		
	%TODO: Strukturētais ASIC
