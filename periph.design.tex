\subsection{Implementācijas īpašības} \label{sec:design}
Izstrādājot mikrokontrolieri tika pieņemti vairāki lēmumi par tā uzbūvi,
veidojot specifikāciju kopu uz kuras
balstās izstrādātātās mikrokontroliera komponentes. \todo \\
\begin{figure}[bhp]
	\centering
	\def\svgwidth{\textwidth}
	{\input{img/uC-sheem.pdf_tex}}
	\caption{Augšējā līmeņa blokshēma (rev.~03).}
	\label{fig:top-rev3}
\end{figure}
Šim mikrokontrolierim ir sekojošas uzbūves īpašības:
\begin{description}
	\item[Vienota adrešu telpa] \hfill \\
		Mikrokontroliera kodols komunicē ar ārējām ierīcēm tikai caur
		atmiņas saskarni, bet ne obligāti šai ierīcei arī jābūt atmiņai.
		Tā vietā kopējai adrešu telpai tiek piekārtotas vairākas ierīces
		izdalot noteiktus atmiņas apgabalus. Adrešu dekoderis veido saskarni
		multipleksējot ierīces atkarībā no pieprasītās adreses.
		(sk.~\ref{sec:mmu}~nod.)
	\item[] \todo
\end{description}
