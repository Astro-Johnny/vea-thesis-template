\subsection{Implementācijas īpašības} \label{sec:design}
Izstrādājot mikrokontrolieri tika pieņemti vairāki lēmumi par tā uzbūvi,
veidojot specifikāciju kopu uz kuras
balstās izstrādātātās mikrokontroliera komponentes. \todo \\
\begin{figure}[bhp]
	\centering
	\def\svgwidth{\textwidth}
	{\input{img/uC-sheem.pdf_tex}}
	\caption{Augšējā līmeņa blokshēma (rev.~03).}
	\label{fig:top-rev3}
\end{figure}
Šim mikrokontrolierim ir sekojošas uzbūves īpašības:
\begin{description}
	\item[Vienota adrešu telpa] \hfill \\
		Mikrokontroliera kodols komunicē ar ārējām ierīcēm tikai caur
		atmiņas saskarni, bet ne obligāti šai ierīcei arī jābūt atmiņai.
		Tā vietā kopējai adrešu telpai tiek piekārtotas vairākas ierīces
		izdalot noteiktus atmiņas apgabalus. Adrešu dekoderis veido saskarni
		multipleksējot ierīces atkarībā no pieprasītās adreses.
		(sk.~\ref{sec:mmu}~nod.)\pagebreak[1]
	\item[Aparatūras kontrolēta SPI saskarne] \hfill \\
		Atšķirībā no citiem mikrokontrolieriem ar SPI saskarni
		[\todo ], % FIXME: Citation needed
		šis mikro\-kontrolieris nodrošina aparatūras kontrolētu datu pārraidi,
		atšķirībā no programmatūras kontrolētu pārraidi ar
		\termEn{bit-banging} metodi%
		\footnote{Metode kurā takts signāls tiek ģenerēts izpildot programmas
			instrukcijas, pārraidot vienu bitu vienā vai vairākos kodola takts ciklos.}.
		Tādējādi tiek vienkāršots programmkods SPI pārraidēm, un atbrīvots
		kodols citu darbību izpildei.
	\item[] \todo
\end{description}
