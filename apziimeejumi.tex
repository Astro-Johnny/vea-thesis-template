\section*{Saīsinājumu un nosacīto apzīmējumu saraksts}
\addcontentsline{toc}{section}{Saīsinājumu un nosacīto apzīmējumu saraksts}
Darbā izmantoti dažādi termini, saīsinājumi un apzīmējumi, kuri īsi
izskaidroti un/vai atšifrēti sekojošā sarakstā. Papildus šim sarakstam,
termini un saīsinājumi ir arī paskaidroti tekstā.\\[\parskip]

\noindent%
\begin{tabularx}{\textwidth}{lX}
	\emph{CPU}           & mikroprocesors vai centrālais procesors
	                       (sk.~\ref{sec:cpu}~nod.);\\[1ex]
	\emph{deskriptors}   & informācija par punkta apkārtni, kas ir
	                       invarianta noteikta veida transformācijām
	                       (sk.~\ref{sec:matching}~nod.);\\[1ex]
	\emph{FAST}          & Rostena~un~Dramonda\cite{FAST} stūru detektēšanas
	                       algoritms
	                       (sk.~\ref{sec:fast}~nod.);\\[1ex]
	\emph{FPGA}          & pārprogrammējams loģisko elementu masīvs, jeb
	                       vienkāršoti --- pārprogrammējamas funkcionalitātes
	                       mikroshēma (sk.~\ref{sec:fpga}~nod.);\\[1ex]
	\emph{GPGPU}         & vispārēju skaitļošanas uzdevumu izpilde ar
	                       GPU~\cite{Owens-GPU};\\[1ex]
	\emph{GPU}           & grafiskais (video kartes) procesors
	                       (sk.~\ref{sec:gpu}~nod.);\\[1ex]
	\emph{ORB}           & Rublē~u.c.\cite{ORB} raksturpunktu pāru
	                       noteikšanas algoritms
	                       (sk.~\ref{sec:orb}~nod.);\\[1ex]
	\emph{latentums}     & laika periods, starp datu pieprasījumu un to saņemšanu;\\[1ex]
	\emph{PC}            & personālais dators;\\[1ex]
	\emph{raksturpunkts} & attēla punkts (parasti kāda objekta stūris) kura
	                       apkārtne ir (ļoti vēlams) unikāli identificējama
	                       (sk.~\ref{sec:algo}~nod.);\\[1ex]
	\emph{raksturpunktu pāris} & punkti dažādos attēlos, kuri atbilst vienam
	                            attēla objektam, vai tā daļai;\\[1ex]
	\emph{RAM}           & operatīvā atmiņa;\\[1ex]
	\emph{rasterizēšana} & attēla atveidošana ar pikseļiem;
\end{tabularx}

