%% Abstracts
% The 'abstract' enviroment makes more problems than it is useful

\phantomsection\addcontentsline{toc}{section}{\abstractname}
\abstitlestyle{\abstractname} % The abstract title (centered)
% Abstract text goes here
\noindent%
\begin{tabularx}{\textwidth}{lX}
	\textbf{Darba nosaukums:} & 
		\textit{Iekļautās sistēmas mikrokontroliera kodola izstrāde}\\[1ex]
	\textbf{Darba autors:} & Jānis Šmēdiņš\\[1ex]
	\textbf{Darba vadītājs:} & Mg.~sc.~comp.~Gatis Gaigals\\[1ex]
	\textbf{Darba apjoms:} & 68~lpp., 3~tabulas, 26~attēli,
		12~pirmkoda izdrukas, 19~bibliogrāfiskie avoti, 5~pielikumi\\[1ex]
	\textbf{Atslēgas vārdi:} & Mikrokontroliera kodols, aparatūras apraksta valodas,
		sintezējams mikrokontrolieris
\end{tabularx}

\vspace{1em}
Šī bakalaura darba mērķis ir izstrādāt modulāru mikrokontroliera kodolu,
kurš izmantojams dažādu, specializētu mikrokontrolieru implementācijās,
kuras bieži nepieciešamas uzdevuma specifiskos elektroniskajos risinājumos,
galvenokārt, zinātniskajos projektos. Mikrokontroliera, un tādējādi arī
izstrādājamā kodola mērķa platforma ir FPGA.
%FIXME: Saīsinājuma atšifrējumu vajag uzrādīt?

Izstrāde veikta izmantojot aparatūras apraksta valodas, kas ir galvenais
FPGA projekta izstrādes instruments. Darbā analizēta eksistējoša procesora
arhitektūra, no kura, veicot pielāgojumus, identificējot arhitektūras 
trūkumus un izvirzot to risinājumus, izveidota bāzes arhitektūra pēc kuras
realizēts izstrādājamais mikrokontroliera kodols.

Šī darba laikā kodols tika veiksmīgi izstrādāts un tā pārbaudei un 
demonstrācijai izstrādāts arī parauga mikrokontrolieris, kurā šis kodols
izmantots. Izvirzīti arī vairāki priekšlikumi, kuri uzlabo izstrādāto
mikrokontroliera kodolu, bet nav fundamentāli izvirzīto mērķu sasniegšanai.

\clearpage
\begin{english} % The English abstract
	\phantomsection\addcontentsline{toc}{section}{\abstractname}
	\abstitlestyle{\abstractname} % The abstract title (centered)
	% Abstract text goes here
	\noindent%
	\begin{tabularx}{\textwidth}{lX}
		\textbf{Title:} & 
			\textit{Development of embedded system microcontroller core}\\[1ex]
		\textbf{Author:} & Jānis Šmēdiņš\\[1ex]
		\textbf{Supervisor:} & Mg.~sc.~comp.~Gatis Gaigals\\[1ex]
		\textbf{Scope:} & ??~pages, ??~tables, ??~figures, ??~listings,
			??~bibliographical references, ?~appendices\\[1ex]
		\textbf{Keywords:} & Microcontroller core, hardware description languages,
				synthesisable microcontroller
	\end{tabularx}
	
	\vspace{1em}
	This bachelors thesis purpose is to develop a microcontroller core for
	synthesis on FPGA, as part of microcontroller, whether stand-alone or
	control device within special purpose integrated circuit.
	The thesis also focuses on hardware description languages
	(HDL) as a main tool of attaining this goal.
	
	The author takes a more practical approach by developing first and
	addressing problems as they arise. The work used an existing
	architecture of central processing unit as base which was then improved
	and modified in successive, iterative development cycles
	(dubbed revisions), where each such cycle is to result in a working
	prototype.
	
	During the development the instruction set of core was completely
	revised and several notable architectural changes done, most evidently
	the internal data bus was reimplemented as one-way, branched, circular
	pipeline. The author successfully developed a synthesisable
	microcontroller core which is the result of third development cycle (rev.~03).
	For testing	and demonstration purposes a complete microcontroller prototype
	incorporating the core was developed as well.
	
	The author proposes several possible improvements for a
	potential next development cycle, but also acknowledging that there
	could be indefinite number of such cycles, and that one fulfilling
	requirements will suffice.
	
	%%%%%%%%%%%%
	%This bachelors thesis focuses on an existing architecture of 
	%central processing unit	prototype. The author analyses the architecture,
	%identifying design flaws and suggesting solutions to these flaws.
	%These solutions are put to practical use by developing a
	%synthesisable microcontroller core (essentialy a CPU),
	%improving upon the existing	architecture.
	
	%To demonstrate use of the developed microcontroller core, a prototype
	%microcontroller is developed as well, which incorporates the core.
	
	%Hardware description languages are analysed as well, of which the VHDL
	%is used as a main tool of development. A short introduction and
	%analysis is done on FPGAs, which is the target platform for design
	%synthesis.
\end{english}

\clearpage
\begin{russian} % The Russian abstract
	\phantomsection\addcontentsline{toc}{section}{\abstractname}
	\abstitlestyle{\abstractname} % The abstract title (centered)
	% Abstract text goes here
	
	\todo \\
	%Чо такоё? 13
\end{russian}
