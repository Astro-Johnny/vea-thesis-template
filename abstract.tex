%% Abstracts
% The 'abstract' enviroment makes more problems than it is useful

\phantomsection\addcontentsline{toc}{section}{\abstractname}
\abstitlestyle{\abstractname} % The abstract title (centered)
% Abstract text goes here
\noindent%
\begin{tabularx}{\textwidth}{lX}
	\textbf{Darba nosaukums:} & 
		\textit{Iekļautās sistēmas mikrokontroliera kodola izstrāde}\\[1ex]
	\textbf{Darba autors:} & Jānis Šmēdiņš\\[1ex]
	\textbf{Darba vadītājs:} & Mg.~sc.~comp.~Gatis Gaigals\\[1ex]
	\textbf{Darba apjoms:} & ??~lpp., ??~tabulas, ??~attēli,
		??~pirmkoda izdrukas, ??~bibliogrāfiskie avoti, ?~pielikumi\\[1ex]
	\textbf{Atslēgas vārdi:} & Mikrokontroliera kodols, aparatūras apraksta valodas,
		sintezējams mikrokontrolieris, \todo{}?
\end{tabularx}

\vspace{1em}
Šī bakalaura darba mērķis ir izstrādāt mikrokontroliera
kodolu, kurš izmantojams gan vispārēja pielietojuma mikrokontrolieru,
gan specializētu integrēto mikrokontrolieru sastāvdaļa sintēzei uz FPGA
platformas. Tiek arī apskatītas aparatūras apraksta valodas,
kā galvenais izstrādes rīks.

Autors izmanto visai praktisku pieeju, risinot un aprakstot problēmas tad,
kad tās rodas izstrādes procesā. Darbā apskatīta konkrēta procesora
arhitektūra, uz kura bāzes veikta kodola izstrāde iteratīvos soļos
(sauktas par revīzijām), katrā
uzlabojot prototipu un risinot sastaptās problēmas.

Izstrādes laikā tika ieviesta jauna kodola instrukciju kopa un veiktas
vairākas arhitektūras izmaiņas, no kurām ievērojamākā ir trīsstāvokļu datu šinas
pārveidošana sazarotā, vienvirziena, cirkulārā datu apmaiņas šinā.
Darba rezultātā tika izstrādāts sintezējams mikrokontroliera kodols, kas ir
trešās revīzijas (rev.~03) galavariants. Papildus izstrādāts arī
mikrokontroliera, kurā izmantots izstrādātais kodols.
Šis mikrokontrolieris izmantots kodola pārbaudei un darbības demonstrēšanai,
un ir izmantojams par paraugu mikrokontrolieru realizācijai, kas izmanto
šajā darbā izstrādāto kodolu.

Autors darba noslēgumā izvirza vairākus iespējamus uzlabojumus potenciālajam,
nākamajam izstrādes ciklam, bet secina ka šādi izstrādes cikli var būt
bezgalīgi, un nav nepieciešamība uzsākt nākamo, ja izvirzītās prasības
tiek pilnībā apmierinātas.



%%%%%%%%%
%Šajā bakalaura darbā ir apskatīta konkrēta procesora prototipa arhitektūra un
%analizēta tās nianses un trūkumi izvirzot problēmu risinājumus un uzlabojumus.
%Autors patsāvīgi izstrādājis sintezējamu mikrokontroliera kodolu, daļēji balstoties uz
%apskatītā procesora arhitektūru, iestrādājot arī izvirzītos uzlabojumus.
%Papildus izstrādāts arī parauga mikrokontroliera prototips,
%kurā šis kodols izmantots.

%Darbā arī apskatītas aparatūras apraksta valodas, no kurām VHDL izmantots kā galvenais
%izstrādes instruments, un FPGA, kas ir izstrādātā mikrokontroliera fiziskās
%realizācijas mērķa platforma.
%%%%%%%%%

\clearpage
\begin{english} % The English abstract
	\phantomsection\addcontentsline{toc}{section}{\abstractname}
	\abstitlestyle{\abstractname} % The abstract title (centered)
	% Abstract text goes here
	\noindent%
	\begin{tabularx}{\textwidth}{lX}
		\textbf{Title:} & 
			\textit{Development of embedded system microcontroller core}\\[1ex]
		\textbf{Author:} & Jānis Šmēdiņš\\[1ex]
		\textbf{Supervisor:} & Mg.~sc.~comp.~Gatis Gaigals\\[1ex]
		\textbf{Scope:} & ??~pages, ??~tables, ??~figures, ??~listings,
			??~bibliographical references, ?~appendices\\[1ex]
		\textbf{Keywords:} & Microcontroller core, hardware description languages,
				synthesisable microcontroller
	\end{tabularx}
	
	\vspace{1em}
	This bachelors thesis purpose is to develop a microcontroller core for
	synthesis on FPGA, as part of microcontroller, whether stand-alone or
	control device within special purpose integrated circuit.
	The thesis also focuses on hardware description languages
	(HDL) as a main tool of attaining this goal.
	
	The author takes a more practical approach by developing first and
	addressing problems as they arise. The work used an existing
	architecture of central processing unit as base which was then improved
	and modified in successive, iterative development cycles
	(dubbed revisions), where each such cycle is to result in a working
	prototype.
	
	During the development the instruction set of core was completely
	revised and several notable architectural changes done, most evidently
	the internal data bus was reimplemented as one-way, branched, circular
	pipeline. The author successfully developed a synthesisable
	microcontroller core which is the result of third development cycle (rev.~03).
	For testing	and demonstration purposes a complete microcontroller prototype
	incorporating the core was developed as well.
	
	The author proposes several possible improvements for a
	potential next development cycle, but also acknowledging that there
	could be indefinite number of such cycles, and that one fulfilling
	requirements will suffice.
	
	%%%%%%%%%%%%
	%This bachelors thesis focuses on an existing architecture of 
	%central processing unit	prototype. The author analyses the architecture,
	%identifying design flaws and suggesting solutions to these flaws.
	%These solutions are put to practical use by developing a
	%synthesisable microcontroller core (essentialy a CPU),
	%improving upon the existing	architecture.
	
	%To demonstrate use of the developed microcontroller core, a prototype
	%microcontroller is developed as well, which incorporates the core.
	
	%Hardware description languages are analysed as well, of which the VHDL
	%is used as a main tool of development. A short introduction and
	%analysis is done on FPGAs, which is the target platform for design
	%synthesis.
\end{english}

\clearpage
\begin{russian} % The Russian abstract
	\phantomsection\addcontentsline{toc}{section}{\abstractname}
	\abstitlestyle{\abstractname} % The abstract title (centered)
	% Abstract text goes here
	
	\todo \\
	%Чо такоё? 13
\end{russian}
