%% Abstracts
% The 'abstract' enviroment makes more problems than it is useful

\phantomsection\addcontentsline{toc}{section}{\abstractname}
\abstitlestyle{\abstractname} % The abstract title (centered)
% Abstract text goes here
\noindent%
\begin{tabularx}{\textwidth}{lX}
	\textbf{Darba nosaukums:} & 
		\textit{Iekļautās sistēmas mikrokontroliera kodola izstrāde}\\[1ex]
	\textbf{Darba autors:} & Jānis Šmēdiņš\\[1ex]
	\textbf{Darba vadītājs:} & Mg.~sc.~comp.~Gatis Gaigals\\[1ex]
	\textbf{Darba apjoms:} & 50~lpp., 3~tabulas, 26~attēli,
		12~pirmkoda izdrukas, 19~bibliogrāfiskie avoti, 5~pielikumi\\[1ex]
	\textbf{Atslēgas vārdi:} & Mikrokontroliera kodols, aparatūras apraksta valodas,
		sintezējams mikrokontrolieris
\end{tabularx}

\vspace{1em}
Šī bakalaura darba mērķis ir izstrādāt modulāru mikrokontroliera kodolu,
kurš izmantojams dažādu, specializētu mikrokontrolieru implementācijās,
kuras bieži nepieciešamas specifiskos elektroniskajos risinājumos,
galvenokārt, zinātniskajos projektos. Mikrokontroliera un
izstrādātā kodola mērķa platforma ir FPGA.

Izstrāde veikta izmantojot aparatūras apraksta valodas, kas ir galvenais
FPGA projekta izstrādes instruments. Ir veikts divu galveno aparatūras
apraksta valodu (VHDL un Verilog) salīdzinājums, lai pamatotu izstrādes valodas izvēli.
Darbā analizēta eksistējoša procesora
arhitektūra, identificēti tās
trūkumi un izstrādāti uzlabošanas risinājumi, kā rezultātā izveidota 
bāzes arhitektūra, pēc kuras
realizēts izstrādājamais mikrokontroliera kodols.

Šī darba laikā kodols tika veiksmīgi izstrādāts, tā pārbaudei un 
demonstrācijai izstrādāts arī parauga mikrokontrolieris, kurā šis kodols
izmantots. Papildus izstrādāts asemblerkoda translators.
Izvirzīti vairāki priekšlikumi, kā uzlabot izstrādāto
mikrokontroliera kodolu.

\clearpage
\begin{english} % The English abstract
	\phantomsection\addcontentsline{toc}{section}{\abstractname}
	\abstitlestyle{\abstractname} % The abstract title (centered)
	% Abstract text goes here
	\noindent%
	\begin{tabularx}{\textwidth}{lX}
		\textbf{Title:} & 
			\textit{Development of embedded system microcontroller core}\\[1ex]
		\textbf{Author:} & Jānis Šmēdiņš\\[1ex]
		\textbf{Supervisor:} & Mg.~sc.~comp.~Gatis Gaigals\\[1ex]
		\textbf{Extent:} & 50~pages, 3~tables, 26~figures, 12~listings,
			19~bibliographical references, 5~appendices\\[1ex]
		\textbf{Keywords:} & Microcontroller core, hardware description languages,
				synthesisable microcontroller
	\end{tabularx}
	
	\vspace{1em}
	This purpose of this bachelor's thesis is to develop a modular 
	microcontroller core that can be used in 
	different kinds of specialised microcontroller implementations,
	which are often needed for custom electronic solutions in various projects,
	scientific projects in particular.
	
	The development was done mostly by use of hardware description
	language which is an instrumental part of FPGA project development.
	A comparison of two industry standard hardware description languages
	(VHDL and Verilog) was carried out to justify the choice of language for development.
	An existing processor architecture was analysed, it's flaws identified
	and improvements devised resulting in base architecture upon which
	the core was developed.
	
	During the work the core was successfully completed, for it's testing
	and demonstration a example microcontroller incorporating the core
	was developed as well. In addition an assembler for the core was written.
	Several further improvements to the core were proposed at the conclusion
	of the work.
\end{english}

\clearpage
\begin{russian} % The Russian abstract
	\phantomsection\addcontentsline{toc}{section}{\abstractname}
	\abstitlestyle{\abstractname} % The abstract title (centered)
	% Abstract text goes here
	\noindent%
	\begin{tabularx}{\textwidth}{lX}
		\textbf{Название работы:} & 
			\textit{Разработка ядра микроконтроллера во встроенной системы}\\[1ex]
		\textbf{Автор работы:} & Jānis Šmēdiņš (транслит.~Янис Шмэдиньш)\\[1ex]
		%\textbf{Руководитель работы:}
		\textbf{Руководитель:} & Mg.~sc.~comp.~Gatis Gaigals 
			(транслит.~Гатис Гайгалс)\\[1ex]
		\textbf{Размер:} & 50 стр., 3 таблицы, 26 изображений,
			12 листингов, % (печатей исходного кода),
			19 библиографических источников, 5 приложений\\[1ex]
		\textbf{Ключевые слова:} & ядро микроконтроллера,
			языки описания аппаратуры, синтезируемый микроконтроллер
	\end{tabularx}
	
	\vspace{1em}
	Цель работы бакалавра разработать модулярное ядро микроконтроллера
	которого можно использовать в разных специализированных имплементациях
	микроконтроллеров которые часто нужны в специфичных электронных решениях,
	в основном, в научных проектах. Целевой платформой микроконтроллера и
	разработанного ядра является FPGA.

	Разработка сделана, используя языки описания аппаратуры, которые
	являются главным инструментом проекта FPGA. Сделано сравнение двух
	главных языков описания аппаратуры (VHDL и Verilog), что-бы основать
	выбор языка разработки. В работе анализируется архитектура существующего
	процессора, идентифицированы недостатки и разработаны способы улучшения,
	в результате сделана базовая архитектура, по которой реализовано
	ядро разрабатываемого микроконтроллера.

	Во время этой работы ядро было успешно разработано, для его
	проверки и демонстрации  тоже разработан образцовый микроконтроллер,
	в котором это ядро будет использоваться. Дополнительно разработан
	транслятор кода ассемблера. Выдвинуто несколько предложений,
	как улучшить разрабатываемое ядро микроконтроллера.
	
\end{russian}
