%\section{Mikrokontrolieri} %TODO: Normālu nosaukumu?
%\todo

%Dzirdot vārdu ,,mikroprocesors'' un ,,datorsistēma'' vairums cilvēku 
%iztēlojas personālos datorus un klēpjdatorus, kas arī pēdējos gados ir teju
%vai katram. Tā gan ir tikai --- tēlaini izsakoties --- aisberga redzamā
%daļa. Mūsdienās mikroprocesori atrodami ļoti plaša klāsta dažādās sadzīves
%un industriālās ierīcēs, piem.~automašīnu elektronikā, ražošanas iekārtās,
%datortīklu maršutētājos, elektromēriekārtās, mobilajos telefonos,
%veļas mašīnu kontroles blokos un pat rotaļlietās.\cite[1.~lpp.]{Heath}

%% TODO: Šito mošk vajag pārfrāzēt
%Iekļautā sistēma arī ir datorsistēma, kuras primārā atšķirība no 
%vispārēja pielietojuma datorsistēmas ir darba specializācija,
%no kā seko praktiski visas realizācijas atšķirības. \todo

%TODO?: Procesors

\section{Aparatūras apraksta valodas}
Šajā darbā mikrokontroliera kodols un paraugimplementācija izstrādāta
galvenokārt izmantojot aparatūras apraksta valodu, un tādēļ lai radītu
priekšstatu par darba metodiku, šī nodaļa apskata 
aparatūras apraksta valodas, to īpašības un pielietojumu.

Elektronikai attīstoties un paaugstinoties tirgus prasībām produkta funkcionalitātei,
izstrādājamās sistēmas kļūst aizvien sarežģītākas. Jo sevišķi komplicētu ciparu shēmu, 
kā piemēram mikroprocesoru, iekšējo uzbūvi izstrādāt tradicionālā ceļā,
shematiski attēlojot komponentes un to savstarpējos savienojumus, kļūst
nepraktiski. Aparatūras apraksta valodas, savukārt, piedāvā augstākas
abstrakcijas izstrādes modeli, kas saīsina izstrādes laiku [\todo{}?].

Aparatūras apraksta valoda jeb HDL
(no angļu \termEn{Hardware description language})
ir valoda, ar kuras izteiksmēm, sekojot konkrētās 
valodas sintaktiskajiem nosa\-cī\-jumiem, ir iespējams aprakstīt
izstrādājamās shēmas uzbūvi un darbību \cite{HDL}. %Šo aprakstu dēvē par ,,kodu''.
HDL gandrīz vienmēr (bet ne obligāti) apraksta ciparu shēmas.
Līdzīgi program\-mē\-šanas valodām, HDL piedāvā dažādas sintaktiskās 
konstrukcijas, kas ļauj strukturēt aprakstu un veikt abstrakcijas,
ar kurām iespējams kompakti aprakstīt sarežģītas shēmas
\cite[1.~lpp.]{Perry-VHDL}.
Savukārt, atšķirībā no vairuma programmēšanas valodu,
izpildāmās izteiksmes ir ,,konkurentas'', t.i.,~tās tiek izpildītas paralēli, 
tādējādi shēmas definējošā apgabala izteiksmju secība ir maznozīmīga%
\footnote{Izteiksmju secībai ir nozīme secīgajās konstrukcijās.
	(sk.~\ref{sec:hdl-styles}~nod.)}.

HDL apraksti ir mašīn\-apstrādājami, kas dod iespēju izmantot
HDL simulācijas rīku, lai varētu veikt HDL aprakstītās shēmas 
darbības simulāciju, t.i.,~programmatūras līmenī 
atveidota aprakstītās shēmas darbība, no kuras iegūstamas jebkuru
shēmas signālvadu laika oscilogrammas, atmiņas elementu saturs un to izmaiņas,
kā arī iespējams citi dati. Analizējot simulācijas rezultātus ir 
iespējams pārbaudīt aprakstītās shēmas darbības korektumu pirms
tās fiziskās realizācijas. Iespēja simulēt HDL aprakstu ir ievērojama
priekšrocība izstrādei ar HDL.

Tomēr galvenais iemesls HDL izmantošanai un popularitātei ir iespējai veikt 
shēmu sintēzi --- automatizētu fiziskās realizācijas ģenerēšanu pēc HDL apraksta
\cite{HDL}\cite{Perry-VHDL}\cite{Vahid-RTL}. Tieši HDL aprakstu sintezējamība ir
īpašība, kas padara izstrādi ar HDL rūpnieciski nozīmīgu.
HDL pieraksta nianses sintēzei tiks apskatītas sekojošā \ref{sec:hdl-styles}~%
nodaļā, bet konkrētās sintēzei izmantotās tehnoloģijas --- 
\ref{sec:synth}~nodaļā (\pageref{sec:synth}~lpp.).

Eksistē daudz aparatūras apraksta valodu, gan valodas, kas izveidotas
speciāli aparatūras aprakstam, gan speciālas pakotnes, kuras pielāgo
eksistējošas programmēšanas valodas aparatūras aprakstam.
Tomēr nozarē svarīgākās ir VHDL un ,,Verilog'', kas ir arī
vienas no vecākajām un visvairāk attīstītajām HDL.
Šīs valodas īsumā tiek apskatītas \ref{sec:vhdl}~un \ref{sec:verilog}~%
nodaļā, un to salīdzinājums --- \ref{sec:hdl-comparison} nodaļā.

\subsection{Aprakstu stili} \label{sec:hdl-styles}
HDL vienādas komponentes var aprakstīt dažādos veidos, bet
HDL aprakstus var sadalīt divos pieraksta stilos, pēc problēmas pieejas un
no valodas konstrukciju kopas kas šai pieejai raksturīga.
\begin{enumerate}
	\item \textbf{Strukturālais} pieraksta stils --- komponentes tiek
		izveidotas izmantojot vienkāršākas apakškomponentes un to savstarpējos
		savienojumus. Šis stils uzskatāms par tekstuālu analogu
		tradicionālajam, shematiskajam izstrādes veidam.
	\item \textbf{Funkcionālās} vai izturēšanās modelēšanas pieraksta
		stils --- komponentes tiek izveidotas aprakstot tās funkcionalitāti
		abstrahējoties no tās iespējamās uzbūves.
	\begin{itemize}
		\item \textbf{Secīgās izturēšanās} pieraksta stils --- apakškopa
			no funkcionālā stila, kurā izmantotas konstrukcijas, kas ļauj
			komponentes darbību aprakstīt ar secīgām izpildāmām izteiksmēm,
			līdzīgi imperatīvām programmēšanas valodām (kā C, Python, u.c.),
			pretstatā paralēli izpildāmām datu plūsmas izteiksmēm.
	\end{itemize}
\end{enumerate}
\pagebreak[3]

Šie stili ne tikai ietekmē koda (konkrētā shēmas apraksta teksta) pierakstu,
bet arī simulācijas un sintēzes procesus, kuriem šis pieraksts ir jāinterpretē.
Strukturālā pieraksta un tādu funkcionālā pieraksta primitīvu konstrukciju,
kā loģisko elementu (\texttt{UN}, \texttt{VAI}, utt.),
sintēze ir tieši translējama uz aparatūras komponentēm un to savienojumiem.
Savukārt sarežģītu funkcionālo konstrukciju, jo sevišķi secīgo
konstrukciju, sintēzei nepieciešams translēt funkcionalitātes aprakstu
uz konkrētu aparatūras implementāciju, t.i.~sintēzes rīkam nepieciešams
piemeklēt aparatūras komponentes kas realizē konkrēto funkcionalitāti.
Tā kā funkcionalitātes un implementācijas saistība bieži nav viennozīmīga,
dažādu sintēzes rīku interpretācija vienādam kodam var stipri atšķirties.

Praktiski visi sintēzes rīki funkcionālo pierakstu translē %uz aparatūras 
reģistru datu pārraides abstrakcijas līmenī jeb RTL
(no angļu \termEn{Register tranfer level}).\cite[2.~lpp.]{HDL}%
\cite[235.~lpp.]{Perry-VHDL}
RTL abstrakcijas vienību (elementu) veido kombinacionālās datu transformācijas shēma un
datu uzglabāšanas reģistrs (sk.~\ref{fig:rtl}~att.).
\begin{figure}[thb]
	\centering
	%\def\svgwidth{7cm}
	\def\svgscale{1.25}
	\input{img/rtl-unit.pdf_tex}
	\caption[RTL abstrakcijas vienība.]{RTL abstrakcijas vienība.\cite[233.~lpp.]{Perry-VHDL}}
	\label{fig:rtl}
\end{figure}
Izstrādājot ierīci sintēzei ar HDL kodu, tiek bieži pieminēts
,,RTL apraksta stils''. Šis stils efektīvi ir HDL konstrukciju kopa, kas
pakļaujas sintēzei, un pārklāj visas strukturālā stila konstrukcijas, kā arī
lielu daļu funkcionālā stila konstrukcijas.

Jāuzsver gan, ka RTL pieraksta stils nepieprasa, lai katrai aprakstītai
komponentei būtu piekārtojams tieši viens RTL elements. Vienas komponentes
apraksts var saturēt vairākus RTL elementus, kā arī viens RTL elements var
pārklāt vairākas komponentes. Kā piemēru pēcākajam variantam, var apskatīt
darba praktiskajā daļā izstrādāto mikrokontroliera datu kalkulācijas
signālceļu (sk.~\ref{fig:rtl-alu}~att.).
\begin{figure}[thb]
	\centering
	\def\svgwidth{0.9\textwidth}
	%\def\svgscale{1.25}
	{\ttfamily\scriptsize\input{img/rtl-alu.pdf_tex}}
	\caption{RTL sadalījuma piemērs ALU signālceļam.}
	\label{fig:rtl-alu}
\end{figure}
Šeit \texttt{alu} (Aritmētiski loģiskā ierīce) un \texttt{shift}
(Bitu bīdes loģiskā ierīce) ir kombinacionālas shēmas, bez iekšējiem atmiņas
elementiem, savukārt \texttt{reg} (Reģistrs), kura uzdevums ir saglabāt kalkulācijas
rezultātu, acīmredzami, ir reģistra jeb atmiņas daļa šajā RTL elementa piemērā.
Tādējādi šīs trīs komponentes sastāda vienu RTL elementu.

Konstrukcijas, kuras nav translējamas RTL, sauc par --- visai aprakstoši ---
,,tikai simulējamām konstrukcijām''. Tās nav sintezējamas un tiek
izmantotas simulācijas ietvaros. Viena no vienkāršākajām un biežāk
izmantotajām ir novēlotā signāla piešķire, piem.~VHDL izteiksme 
|q <= a after 10 ns;| piešķir |a| vērtību signālam |q| pēc 10 nanosekundēm.
Tas ir labs veids, kā simulēt komponenšu aiztures, bet tiks ignorēta
sintēzē, jo elementiem uz kuriem tiks realizēta shēma ir konkrēta aizture,
ko nosaka tā fiziskā uzbūve.

Svarīgi piezīmēt, ka pretstatā sintēzes rīkam, simulatoram nav nepieciešama
pieraksta translācija RTL. Tā kā simulācija notiek
programmatūras līmenī, vairumā gadījumu secīgās izturēšanās modeli
simulatoram ir vienkāršāk (un tādējādi arī ātrāk) simulēt nekā paralēli
izpildāmās konkurentās konstrukcijas.
 \clearpage %\pagebreak[3]
\subsection{VHDL} \label{sec:vhdl}
	VHDL (no angļu \termEn{VHSIC hardware description language}) ir aparatūras apraksta
	valoda, kura ieviesta 1980-ajos gados Amerikas Savienoto Valstu
	Aizsardzības ministrijas
	VHSIC (\termEn{very-high-speed integrated circuits})
	projekta ietvaros \cite[141.~lpp.]{VHSIC}\cite[1.~lpp.]{Perry-VHDL}.

	VHDL sintakse ir modelēta pēc Ada programmēšanas valodas sintakses, kur
	valodas konstrukcijas un to semantiskā nozīme pielāgota aparatūras aprakstam.
	VHDL sintakse tiek uzskatīta par visai izsmeļošu, t.i.,~valodas 
	konstrukcijas un izteiksmes ir pierakstāmas visai gari, kas seko no
	Ada sintakses, kā arī fakta, ka VHDL sākotnēji bija paredzēta kā 
	formāla valoda komponenšu dokumentācijai, kas viennozīmīgi aprakstītu
	komponenšu darbību.
	VHDL apraksti papildināja komponenšu dokumentāciju,
	aizstājot komplicēto un, bieži vien,
	nekonkrēto vai neviennozīmīgo rakstveidā dokumentēto darbības modeli.
	
	Tā kā VHDL apraksti seko noteiktai striktai, viennozīmīgai sintaksei,
	tie ir mašīnlasāmi un tika izstrādāta programmatūra, kas deva
	iespēju veikt aprakstītās shēmas simulāciju un vēlāk arī sintēzi.

	\begin{lstlisting}[language={[qucs]VHDL},label=kb:vhdl-example,gobble=4,%
			caption={VHDL apraksts loģiskajam \texttt{UN} elementam.}]
		-- This adds IEEE library and nine-valued logic
		library ieee;
		use ieee.std_logic_1164.all;
		
		entity andgate is
			port ( A : in std_logic;
			       B : in std_logic;
			       Q : out std_logic );
		end entity andgate;
		
		architecture rtl of andgate is
		begin
			Q <= A and B; -- Simply 'AND' the inputs and output result
		end architecture rtl;
	\end{lstlisting}

	Vienkāršs loģiskā elementa \texttt{UN} VHDL apraksta paraugs redzams
	\ref{kb:vhdl-example}~pirmkoda izdrukā. Piemērā redzami pamata sintaktiskās
	konstrukcijas komponentes aprakstam.
	Pirmkārt, ,,entītija'' definē aprakstāmās komponentes ,,no ārpuses''
	redzamās daļas, no kurām svarīgākā (un piemērā vienīgā) ir pieslēgvietu
	saraksts; otrkārt, ,,arhitektūra'' ir komponentes implementācijas definīcija,
	kurā tiek aprakstīta ,,entītijā'' definētās komponentes darbība.

	Tā kā izmantojot aprakstīto komponenti hierarhiālā shēmas projektēšanā, 
	galvenokārt ir svarīga komponentes darbība, nevis tās konkrētā implementācija,
	VHDL apraksta arhitektūru var uzskatīt par ,,slēpto'' vai ,,privāto''
	komponentes daļu.

	VHDL iebūvētais \texttt{bit} tips (kas pieņem tikai |0| vai |1| vērtības)
	nevar parādīt tādus aparatūras loģikas stāvokļus, kā augstas impedances
	(|Z|) stāvoklis, atsaites rezistoru ,,vājos'' loģikas līmeņus un nezināmu
	(|X|) loģisko stāvokli. Šo problēmu atrisina IEEE 1164 standarts, kurš definē
	\texttt{std\_logic} tipu ar deviņiem dažādiem stāvokļiem:
	\begin{itemize}
		\item \verb|'U'| --- neinicializēta (nezināma) loģiskā vērtība;
		\item \verb|'X'| --- nezināma loģiskā vērtība no spēcīga devēja;
		\item \verb|'0'| --- loģiskā 0 no spēcīga devēja;
		\item \verb|'1'| --- loģiskais 1 no spēcīga devēja;
		\item \verb|'Z'| --- augstas impedances stāvoklis (nav devēja);
		\item \verb|'W'| --- nezināma loģiskā vērtība no vāja devēja;
		\item \verb|'L'| --- loģiskā 0 no vāja devēja (piem.,~atsaites rezistors pret zemi);
		\item \verb|'H'| --- loģiskais 1 no vāja devēja (piem.,~atsaites rezistors pret barošanas spriegumu);
		\item \verb|'-'| --- ignorējama vērtība (\termEn{don't care})%
			\footnote{Vērtību \verb|'-'| var izmantot salīdzināšanā,
				lai ignorētu konkrētus bitus no bitu vektora.},
	\end{itemize}
	kā arī definē operācijas ar \texttt{std\_logic} tipa signāliem vai mainīgajiem. 
	Praktiski visi VHDL simulācijas rīki šo bibliotēku piedāvā.
	
	Sintēzes rīki arī atbalsta \texttt{std\_logic} tipu, bet ne visas
	\texttt{std\_logic} tipa pieņemamās vērtības ir sintēzei saistošas.
	Reālu ciparu shēmu ieejas signāls vienmēr tiks interpretēts kā |0| (zems līmenis) 
	vai |1| (augsts līmenis), un izejas var pieņemt |0|, |1| vai augstas
	impedances (|Z|) stāvokli. Izvadīt nezināmu (|X|) loģisko vērtību
	fiziskai komponentei nav iespējams.	%\todo{}?
	% TODO: Uztais normālu nobeigumu

\pagebreak[3]
\subsection{Verilog} \label{sec:verilog}
	,,Verilog'' ir aparatūras apraksta valoda, kuru 1980-ajos gados izstrādāja
	\termEn{P.~Moorby} un \termEn{P.~Goel}. Verilog sintakse ir daļēji modelēta
	pēc C programmēšanas valodas sintakses, kas papildus faktam, ka Verilog
	jau sākotnēji izstrādāta shēmu simulācijai, nevis
	kā formāla shēmu apraksta valoda, padara Verilog pierakstu visai kompaktu.

	\begin{lstlisting}[language={Verilog},label=kb:verilog-example,gobble=4,%
			caption={Verilog apraksts loģiskajam \texttt{UN} elementam.}]
		module andgate(A, B, Q);
			input A;
			input B;
			output Q;
			// inputs and outputs are of 'wire' type
			// unless redeclared otherwise
			
			assign Q = A & B; // Simply 'AND' the inputs and output result
		endmodule
	\end{lstlisting}

	Vienkāršs loģiskā elementa \texttt{UN} Verilog apraksta paraugs redzams
	\ref{kb:verilog-example}~pirmkoda izdrukā.
	Verilog ,,modulis'' ir konstrukcija kas satur komponentes aprakstu
	--- gan ārējo saskarni, gan implementāciju.
	
	Verilog moduļus var izmantot citu moduļu definēšanā, tādējādi veicot
	hierarhiālu shēmas projektēšanu, bet Verilog neatbalsta pakotnes vai
	bibliotēkas konceptu, tādēļ moduļu otreizēja izmantošana citos projektos
	ir apgrūtināta.
	
	Verilog datu tipi, atšķirībā no VHDL, ne tikai izsaka
	to pieņemamās vērtības, bet arī to semantisko nozīmi shēmā
	\cite[21.~lpp.]{Vivek-Verilog}. Piemēram,
	gan \texttt{wire} (apakštips no \texttt{net}), gan \texttt{reg} tips
	satur četru stāvokļu loģikas datus, bet \texttt{wire} apzīmē
	apzīmē savienojumu mezglu (\termEn{net}), savukārt, \texttt{reg}
	vairumā gadījumu (bet ne obligāti) apzīmē atmiņas elementu (trigeri).
	Verilog četri loģiskie stāvokļi ir:
	\begin{itemize}
		\item |0| --- loģiskā 0, zems līmenis;
		\item |1| --- loģiskais 1, augsts līmenis;
		\item |z| --- augstas impedances stāvoklis;
		\item |x| --- nezināma loģiskā vērtība.
	\end{itemize}
	
	Verilog četru stāvokļu loģikas iespējamās vērtības neiekļauj
	,,vājos'' loģiskos stāvokļus, bet tos iespējams izteikt izmantojot
	\texttt{wire} tipa devēja stipruma uzstādījumu. \ref{kb:nand-oc}~kods
	parāda vienu no iespējamiem veidiem, kā aprakstīt \texttt{UN-NĒ}
	elementu ar atvērtā kolektora izeju, kuram pievienots atsaites rezistors
	pret barošanas avotu.
	\begin{lstlisting}[language={Verilog},label=kb:nand-oc,gobble=4,%
			caption={Verilog pieraksta paraugs devēja stipruma uzstādījumam.}]
		module nand_resistive_high(A, B, Q);
			input A;
			input B;
			output Q;
			// This emulates an open-collector output
			// with a pull-up resistor to Vcc
			wire (pull1, strong0) tmp;
			
			assign tmp = ~(A & B);
			assign Q = tmp;
		endmodule
	\end{lstlisting}
	
	Fakts, ka Verilog sākotnēji izstrādāts shēmu sintēzei, atstājis iespaidu
	uz valodas pieraksta sintezējamību. Verilog var aprakstīt sarežģītas
	sintezējamas shēmas ne sliktāk par citām HDL, bet eksistē virkne
	nosacījumu apraksta sintezējamības nodrošināšanai \cite{ieee-1364.1},
	kuri ir vairāk nekā citām HDL.
	

%\subsection[SystemC]{SystemC \textcolor{red}{???}}
	%\todo ?

\clearpage
\subsection{Valodu salīdzinājums} \label{sec:hdl-comparison}
	Ciparu elektronikas industrijā VHDL un Verilog ir visplašāk izmantotās
	HDL. Tā kā vairums FPGA un ASIC ražotāju (sk.~\ref{sec:synth}~nod.)
	piedāvātie izstrādes rīki atbalsta abas valodas, izvēles kritērijs
	valodas atbalstam ir kļuvis mazsvarīgs.
	
	Viena no būtiskākajām atšķirībām starp VHDL un Verilog ir datu tipu
	konversija. VHDL objektiem (pieslēgvietām, signāliem un mainīgajiem)
	piešķirama tikai vērtība, kas atbilst konkrētā objekta tipam
	(jāsakrīt gan tipam, gan vektoru garumiem)~\cite{vhdl-vs-verilog}.
	Ja nav tiešas atbilstības
	nepieciešams izmantot konversijas funkcijas vai citus pārvei\-do\-jumus,
	pat ja šo tipu sintēzes fiziskā realizācija ir identiska.
	Savukārt, Verilog
	dažāda garuma bitu vektori un \texttt{integer} tipa vērtības tiek
	savā starpā pārveidotas bez speciāla pieraksta pēc noklusējuma~\cite{vhdl-vs-verilog}.
	Abām pieejām ir 
	ieguvumu un trūkumi. ,,Nestingrajā'' tipu sistēmā iespējams īsāks un,
	iespējams, pārskatāmāks pieraksts, savukārt, ,,striktā'' tipu sistēma
	var novērst kļūdas, kas varētu rasties tipu pārveidošanas rezultātā,
	piem.,~bitu atmešana dažāda garuma bitu vektoru piešķirēs vai nekorekta
	zīmes bita interpretācija.
	
	\begin{figure}[thb]
		\centering
		\def\svgwidth{\textwidth}
		\input{img/hdl-abstractions.pdf_tex}
		\caption[VHDL un Verilog modelēšanas abstrakciju līmeņu salīdzinājums.]%
		        {VHDL un Verilog modelēšanas abstrakciju līmeņu salīdzinājums \cite{vhdl-vs-verilog}.}
		\label{fig:abstractions}
	\end{figure}
	
	Otra, ne mazāk būtiska, bet ne uzreiz redzama, atšķirība ir abstrakcijas
	līmeņi, kādos ar šim valodām var strādāt. VHDL un Verilog piedāvātais
	modelēšanas abstrakcijas līmeņu ,,spektrs'' ir nedaudz atšķirīgs
	(sk.~\ref{fig:abstractions}~att.) \cite{vhdl-vs-verilog}, kuru nosaka
	valodas sintaktiskās konstrukcijas. VHDL piedāvā dažādas konstrukcijas
	augstai abstrakcijai, t.sk.,~neatkarīgas, globālas apakšprogrammas;
	parametriski ģenerētas struktūras; lietotāja definētu abstraktu uzskaitāmo
	(\texttt{enumerate}) tipus \cite{Mealy-VHDL}\cite{Perry-VHDL}. Verilog nav konstrukciju šādiem valodas
	konceptiem. No otras puses, Verilog piedāvā konstrukcijas detalizētiem, zemas
	līmeņa abstrakcijas aprakstiem, piem.,~tranzistoru modeļi un
	pamata loģisko elementu modeļi ar detalizētiem signālaizturu un izejas
	līmeņu ,,spēcīguma'' parametriem \cite{vhdl-vs-verilog}\cite{Kumar-Verilog}.
	
	Lai korektāk būtu iespējams modelēt elementu signālaiztures ar VHDL,
	tika izveidots VITAL (\termEn{VHDL Initiative Toward ASIC Libraries})
	papildinājums \cite{VITAL}. VITAL definē pakotnes un apakšprogrammas,
	kuras ASIC un FPGA ražotāji izmanto lai konstruētu modeļus
	ASIC standartšunām (sk.~\ref{sec:asic}~nod.) un FPGA loģiskajiem blokiem
	(sk.~\ref{sec:fpga}~nod.), sastādot VITAL modeļu bibliotēkas.
	Izmantojot šīs bibliotēkas ir iespējams simulēt loģisko elementu līmeņa
	(\termEn{gate level}) shēmas modeli, kurai pēc izvietošanas un trasēšanas
	soļa ir zināmas reālās elementu signālaiztures, pretstatā shēmas 
	funkcionālai RTL simulācijai \cite{Perry-VHDL}.
	VITAL loģisko elementu līmeņa modeli praktiski vienmēr ģenerē izstrādes
	rīks no shēmas RTL apraksta, jo	
	lai gan VITAL ir
	lietotāja izmantojams, tas nav VITAL mērķis, un VHDL pieraksts ar VITAL
	ir ļoti izsmeļošs (un nepārskatāms).
	
	
	\begin{table}[thb]\small
		\centering
		\caption{VHDL un Verilog īpašību salīdzinājuma kopsavilkums.}
		\label{tbl:hdl-comparison}
		\begin{tabular}{p{0.2\textwidth}p{0.36\textwidth}p{0.36\textwidth}}
			\toprule
			~ & \textbf{VHDL} & \textbf{Verilog}\\ 
			\midrule
			\textbf{Sākotnējais\newline pielietojums} &
				Formāla komponenšu\newline dokumentēšana & Shēmu simulācija\\
			\midrule
			\textbf{Datu tipu\newline konversija} &
				Tikai ar speciālām konversijas funkcijām & 
				Vairumam tipu iespējama pēc\newline noklusējuma\\
			\midrule
			\textbf{Vairāku\newline stāvokļu loģika} &
				Ir, ar IEEE 1164 bibliotēku & Ir, iebūvēta\\
			\midrule
			\textbf{Pakotņu atbalsts} &
				Ir & Nav\\
			\midrule
			\textbf{Lietotāja definēti tipi} &
				Iespējami & Nav iespējami\\
			\midrule
			\textbf{Tranzistoru\newline modeļi} &
				Nav & Ir\\
			\midrule
			\textbf{Advancēti signālaizturu modeļi} &
				Ir (ar VITAL), bet nav domāti lietotāja izmantošanai &
				Ir, iebūvēti\\
			\bottomrule
		\end{tabular}
	\end{table}
	
	Kā redzams VHDL un Verilog valodu atšķirības ir daudz komplicētākas,
	nekā tikai shēmas modeļu pieraksts. Un atkarībā no projekta specifikas,
	katrai valodai eksistē objektīvi iemesli tās izmantošanai. Īss pārskats
	valodu īpašību salīdzinājumam redzams \ref{tbl:hdl-comparison}~tabulā.
	
	Autora izvēlētā HDL mikrokontroliera kodola izstrādei ir VHDL. Kā viens
	no objektīviem iemesliem ir VHDL augsta līmeņa abstrakcijas, kas
	izstrādājot mikrokontroliera kodolu var ievērojami uzlabot
	produktivitāti. Darbā tiek izmantotas paša definētas
	pakotnes un datu tipi, un parametriski ģenerētās struktūras. Šādām
	konstrukcijām Verilog nav analogu.
	
	


%TODO: SPI
