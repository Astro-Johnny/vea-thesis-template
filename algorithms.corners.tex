\subsection{Attēlu raksturpunktu atlases algoritmi} \label{sec:corners}
Attēlu raksturpunktu atlase ir punktu apakškopas atlase pēc noteiktiem
kritērijiem. Kā jau minēts nodaļas ievadā (\pageref{sec:algo}~lpp.),
raksturpunktu atlasei bieži tiek izmantoti ,,stūru'' meklēšanas algoritmi,
kādi ir arī visi šajā darbā apskatītie raksturpunktu atlases algoritmi.

Attēlu raksturpunktu atlasei ir vairāki pamatojumi, bet šī darba kontekstā
--- raksturpunktu salāgošanai --- būtiski ir atlasīt atbilstošos punktus
dažādos attēlos, kuru lokālajai apkārtnei konkrētajā attēlā piemīt
augsta variance. Sekundārs pamatojums raksturpunktu atlasei ir
skaitļošanas apjoma samazināšanai, jo turpmākā salāgošana tiek veikta
tikai atlasītajiem punktiem. Raksturpunktu atlase gan ir nepieciešama
datu priekšapstrāde, jo veicot salāgošanu visiem attēlu punktiem tiek
degradēta arī salāgošanas kvalitāte, ne tikai ātrdarbība, jo tiek
iekļauti daudz zemas variances punkti, kuri sastādīs
,,nepatiesi~pozitīvi'' salāgotu punktu pāru kopu, kas būs ievērojami
lielāka par derīgo rezultātu kopu, kā arī radīs neviennozīmīgus
konfliktējošus pārus,
t.i.~viena attēla vairāku punktu ,,atbilstība'' vienam (vai vairākiem)
punktiem otrā attēlā.
% TODO?: "Konfliktējošo" pāru attēls?

Eksistē vairāki stūru atrašanas algoritmi: \termEn{Harris} stūru detektors,
SUSAN, DoG~(SIFT), KLT, FAST, u.c.. 
Lai gan stūru atrašanas algoritmu skaits un to aprakstošā literatūra ir
visai liela, ļoti maz ir informācijas ir pieejama to salīdzināšanai. Pie
tam, aprakstītās metodes ne vienmēr ir adevkāti raksturojošas vai pat
objektīvas~\cite{Mokhtarian}\cite{FAST}. Autors par nozīmīgu un 
pietiekami kvantitatīvu īpašību atzīst 
Rostena~un~Dramonda\cite{FAST}~(\termEn{Rosten~and~Drummond}) piedāvāto
\newTerm{atkārtojamību}, kas novērtē punktu salāgojamību vairākos attēlos
%TODO: Vai atkārtojamība ņem vērā punktu salāgojamību???
no datu kopas, tādējādi izvērtējot algoritmus pēc to mērķa pielietojuma.
Tātad izvirzītās īpašības ir:
\begin{itemize}
	\item \emph{pozīcijas} un \emph{rotācijas invariance} --- spēja atlasīt
		raksturpunktus neatkarīgi no to koordinātēm un rotācijas attēlā;
	\item \emph{intensitātes nobīdes invariance} --- spēja atlasīt
		raksturpunktus neatkarīgi no globālas intensitātes izmaiņas;
	\item \emph{mēroga invariance} --- spēja atlasīt raksturpunktu
		neatkarīgi no mēroga;
	\item \emph{atkārtojamība}, kas apraksta izvēlēto punktu salāgojamību%
			\footnote{Pārbaudot ar raksturpunktu salāgošanas algoritmu.}
		vairākos attēlos no datu kopas, ko --- ne tik striktā terminoloģijā ---
		var arī uzdot kā ,,invarianci laikā'', ja datu kopas attēli ir
		uzņemti secīgi.
	\item \emph{trokšņu noturība} --- spēja atlasīt kvalitatīvus%
			\footnote{Spriežot pēc atkārtojamības rādītāja.}
		raksturpunktus attēlā (iespējami) neatkarīgi no ,,trokšņa'' līmeņa attēlā;
\end{itemize}

No uzskaitītajiem algoritmiem FAST stūru detektors ir
visātrākais~\cite{Rosten-tracking}\cite{FAST} izpildei uz CPU un
tas uzrāda līdzīgus vai labākus atkārtojamības rādītājus~\cite{FAST},
pretstatā citiem salīdzinātajiem stūru detektoriem. Iepriekšējā apakšnodaļā
izvēlētais raksturpunktu salāgošanas algoritms ORB par raksturpunktu atlases
komponenti izmanto FAST, kas arī izvēlēta
%FIXME: Nav izskaidrota lokālo maksimumu noteikšanas (non-max suppression)
turpmākai to algoritmu implementāciju salīdzināšanai
(sk.~\ref{sec:fast}~nod.).

Jāpiemin gan, ka FAST trokšņa noturība%
	\footnote{Pārbaudot mākslīgi pievienojot attēlam Gausa (\textit{Gauß}) troksni.}
atpaliek no DoG~(SIFT) un SUSAN, bet ir līdzvērtīga ar citiem
salīdzinātajiem detektoriem (t.sk.,~\termEn{Harris})~\cite{FAST}.
Autorprāt, FAST trokšņa noturība ir adekvāta, bet norāda, ka augsta signāla
trokšņa gadījumā, DoG vai SUSAN varētu būt labāka izvēle.

FAST, no iepriekšminētajām īpašībām, nepiemīt mēroga invariance, bet šo
trūkumu var novērst apstrādājot vienu attēlu vairākos mērogos (mēroga piramīda).
Šādu pieeju Rublē~u.c.\cite{ORB} (\termEn{Rublee~et~al.}) izmanto ORB algoritmā.

%TODO?: Smalkāk aprakstīt stūru/raksturpunktu detektoru darbību?
