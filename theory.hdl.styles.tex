\subsection{Aprakstu stili} \label{sec:hdl-styles}
HDL vienādas komponentes var aprakstīt dažādos veidos, bet
HDL aprakstus var sadalīt divos pieraksta stilos, pēc problēmas pieejas un
no valodas konstrukciju kopas kas šai pieejai raksturīga.
\begin{enumerate}
	\item \textbf{Strukturālais} pieraksta stils --- komponentes tiek
		izveidotas izmantojot vienkāršākas apakškomponentes un to savstarpējos
		savienojumus. Šis stils uzskatāms par tekstuālu analogu
		tradicionālajam, shematiskajam izstrādes veidam.
	\item \textbf{Funkcionālās} vai izturēšanās modelēšanas pieraksta
		stils --- komponentes tiek izveidotas aprakstot tās funkcionalitāti
		abstrahējoties no tās iespējamās uzbūves.
	\begin{itemize}
		\item \textbf{Secīgās izturēšanās} pieraksta stils --- apakškopa
			no funkcionālā stila, kurā izmantotas konstrukcijas, kas ļauj
			komponentes darbību aprakstīt ar secīgām izpildāmām izteiksmēm,
			līdzīgi imperatīvām programmēšanas valodām (kā C, Python, u.c.),
			pretstatā paralēli izpildāmām datu plūsmas izteiksmēm.
	\end{itemize}
\end{enumerate}

Šie stili ne tikai ietekmē koda (konkrētā shēmas apraksta teksta) pierakstu,
bet arī simulācijas un sintēzes procesus, kuriem šis pieraksts ir jāinterpretē.
Strukturālā pieraksta un tādu funkcionālā pieraksta primitīvu konstrukciju,
kā loģisko elementu (\texttt{UN}, \texttt{VAI}, utt.),
sintēze ir tieši translējama uz aparatūras komponentēm un to savienojumiem.
Savukārt sarežģītu funkcionālo konstrukciju, jo sevišķi secīgo
konstrukciju, sintēzei nepieciešams translēt funkcionalitātes aprakstu
uz konkrētu aparatūras implementāciju, t.i.~sintēzes rīkam nepieciešams
piemeklēt aparatūras komponentes kas realizē konkrēto funkcionalitāti.
Tā kā funkcionalitātes un implementācijas saistība bieži nav viennozīmīga,
dažādu sintēzes rīku interpretācija vienādam kodam var stipri atšķirties.

Praktiski visi sintēzes rīki funkcionālo pierakstu translē %uz aparatūras 
reģistru datu pārraides abstrakcijas līmenī jeb RTL
(no angļu \termEn{Register tranfer level}).\cite[2.~lpp.]{HDL}%
\cite[235.~lpp.]{Perry-VHDL}
RTL \todo


Citāda aina ir ar apraksta simulāciju. Tā kā simulācija notiek
programmatūras līmenī, vairumā gadījumu secīgās izturēšanās modeli
simulatoram ir vienkāršāk (un tādējādi arī ātrāk) simulēt nekā paralēli
izpildāmās konkurentās konstrukcijas.

