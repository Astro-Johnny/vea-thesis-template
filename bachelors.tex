\documentclass[12pt,a4paper]{article} % Maybe use 'report' ???
% REMEMBER TO REMOVE THE 'draft' TAG!
\usepackage{polyglossia}	% XeTeX multi-lingual support
\usepackage{fontspec}		% The XeTeX font spec package
\usepackage{xunicode}		% XeTeX Unicode character support
\usepackage{xltxtra}		% Umm something else XeTeX

\usepackage[intlimits]{amsmath}	% Math stuff
\usepackage{listings}	% Package for code formatting
\usepackage[dvipsnames]{xcolor}
%\include{/home/johnlm/code/avr_assembler_listing_def.tex}
%\input{avr_assembler_listing_def.tex}
%\renewcommand{\lstlistingname}{Koda bloks}
\usepackage{booktabs}	% Package for nicer tables
\usepackage[perpage,symbol*]{footmisc}	% Resets footnote marks per page
%\usepackage{wrapfig}
\usepackage[superscript]{cite}	% Superscript cites
\usepackage{rotating}	% Package for landscape figures
\usepackage{indentfirst}	%Indent first paragraph too
\usepackage{setspace}	% Line spacing package
\usepackage{fancyhdr}	% Fancy header/footer package
\usepackage[figurewithin=section,tablewithin=section]%
	{caption}	% Package for configuring captions format
\usepackage[a4paper,top=20mm, bottom=20mm, left=35mm, right=20mm]%
%\usepackage[a4paper,top=22mm, bottom=25mm, left=35mm, right=20mm]% DEBUG
	{geometry}
\usepackage{fixlatvian}	% Fixes for Latvian typography rules

% Preamble formatting settings defined included from external file
% Polyglossia settings
\setmainlanguage{latvian}
\setotherlanguages{english,russian}

% FONTS (Empty \setmainfont gives you new Computer Modern)
% Deja Vu Family (rather good but wide chars)
%\setmainfont{DejaVu Serif}
%\setsansfont{DejaVu Sans}

% Liberation Family (my current favourite)
%\setmainfont{Liberation Serif}
%\setsansfont{Liberation Sans}

% Monos
%\setmonofont{DejaVu Sans Mono}

% Russian Font
%\newfontfamily\cyrillicfont{Liberation Serif}
\newfontfamily\cyrillicfont{CMU Serif}	% Unicode Computer modern font
%Other fonts are too "fat"/wide

%% Liberation Serif for VeA title font
\definecolor{SeaBlue}{cmyk}{1,0,0.37,0.52}
%% Use proper Aurora Bold Condensed BT
\newfontfamily\TitleFont[%
	Path = /home/johnlm/Development/resources/tex_graphics_lib/fonts/ ,%
	FakeStretch = 1.5 ,
	Color = SeaBlue
	]{aurora-BT-condensed-bold}

% External common image library
\graphicspath{{/home/johnlm/Development/resources/tex_graphics_lib/}{img/}}

\hyphenpenalty=7500
\clubpenalty=9000
\widowpenalty=9000

% Set rubberband values
\setlength{\parskip}{1ex plus 3ex minus 1ex}	% Space between paragraphs


%\renewcommand{\thefootnote}{\alph{footnote}} % Piezīmes ar burtiem
\newcommand{\citeet}[1]{\textsuperscript{\cite{#1}}}

%% Numbering format fix for sections and subsections (trailing full-stop)
%\def\thesection{\arabic{section}.}
%\def\thesubsection{\thesection\arabic{subsection}.}
% Add deeper levels if required
%% FIXED: fixlatvian.sty provides these (REMOVE THIS)

% Reformatting captions
\captionsetup{format=hang,font={small}}	% GLOBAL
%\captionsetup[table]{position=top,

% Footer adjustements (by redefining plain pagestyle)
\setlength{\headheight}{15.2pt}
\fancypagestyle{plain}{%
	\fancyhf{}	% Clear current (default) header/footer
	\rfoot{\thepage}	% Page number on outside
	\renewcommand{\headrulewidth}{0pt}
	\renewcommand{\headrulewidth}{0pt}}


% Other stuff
\newcommand{\todo}{\textcolor{red}{TODO}}

%% Inline Word inclusion formatting
\newcommand{\termEn}[1]{\textenglish{\itshape {#1}}} % inline English

% Superscript cite
\makeatletter
	\renewcommand\@citess[1]{\textsuperscript{[#1]}}
\makeatother




%% This file contains special, document-specific preamble settings
% Binary code formatting for instruction table
\newcommand{\instr}[7]{\texttt{%
	\textcolor{purple}{#1}%		Group OPCODE
	\textcolor{blue}{#2}%		Vector OPCODE (part 1)
	\textcolor{cyan}{#3}%		Vector OPCODE (part 2)
	\textcolor{lightgray}{#4}%		Middle Slack bits
	\textcolor{OliveGreen}{#5}%		Argument 1
	\textcolor{Green}{#6}%	Argument 2
	\textcolor{lightgray}{#7}}}%		Trailing Slack bits
\newcommand{\mnem}[1]{\texttt{\bfseries #1}}

%% listings definiicijas
\lstdefinelanguage[qucs]{VHDL}[]{VHDL}%
{%morekeywords=[1]{},% MOAR core keywords
morekeywords=[3]{ieee,std_logic_1164,std_logic_arith,std_logic_unsigned,%
	work,std,textio,numeric_std,% usual libraries
	std_logic,std_logic_vector,unsigned,signed,%logic data types
	time,integer,real,line,string,positive,natural,% data types
	ns},%real type units
morekeywords=[4]{reverse_range,length,event,last_value,high,low},% attributes
morekeywords=[5]{write,writeline,conv_integer,to_unsigned,rising_edge,falling_edge}%,%funkcijas
%morestring=[b]',
}

\lstdefinelanguage{JSI}%
 {morekeywords={LD,LDI,MOV,ST,AR,ADD,SUB,INC,DEC,AND,OR,XOR,NOT,CLR,%
	MOV,LSL,LSR,ROR,ROL,BREQ,BRNQ,BRGT,BRGE,BRLE,BRLT,HLT,JMP},%
morekeywords=[2]{DEVCTRL_REG_ADDR,SPI_RING_REG_ADDR,%
	BOOT_ROM_START,BOOT_ROM_END,%
	CONF_REG_START,CONF_REG_END,IO_MAP_START,IO_MAP_END,%
	SRAM_START,SRAM_END,SRAM_A_START,SRAM_A_END,SRAM_B_START,SRAM_B_END},% Ports and internal registers
morekeywords=[3]{def,include,equ,dw,org},% directives
morekeywords=[4]{DCR_SPI_WIDTH,SPI_16_MODE,SPI_8_MODE,%
	DCR_SPI_FLASH_EN},% header constants
   sensitive=true,%
   %morecomment=[l]*,%
   morestring=[b]",
   morecomment=[l];%
   }[keywords,comments,strings]



%% English determined number suffixes
%\renewcommand{\th}{\textsuperscript{th} }
\newcommand{\st}{\textsuperscript{st} }
\newcommand{\rd}{\textsuperscript{rd} }
\newcommand{\nth}{\textsuperscript{th} }


% Hyperref package for links in PDF (should be last package)
\usepackage[linkbordercolor={blue},hyperindex]{hyperref}
\hypersetup{pdftitle={Iekļautās sistēmas mikrokontroliera kodola izstrāde}}
\hypersetup{pdfauthor={Jānis Šmēdiņš}}


% \onehalfspacing

% Chardump: „” em— en– fig‒ “”
\begin{document}
	% Titullapa
	\pagestyle{empty}
	% LM izcilā VeA titullapa bakalauram!
\begin{titlepage}
	% VeA Logo
	% "Ventspils Austskola" virsraksts (\TitleFont jābūt definētam!!!)
	\newsavebox{\veatext}
	\savebox{\veatext}{
		\TitleFont\Huge\MakeUppercase{Ventspils Augstskola}}
	% Teksta platuma noteikšana (lai pielīdzinātu bildi tā platumam}
	\newlength{\veatextwidth}
	\settowidth{\veatextwidth}{\usebox{\veatext}}
	% Šeit drukāts pats logo un nosaukums
	\centering\includegraphics[width=0.98\veatextwidth]{VeA_logo.pdf}\\[4pt]
	\usebox{\veatext}\\[6pt]
	\large Informācijas tehnoloģiju fakultāte\\[2cm]
	
	\textbf{Bakalaura darbs}\\[1.5cm]
	%\textbf{Bakalaura darba 1.~atskaite}\\[1.5cm]
	
	\textsc{\LARGE Iekļautās sistēmas mikrokontroliera kodola izstrāde}
	\vfill % LIELĀ atstarpe
	
	%\raggedleft
	\normalsize
	\begin{minipage}[t]{0.4\textwidth}
		\begin{flushleft}
			Autors
		\end{flushleft}
	\end{minipage}
	\begin{minipage}[t]{0.55\textwidth}
		\begin{flushleft}
			Ventspils Augstskolas \\
			Informācijas tehnoloģiju fakultātes \\
			bakalaura studiju programmas „Elektronika”\\
			3. kursa students \\
			\textbf{Jānis Šmēdiņš}\\
			Matr.~nr.~\texttt{2009120280}\\
			\rule[-1em]{10em}{1pt}\\
			\makebox[10em][c]{\tiny (paraksts)}\\[1cm]
		\end{flushleft}
	\end{minipage}\\[2em]
	\begin{minipage}[t]{0.4\textwidth}
		\begin{flushleft}
			Fakultātes dekāns
		\end{flushleft}
	\end{minipage}
	\begin{minipage}[t]{0.55\textwidth}
		\begin{flushleft}
			asoc.prof.,~Dr.~math.~Gaļina Hiļķeviča\\[1ex]
			\rule[-1em]{10em}{1pt}\\
			\makebox[10em][c]{\tiny (paraksts)}\\[1cm]
		\end{flushleft}
	\end{minipage}\\[2em]
	\begin{minipage}[t]{0.4\textwidth}
		\begin{flushleft}
			Zinātniskais vadītājs
		\end{flushleft}
	\end{minipage}
	\begin{minipage}[t]{0.55\textwidth}
		\begin{flushleft}
			%asoc.prof.,~Dr.~math.~Gaļina Hiļķeviča\\
			\rule[-1em]{20em}{1pt}\\
			\makebox[20em][c]{\tiny
				(ieņemamais amats,
				zinātniskais nosaukums,
				vārds, uzvārds)}\\[1ex]
			\rule[-1em]{10em}{1pt}\\
			\makebox[10em][c]{\tiny (paraksts)}\\[1cm]
		\end{flushleft}
	\end{minipage}\\[2em]
	\begin{minipage}[t]{0.4\textwidth}
		\begin{flushleft}
			Recenzents
		\end{flushleft}
	\end{minipage}
	\begin{minipage}[t]{0.55\textwidth}
		\begin{flushleft}
			\rule[-1em]{20em}{1pt}\\
			\makebox[20em][c]{\tiny
				(ieņemamais amats,
				zinātniskais nosaukums,
				vārds, uzvārds)}\\[1ex]
			\rule[-1em]{10em}{1pt}\\
			\makebox[10em][c]{\tiny (paraksts)}\\[1cm]
		\end{flushleft}
	\end{minipage}\\[1cm]
	
	\centering
	Ventspils\\
	\the\year
\end{titlepage}
 % Ielikt titullapu
	\stepcounter{page} %Palielināt lapu skaitītāju (jeb ieskaitīt titullapu)
	\lstset{%
			numbers=left, numberstyle=\tiny, numbersep=5pt,
			tabsize=4,%
			basicstyle=\footnotesize\ttfamily,%
			keywordstyle={\bfseries\color{blue}},%
			keywordstyle=[2]{\color{Brown}},%
			keywordstyle=[3]{\color{OliveGreen}},%
			keywordstyle=[4]{\color{Bittersweet}},%
			commentstyle=\slshape\color{gray},%
			emphstyle={\color{PineGreen}},%
			stringstyle=\color{orange}} % "listings" koda stils
	
	%\section*{Saturs}
	\renewcommand{\contentsname}%
		{\vspace*{-12mm}\section*{Saturs}\vspace*{-6mm}}
	\tableofcontents
	%\addcontentsline{toc}{section}{Saturs}	% Saturs ir saturaa?
	
	\clearpage
	%\sloppy %Don't care too much about rigid formatting
	\pagestyle{plain}
	\onehalfspacing % 1.5 spacing
	
	% Termini
		% PCI, PCI Express
		% Seriālā, paralēlā datu pāraide
	
	%\section*{Anotācija}
	%\renewcommand{\abstractname}%
	%	{\section*{Anotācija}}
	%%% Abstracts
% The 'abstract' enviroment makes more problems than it is useful

\phantomsection\addcontentsline{toc}{section}{\abstractname}
\abstitlestyle{\abstractname} % The abstract title (centered)
% Abstract text goes here
\noindent%
\begin{tabularx}{\textwidth}{lX}
	\textbf{Darba nosaukums:} & 
		\textit{Iekļautās sistēmas mikrokontroliera kodola izstrāde}\\[1ex]
	\textbf{Darba autors:} & Jānis Šmēdiņš\\[1ex]
	\textbf{Darba vadītājs:} & Mg.~sc.~comp.~Gatis Gaigals\\[1ex]
	\textbf{Darba apjoms:} & 68~lpp., 3~tabulas, 26~attēli,
		12~pirmkoda izdrukas, 19~bibliogrāfiskie avoti, 5~pielikumi\\[1ex]
	\textbf{Atslēgas vārdi:} & Mikrokontroliera kodols, aparatūras apraksta valodas,
		sintezējams mikrokontrolieris
\end{tabularx}

\vspace{1em}
Šī bakalaura darba mērķis ir izstrādāt modulāru mikrokontroliera kodolu,
kurš izmantojams dažādu, specializētu mikrokontrolieru implementācijās,
kuras bieži nepieciešamas specifiskos elektroniskajos risinājumos,
galvenokārt, zinātniskajos projektos. Mikrokontroliera un
izstrādātā kodola mērķa platforma ir FPGA.
%FIXME: Saīsinājuma atšifrējumu vajag uzrādīt?

Izstrāde veikta izmantojot aparatūras apraksta valodas, kas ir galvenais
FPGA projekta izstrādes instruments. Ir veikts divu galveno aparatūras
apraksta valodu (VHDL un Verilog) salīdzinājums, lai pamatotu izstrādes valodas izvēli.
Darbā analizēta eksistējoša procesora
arhitektūra, identificēti tās
trūkumi un izstrādāti uzlabošanas risinājumi, kā rezultātā izveidota 
bāzes arhitektūra, pēc kuras
realizēts izstrādājamais mikrokontroliera kodols.

Šī darba laikā kodols tika veiksmīgi izstrādāts, tā pārbaudei un 
demonstrācijai izstrādāts arī parauga mikrokontrolieris, kurā šis kodols
izmantots. Papildus izstrādāts asemblerkoda translators.
Izvirzīti vairāki priekšlikumi, kā uzlabot izstrādāto
mikrokontroliera kodolu.

\clearpage
\begin{english} % The English abstract
	\phantomsection\addcontentsline{toc}{section}{\abstractname}
	\abstitlestyle{\abstractname} % The abstract title (centered)
	% Abstract text goes here
	\noindent%
	\begin{tabularx}{\textwidth}{lX}
		\textbf{Title:} & 
			\textit{Development of embedded system microcontroller core}\\[1ex]
		\textbf{Author:} & Jānis Šmēdiņš\\[1ex]
		\textbf{Supervisor:} & Mg.~sc.~comp.~Gatis Gaigals\\[1ex]
		\textbf{Scope:} & ??~pages, ??~tables, ??~figures, ??~listings,
			??~bibliographical references, ?~appendices\\[1ex]
		\textbf{Keywords:} & Microcontroller core, hardware description languages,
				synthesisable microcontroller
	\end{tabularx}
	
	\vspace{1em}
	This bachelors thesis purpose is to develop a microcontroller core for
	synthesis on FPGA, as part of microcontroller, whether stand-alone or
	control device within special purpose integrated circuit.
	The thesis also focuses on hardware description languages
	(HDL) as a main tool of attaining this goal.
	
	The author takes a more practical approach by developing first and
	addressing problems as they arise. The work used an existing
	architecture of central processing unit as base which was then improved
	and modified in successive, iterative development cycles
	(dubbed revisions), where each such cycle is to result in a working
	prototype.
	
	During the development the instruction set of core was completely
	revised and several notable architectural changes done, most evidently
	the internal data bus was reimplemented as one-way, branched, circular
	pipeline. The author successfully developed a synthesisable
	microcontroller core which is the result of third development cycle (rev.~03).
	For testing	and demonstration purposes a complete microcontroller prototype
	incorporating the core was developed as well.
	
	The author proposes several possible improvements for a
	potential next development cycle, but also acknowledging that there
	could be indefinite number of such cycles, and that one fulfilling
	requirements will suffice.
	
	%%%%%%%%%%%%
	%This bachelors thesis focuses on an existing architecture of 
	%central processing unit	prototype. The author analyses the architecture,
	%identifying design flaws and suggesting solutions to these flaws.
	%These solutions are put to practical use by developing a
	%synthesisable microcontroller core (essentialy a CPU),
	%improving upon the existing	architecture.
	
	%To demonstrate use of the developed microcontroller core, a prototype
	%microcontroller is developed as well, which incorporates the core.
	
	%Hardware description languages are analysed as well, of which the VHDL
	%is used as a main tool of development. A short introduction and
	%analysis is done on FPGAs, which is the target platform for design
	%synthesis.
\end{english}

\clearpage
\begin{russian} % The Russian abstract
	\phantomsection\addcontentsline{toc}{section}{\abstractname}
	\abstitlestyle{\abstractname} % The abstract title (centered)
	% Abstract text goes here
	\noindent%
	\begin{tabularx}{\textwidth}{lX}
		\textbf{Название работы:} & 
			\textit{Разработка ядра микроконтроллера во встроенной системы}\\[1ex]
		\textbf{Автор работы:} & Jānis Šmēdiņš (транслит.~Янис Шмэдиньш)\\[1ex]
		%\textbf{Руководитель работы:}
		\textbf{Руководитель:} & Mg.~sc.~comp.~Gatis Gaigals 
			(транслит.~Гатис Гайгалс)\\[1ex]
		\textbf{Размер:} & 68 стр., 3 таблицы, 26 изображений,
			12 листингов (печатей исходного кода), 19 библиографических источников, 5 приложений\\[1ex]
		\textbf{Ключевые слова:} & ядро микроконтроллера,
			языки описания аппаратуры, синтезируемый микроконтроллер
	\end{tabularx}
	
	
	Цель работы бакалавра разработать модулярное ядро микроконтроллера
	которого можно использовать в разных специализированных имплементациях
	микроконтроллеров которые часто нужны в специфичных электронных решениях,
	восновном, в научных проектах. Целевой платформой микрокотроллера и
	разработанного ядра является FPGA.

	Разработка сделана, используя языки описания аппаратуры, которые
	являются главным инструментом проекта FPGA. Сделано сравнение двух
	главных языков описания аппаратуры (VHDL и Verilog), что-бы основать
	выбор языка разработки. В работе анализируется архитектура существующего
	процесора, идентифицированы недостатки и разработаны способы улучшения,
	в результате сделана базовая архитектура, по которой реализовано
	ядро разрабатываемого микроконтроллера.

	Во время этой работы ядро было успешно разработано, для его
	проверки и демонстрации  тоже разработан образцовый микроконтроллер,
	в котором это ядро будет использоватся. Дополнительно разработан
	транслятор кода ассемблера. Выдвинуто несколько предложений,
	как улучшить разрабатываемое ядро микроконтроллера.
	
\end{russian}

	
	\clearpage
	%\section*{Ievads} \addcontentsline{toc}{section}{Ievads}
Šis darbs apraksta mikrokontroliera kodola un tā perifērijas izveidi
funkcionālai parauga mikrokontroliera realizācijai. \todo

% TODO: Potenciālais pielietojums
	
	VHDL apraksta valodā. Izklāstītā informācija balstīta vairāk uz
	praktiskā darba rezultātiem un tādējādi arī darba gaitā pieņemtajiem
	lēmumiem sastapto problēmu risinājumam.
	
	Apskatīta procesora uzbūve bez kopējās datu šinas un reducētu instruk\-ciju
	kopu, realizācijai uz FPGA čipa platformas — konkrēti,
	\termEn{Actel Fusion Embedded Development Kit} ar \texttt{M1AFS1500}
	FPGA čipu. Darba mērķi tātad ir:
	\begin{itemize}
		%\item Funkcionāla procesora izstrāde VHDL;
		%\item Perifērijas izstrāde sintezējamai sistēmai;
		%\item Sistēmas sintēze un pārbaude.
		\item \todo
	\end{itemize}
	
	Darba gaitā pieņemtie lēmumiem, kas ievērojami izmaina sistēmas
	imple\-men\-tā\-ciju un ir izstrādes pagrieziena punkts, nosauktas par
	„Revīzijām”.
	Lai gan pamatā darbs atspoguļo galējo rezultātu, nodaļu
	beigās pievienots pārskats par Revīziju vēsturi — to iespaids uz
	izstrādi, kā arī šo izmaiņu pieņemšanas iemesli.
 % TODO: Rewrite this bugger
	
	\clearpage
	\section{Sistēmas perifērija}
	Atsevišķs procesors ir nefunkcionāls, ja tam nevar pievadīt izpildāmās
	instrukcijas, atgūt datus vai vadīt kādu citu perifēro ierīci.
	
	Praktiski vienmēr procesoram nepieciešama atmiņas iekārta (RAM), kur
	glabāt izpildāmo programmu, kā arī pirmsapstrādes, pēcapstrādes un 
	apstrādes laika datus.
	
	Šajā nodaļā apskatītas divas sistēmas shēmas, no kurām viena veiksmīgi
	nosimulēta, bet nav sitezējama (rev.~02) un krietni sarežģītākā
	sintezējamā shēma, kura satur papildus perifērās ierīces.
	
	\subsection{Simulētā shēma (rev.~02)}
Sākotnējā darba augšējā līmeņa sistēmas gala shēma paredzēta, kā
paškomplektējoša, tikai ar procesoru un atmiņu, 
kur no ārpuses tiek tikai dota takts (\texttt{CLOCK})
un atiestatīšanas signāls ($\overline{\texttt{RESET}}$). Darbības pārbaudei
tiek izmantotas izstrādes platformas piedāvātās diodes.

\begin{figure}[bhp]
	\centering
	\def\svgwidth{\textwidth}
	{\ttfamily\tiny\input{img/top-rev2.pdf_tex}}
	\caption{Augšējā līmeņa shēma (rev.~02).}
	\label{fig:top-rev2}
\end{figure}

Šī shēma tika veiksmīgi simulēta pirms sintēzes
(rezultātus sk.~\ref{appx:simulation}.~pielikumā),
bet tā nav korekti sintezējama, jo paredz RAM sākotnējos datus,
kurus sintēzes rīks ignorē.
Realitātē RAM dati tiek pazaudēti tikko tiek noņemts barošanas spriegums un
tātad pēc šādas shēmas nav iespējams saglabāt izpildāmo programmu.

\pagebreak[3]
%\subsubsection{Operatīvā atmiņa}
	Operatīvā atmiņa šeit realizēta ar divām vienvirziena datu apmaiņas
	šinām. Kontroles signāli izmantoti līdzīgi klasiskai trīs-stāvokļu
	divvirzienu datu šinas atmiņai, pielāgojoties procesora atmiņas
	saskarnei.
	
	\singlespacing
	\lstinputlisting[%float=p,
	                language={[qucs]VHDL},%
	                caption={RAM VHDL entītija.},%
	                linerange={7-13},firstnumber=7,
	                breaklines,breakatwhitespace,
	                label=kb:ram-entity%,%
	                ]
		{code/mem.twoport.vhd}
	
	\lstinputlisting[%float=p,
	                language={[qucs]VHDL},%
	                caption={RAM VHDL arhitektūras apraksts (izgriezums).},%
	                linerange={85-99},firstnumber=85,
	                breaklines,breakatwhitespace,
	                label=kb:ram-trimmed%,%
	                ]
		{code/mem.twoport.vhd}
	\onehalfspacing % 1.5 spacing
	
	\noindent Pilno kodu skatīt \ref{appx:ram-code}.~pielikumā.
 \clearpage %\pagebreak[3]
	\input{sys.rev3.tex} \clearpage %\pagebreak[3]
	\input{sys.revisions.tex}
	
	% Literatūras saraksts
	\bibliographystyle{ieeetr}
	%\renewcommand{\refname}%
	%	{\vspace*{-12mm}\section*{Izmantotā literatūra}\vspace*{-6mm}}
	\clearpage
	%\pagestyle{empty}
	\raggedright
	\begin{thebibliography}{9}
		\addcontentsline{toc}{section}{\refname}
		\bibitem{Perry-VHDL}
			Douglas L.~Perry,
			\textit{VHDL: Programming by Example}, 4\nth Edition. \linebreak[2]
			New York: McGraw-Hill, 2002. ISBN~0-07-140070-2
		
		\bibitem{FusionGuide}
			Actel corp.,
			\textit{Fusion Embedded Development Kit User's Guide}. %\linebreak[2]
			%USA: Actel,
			2009.
		
		\bibitem{FusionFAQ}
			Actel corp.,
			\textit{Synplify — Synthesis Frequently Asked Questions}. %\linebreak[2]
			%USA: Actel,
			2009.
		
		\bibitem{FlashROM}
			Actel corp.,
			\textit{Fusion FlashROM}, Application Note AC236. %\linebreak[2]
			%USA: Actel,
			2005.
		
		%\bibitem{LDD3}
		%	Corbet~J., Rubini~A., Kroah-Hartman~G.,
		%	\textit{Linux Device Drivers, Third Edition}.
		%	Sebastopol: O'Reilly, 2005. ISBN~0-596-00590-3
		
		%\bibitem{PCIeArch}
		%	Burduk~R., Anderson~D., Shanley~T.,
		%	\textit{PCI Express System Architecture}.
		%	\linebreak[3]
		%	Boston: Addison-Wesley, 2009. ISBN~0-321-15630-7
	\end{thebibliography}
	
\end{document}
