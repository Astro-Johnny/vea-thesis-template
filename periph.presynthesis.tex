\subsection{Simulētā shēma (rev.~02)}
Sākotnējā darba augšējā līmeņa sistēmas gala shēma paredzēta, kā
paškomplektējoša, tikai ar procesoru un atmiņu, 
kur no ārpuses tiek tikai dota takts (\texttt{CLOCK})
un atiestatīšanas signāls ($\overline{\texttt{RESET}}$). Darbības pārbaudei
tiek izmantotas izstrādes platformas piedāvātās diodes.

\begin{figure}[bhp]
	\centering
	\def\svgwidth{\textwidth}
	{\ttfamily\tiny\input{img/top-rev2.pdf_tex}}
	\caption{Augšējā līmeņa shēma (rev.~02).}
	\label{fig:top-rev2}
\end{figure}

Šī shēma tika veiksmīgi simulēta pirms sintēzes
(rezultātus sk.~\ref{appx:simulation}.~pielikumā),
bet tā nav korekti sintezējama, jo paredz RAM sākotnējos datus,
kurus sintēzes rīks ignorē.
Realitātē RAM dati tiek pazaudēti tikko tiek noņemts barošanas spriegums un
tātad pēc šādas shēmas nav iespējams saglabāt izpildāmo programmu.

\pagebreak[3]
%\subsubsection{Operatīvā atmiņa}
	Operatīvā atmiņa šeit realizēta ar divām vienvirziena datu apmaiņas
	šinām. Kontroles signāli izmantoti līdzīgi klasiskai trīs-stāvokļu
	divvirzienu datu šinas atmiņai, pielāgojoties procesora atmiņas
	saskarnei.
	
	\begin{singlespace}
		\lstinputlisting[language={[qucs]VHDL},%float=p,
		                caption={RAM VHDL entītija.},%
		                label=kb:ram-entity,%
		                linerange={7-13},firstnumber=7,
		                breaklines,breakatwhitespace]
			{code/mem.twoport.vhd}
	
		\lstinputlisting[language={[qucs]VHDL},%float=p,
		                caption={RAM VHDL arhitektūras apraksts (izgriezums).},%
		                label=kb:ram-trimmed,%
		                linerange={85-99},firstnumber=85,%
		                breaklines,breakatwhitespace]
			{code/mem.twoport.vhd}
	\end{singlespace}
	
	\noindent Pilno kodu skatīt \ref{appx:ram-code}.~pielikumā.
