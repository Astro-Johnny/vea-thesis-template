{\raggedright
	\begin{thebibliography}{99}
		\addcontentsline{toc}{section}{\refname}
		%~ \bibitem{VHSIC}
			%~ Creasey~D.J.,
			%~ \textit{Advanced Signal Processing}.\linebreak[1]
			%~ London: Peter~Peregrinus, 1985. ISBN~0-86341-037-5
		
		\bibitem{Fatahalian}
			Fatahalian~K., Houston~M.,
			``A Closer Look at GPUs.''\linebreak[1]
			\textit{Communications of the ACM} 51.10, pp.~50--57, 2008.
		
		\bibitem{Flynn-arch}
			%Michael J.~Flynn,
			Flynn~M.J.,
			\textit{Computer Architecture: Pipelined and Parallel Processor Design}.
			\linebreak[3]
			London: Jones~and~Bartlett, 1995. ISBN~0-86720-204-1
		
		\bibitem{Mokhtarian}
			Mokhtarian~F., Mohanna~F.,\linebreak[1]
			``Performance evaluation of corner detectors using
			consistency and accuracy measures.''
			\textit{Computer Vision and Image Understanding}~102, pp.~81--94, 2006.
		
		\bibitem{von-Neumann}
			John von Neumann,
			\textit{First Draft of a Report on the EDVAC}, (1945).\linebreak[2]
			Ed.~Godfrey~M.D., 2011.
		
		\bibitem{Owens-GPU}
			Owens~J.D., Houston~M., Luebke~D.,~et~al.,
			``GPU Computing.''\linebreak[1]
			\textit{Proceedings of the IEEE}~96.5, pp.~879--899,
			2008.
		
		\bibitem{Patterson2}
			Patterson~D.A., Hennessy~J.L.,\linebreak[1]
			\textit{Computer Architecture: A Quantitative Approach},
				5\nth ed..\linebreak[1]
			Waltham: Morgan~Kaufmann, 2012. ISBN~978-0-12-383872-8
		
		\bibitem{Patterson}
			Patterson~D.A., Hennessy~J.L., \linebreak[1]
			\textit{Computer Organization Design: %\linebreak[1]
				The Hardware/Software Interface}, 3\rd ed..\linebreak[1]
			San~Francisco: Morgan~Kaufmann, 2005. ISBN~1-55860-604-1
		
		\bibitem{Rosten-tracking}
			Rosten~E., Drummond~T.,\linebreak[1]
			``Fusing Points and Lines for High Performance Tracking.''\linebreak[1]
			\textit{IEEE International Conference on Computer Vision} vol.~2,
			pp.~1508--1511, 2005.
		
		\bibitem{FAST}
			Rosten~E., Drummond~T.,
			``Machine learning for high-speed corner detection.''
			\textit{European Conference on Computer Vision}, vol.~1, pp.~430--443,
			2006.
		
		\bibitem{ORB}
			Rublee~E., Rabaud~V., Konolige~K.,~et~al.,\linebreak[1]
			``ORB: an efficient alternative to SIFT or SURF.''\linebreak[1]
			\textit{IEEE International Conference on Computer Vision},
			pp.~2564--2571, 2011.
		
		\bibitem{Cache}
			Sorin~D.J., Hill~M.D., Wood~D.A.,
			\textit{A Primer on Memory Consistency and Coherence}.
			San~Rafael: Morgan~\&~Claypool, 2011. ISBN~978-1608455645
		
		\bibitem{SIMD}
			Stokes~J.,
			\textit{SIMD architectures} [online]. Ars Technica, 2000 %---
			[cites~April~16,~2014].\linebreak[1]
			Available: \url{http://arstechnica.com/features/2000/03/simd/}
		
		\bibitem{JIS}
			Šmēdiņš~J.,
			\textit{Iekļautās sistēmas mikrokontroliera kodola izstrāde}.\linebreak[1]
			Venstpils Augstskola, 2012.
		
		\bibitem{SIFT-FPGA}
			Tabib~W.,
			\textit{FPGA-Based Feature Detection}.
			Carnegie Mellon University, 2012.
		
		%~ \bibitem{Golshan-ASIC}
			%~ Golshan~K.,
			%~ \textit{Physical Design Essentials: An ASIC Design Implementation Perspective}.
			%~ New York: Springer, 2007. ISBN~0-387-36642-3
		
		%~ \bibitem{Heath}
			%~ %Steve Heath,
			%~ Heath~S.,
			%~ \textit{Embedded systems design}, 2\nd edition.\linebreak[3]
			%~ Oxford: Newnes, 2003. ISBN~0-7506-5541-1
		
		%~ \bibitem{VITAL}
			%~ IEEE,
			%~ \textit{IEEE 1076.4/D1, DRAFT Standard, VITAL ASIC Modeling Specification}.
			%~ New York: IEEE, 2000.
		
		%~ \bibitem{ieee-1364.1}
			%~ IEEE,
			%~ \textit{IEEE Std. 1364.1-2002, IEEE Standard for Verilog Register Transfer Level Synthesis}.
			%~ New York: IEEE, 2002.
		
		%~ \bibitem{HDL}
			%~ %Gaurav Mehta, Sridhar Kintali,
			%~ Mehta~G., Kintali~S.,
			%~ \textit{Hardware Description Languages}.\linebreak[2]
			%~ Santa Barbra: University of California,
			%~ 2009.
		
		
		%~ \bibitem{Perry-VHDL}
			%~ %Douglas L.~Perry,
			%~ Perry~D.L.,
			%~ \textit{VHDL: Programming by Example}, 4\nth edition. \linebreak[2]
			%~ New York: McGraw-Hill, 2002. ISBN~0-07-140070-2
		
		%~ \bibitem{Vivek-Verilog}
			%~ %Vivek Sagadeo,
			%~ Sagdeo~V.,
			%~ \textit{The Complete Verilog Book}.\linebreak[2]
			%~ Norwell: Kluwer Academic Publishers, 1998. ISBN~0-7923-8188-2
		
		%~ \bibitem{Vahid-RTL}
			%~ %Frank Vahid,
			%~ Vahid~F.,
			%~ \textit{Digital Design with RTL Design, VHDL, and Verilog}, 2\nd edition.\linebreak[2]
			%~ New York: % Hell knows which city
			%~ John Wiley \& Sons, 2011. ISBN~978-0-470-53108-2
		
		%~ \bibitem{FusionGuide}
			%~ Actel corp.,
			%~ \textit{Fusion Embedded Development Kit User's Guide}. %\linebreak[2]
			%~ %USA: Actel,
			%~ 2009.
		
		%~ \bibitem{FlashROM}
			%~ Actel corp.,
			%~ \textit{Fusion FlashROM}, Application Note AC236. %\linebreak[2]
			%~ %USA: Actel,
			%~ 2005.
		
		%~ \bibitem{RAM4K9}
			%~ Actel corp.,
			%~ \textit{Fusion SRAM/FIFO Blocks}, Application Note AC237. %\linebreak[2]
			%~ %USA: Actel,
			%~ 2005.
		
		%~ \bibitem{FusionFAQ}
			%~ Actel corp.,
			%~ \textit{Synplify — Synthesis Frequently Asked Questions}. %\linebreak[2]
			%~ %USA: Actel,
			%~ 2009.
		
		%~ \bibitem{SmartFusionFabric}
			%~ Microsemi corp.,
			%~ \textit{SmartFusion FPGA Fabric User's Guide}.
			%~ %USA: Microsemi,
			%~ 2011.
		
		%~ \bibitem{Xilinx7}
			%~ Xilinx,
			%~ \textit{7 Series FPGAs Overview}.
			%~ %USA: Xilinx,
			%~ 2012.
		
		%~ \bibitem{Mealy-VHDL}
			%~ %Bryan Mealy, Fabrizio Tappero,
			%~ Mealy~B., Tappero~F.,
			%~ \textit{Free Range VHDL}.
			%~ 2012. [tiešsaiste] \linebreak[2]
			%~ Pieejams: \url{http://www.freerangefactory.org/dl/free_range_vhdl.pdf}\linebreak[2]
			%~ \mbox{[skatīts on 2012.~gada 2.~maijā]}
		
		%~ \bibitem{vhdl-vs-verilog}
			%~ Smith~D.J.,
			%~ \textit{VHDL \& Verilog Compared \& Contrasted}. [tiešsaiste]\linebreak[2]
			%~ Pieejams: \url{http://www.angelfire.com/in/rajesh52/verilogvhdl.html}
			%~ \mbox{[skatīts on 2012.~gada 21.~maijā]}
		
		%~ \bibitem{Kumar-Verilog}
			%~ % Deepak Kumar Tala,
			%~ Tala~D.K., \textit{Gate Level Modeling}. [tiešsaiste]\linebreak[2]
			%~ Pieejams: \url{http://www.asic-world.com/verilog/gate.html}
			%~ \mbox{[skatīts on 2012.~gada 24.~maijā]}
	\end{thebibliography}
} % "End of \raggedright"
