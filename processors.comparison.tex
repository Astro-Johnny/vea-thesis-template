\subsection{Salīdzinājums} \label{sec:proc-cmp}
CPU, GPU un FPGA salīdzinājums nav vienkāršs, jo šīs platformas ir ļoti
atšķirīgas, gan uzbūves, gan pielietojuma, gan programmēšanas stila ziņā.
Ievērojamas arhitekturālas atšķirības arī ir novērojamas konkrētās
platformas tipa ietvaros, piem.,~dažādu arhitektūru un paaudzes CPU.

Autors secina, ka nozīmīgs šo platformu tipu salīdzinājums ir to
raksturojošo īpašību kvalitatīva analīze nosakot to priekšrocības vai
trūkumus un to iespaidu uz algoritmu realizāciju. Savukārt to
kvantitatīvie rādītāji, kā piemēram, takts frekvence vai operācijas sekundē
(IPS vai FLOPS), ir tikai nozīmīgas salīdzināšanai viena tipa platformas,
bet ir nederīgi salīdzinot dažādus platformas tipus, vai pat salīdzinot viena tipa
platformas, ja netiek ņemta vērā veicamā uzdevuma specifika.

Apakšnodaļas turpinājumā tiks izvirzītas konkrētas platformu īpašības un
platformu tipi salīdzināti šīs īpašības kontekstā.

\subsubsection*{Skaitļošanas pamatelementi un darbības}
\emph{CPU} galvenie skaitļošanas pamatelementi ir veselie skaitļi jeb
\termEn{\emph{integer}} tips. Praktiski vienmēr aparatūrā tiek realizētas
bitu operācijas, saskaitīšana un atņemšana.
Reizināšana un dalīšana tiek realizēta retāk. Augstākas veiktspējas CPU,
tiek arī realizētas \emph{peldošā komata skaitļu} (\termEn{floating-point})
aritmētika, kā arī realizētas \emph{SIMD instrukcijas},
kas ļauj veikt darbības ar skaitļu vektoriem.
\cite{Flynn-arch}\cite{Patterson}\cite{SIMD}

\emph{GPU} veiktspējas pamatā ir SIMD darbības ar \emph{peldošā komata
skaitļu vektoriem}, bet dubultprecizitātes (\termEn{double-precision})
peldošā komata skaitļu atbalsts ir ierobežots vai neeksistējošs.
\termTech{Integer} tipa skaitļu aritmētikai ir atbalsts, bet tās veiktspēja
var būt ierobežota.
\cite{Fatahalian}\cite{Owens-GPU}

\emph{FPGA} uzbūves pamatā ir \emph{bits}, no kura var tik konstruēti
patvaļīga garuma (t.i.,~ne obligāti $2^n \cdot 8$ bitu garuma) reģistri
\termTech{integer} skaitļiem. Veicamās operācijas arī definējamas ar FPGA
programmējumu~\cite{JIS}, un tādējādi tās var būt sākot no pamata aritmētiskajām darbībām
līdz pat ļoti specializētām vai ezotēriskām operācijām. Realizējot FPGA
iepriekš \ref{sec:fpga}~nodaļā aprakstīto signāltrakta modeli, atsevišķu
operāciju izdalīšana ir apgrūtināta un par operāciju var uzskatīt pat visu
realizēto algoritmu. FPGA arī nodrošina \emph{paralelitāti} līdzīgi SIMD, ja
veicamā operācija tiek veikta vairākiem elementiem vienlaikus.
Peldošā komata skaitļu atbalsts ir iespējams, bet izmanto ievērojami vairāk
resursu~\cite{FPGA-fp}.

\subsubsection*{Operāciju izpildes paradigma}
\emph{CPU} pamatā ir viena izpildes pavediena secīgas izpildes ātrdarbības
palielināšana~\cite{Owens-GPU}.
Nosacījuma zarošanās ir tipiska operācija~\cite{Flynn-arch}\cite{Patterson},
kuras izpildei nav īpaši apsvērumi.
Šāda paradigma ir piemērota \emph{secīgos algoritmos},
piem.,~datu parsēšanas algoritmos un vadības programmās,
kur izpildes rezultāti un/vai veicamās operācijas ir atkarīgas
no iepriekšējo soļu rezultātiem. Nozīmīga CPU priekšrocība ir arī tā spēja
izpildīt jebkuru algoritmu, kuru spēj izpildīt GPU vai FPGA, bet ar
ierobežotu ātrdarbību, jo, vairumā gadījumu, tiek zaudēta paralelitāte%
\footnote{Pieejamās SIMD instrukcijas nodrošina
	ierobežotu (salīdzinājumā ar GPU un FPGA) paralelitāti.}.

\emph{GPU} pamatā ir vienas programmas operāciju veikšana \emph{paralēli}
lielam apjomam
datu elementu~\cite{Fatahalian}\cite{Owens-GPU}.
Lai sekmētu ātrdarbību GPU iespējami izmanto SIMD darbības,
kā arī straumes procesori ir apvienoti grupās kuri koplieto instrukciju
(programmas) atmiņu. Tas uzliek ierobežojumu uz nosacījuma zarošanos ---
tai jābūt koherentai visai datu kopai procesoru grupas ietvaros. Pretējā
gadījumā GPU ir spiests izpildīt abus (vai visus) secības variantus~\cite{Owens-GPU}.
Tādējādi GPU ir piemēroti algoritmi kuru operācijas nav atkarīgas no datiem,
kā arī datiem nav savstarpējās sakarības, t.i.,~tos var apstrādāt atsevišķi
un paralēli.

\emph{FPGA} nepiemīt izpildes paradigma --- to definē tā programmējums.
Tomēr, datu caurlaidspējas palielināšanai,
datu un/vai uzdevumu paralelitāte ir būtisks apsvērums
FPGA programmējuma izstrādē.

\subsubsection*{Atmiņas un komunikācijas saskarnes}
Platformas skaitļošanas jauda ir nebūtiska, ja to nevar nodrošināt ar datiem.
Atmiņas apakšsistēmai jābūt pietiekami ātrai, lai nodrošinātu datu plūsmu.
Heterogēnās sistēmās (sk.~\ref{sec:heterogenous}~nod.), kā, piemēram,
datorsistēmās ar CPU un GPU, ir būtiska arī komunikācijas saskarne datu
apmaiņai starp šīm platformām. Šādām sistēmām realizējot algoritmus,
lai efektīvi izmantotu skaitļošanas resursus, cenšas samazināt pārraidāmo
datu apjomu, vai pat izmanto datu kompresiju.
\cite{ACDA}\cite{OpenCL-book}

\subsubsection*{Programmēšanas rīki un valodas}
Vairums programmēšanas valodu ir attīstītas uz \emph{CPU} platformas pamata un
pieejamās programmēšanas valodas, bibliotēkas un rīki ir ļoti lielā
skaitā. Valodu konstrukcijas lielā mērā atspoguļo CPU secīgo instrukciju
izpildi. Autors, CPU programmēšanai izstrādes laikā, izvēlējās C un C++
programmēšanas valodas.

\emph{GPU} programmēšana vispārējus skaitļošanas uzdevumiem (GPGPU) ir jauna
tādēļ programmēšanas rīki GPU platformai ir ierobežoti. GPU gan nekad netiek
izmantots neatkarīgi, tādēļ GPGPU pamatā ir programmēšanas saskarne
kas dod pieeju GPU resursiem, bet programmas vadība ir CPU pārziņā.
Eksistē divas populāras GPU programmēšanas vides: \emph{OpenCL} un
Nvidia~\emph{CUDA}. CUDA pamatā ir C un C++ paplašinājumi un bibliotēkas,
kur ar speciālu kompilatoru var sagatavot lietojumprogrammu, kas izmanto
GPU resursus. Savukārt OpenCL definē programmēšanas valodu --- līdzīgu C ---
kuras pirmkods tiek kompilēts izpildes laikā (\termEn{Just-in-time} kompilācija)
izmantojot ražotāja piedāvātu rīku kopu ar OpenCL atbalsu.
\cite{Fatahalian}\cite{Owens-GPU}\cite{OpenCL-book}\cite{CUDA}

\emph{FPGA} programmējums tiek sagatavots izmantojot \emph{VHDL} vai
\emph{Verilog} aparatūras apraksta valodas, kas līdzinās programmēšanas
valodām, bet tās konstrukcijas atspoguļo FPGA uzbūvi, t.sk.,~operāciju
izpildes paralelitāti~\cite{JIS}.
Eksistē arī grafiskās vides, kur aparatūras aprakstu
var izstrādāt shematiski līdzīgi elektronisko shēmu sagatavošanai.
Aparatūras aprakstu kods tiek pārveidots programmējumā izmantojot
ražotāja sintēzes rīku~\cite{JIS}. Jauna tendence ir arī izmantot FPGA
heterogēnās sistēmās un to programmēt augstākā abstrakcijas līmenī ar
OpenCL~\cite{FPGA-ocl}.
