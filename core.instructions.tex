\subsection{Instrukciju kopa}
Instukciju kopa ir bāzēta uz RISC procesoru arhitektūrām, un
satur nedaudz vairāk par minimumu funkcionālam procesoram.
Instrukciju operāciju kodi ir vektoriski, lai efektīvi izmantotu pieejamo instrukcijas 
vārda garumu.

\small
\begin{longtable}[c]{lp{20ex}lp{0.36\textwidth}}
	%\centering
	\caption{Instrukciju tabula.}\label{tbl:instructions}\\
	\toprule
	\textbf{Apz.} & \textbf{Mašīnkods} & \textbf{Argumenti} & \textbf{Operācija} \\
	\midrule \endfirsthead
	\caption[]{\nameref{tbl:instructions}~(turpinājums).}\\
	%\toprule
	\midrule
	\textbf{Apz.} & \textbf{Mašīnkods} & \textbf{Argumenti} & \textbf{Operācija} \\
	\midrule \endhead
	\multicolumn{4}{c}{Atmiņas datu apmaiņas instrukcijas}\\
	\midrule
	\mnem{LD} & 	\instr{01}{00}{}{XXXXXX}{XXX}{XXX}{} & \texttt{rD, rS} &
		\texttt{rD = *(rS)} \newline
		{\footnotesize Ielasa vārdu no RAM reģistrā \texttt{rD}} \\ \midrule
	\mnem{LDI} & 	\instr{01}{01}{}{XXXXXX}{XXX}{}{XXX} \newline
					\instr{}{}{}{}{}{XXXXXXXXXXXXXXXX}{} & \texttt{rD, C} &
		\texttt{rD = C} \newline
		{\footnotesize Ielasa konstanti reģistrā \texttt{rD}} \\ \midrule
	\mnem{ST} & 	\instr{01}{11}{}{XXXXXX}{XXX}{XXX}{} & \texttt{rD, rS} &
		\texttt{*(rD) = rS} \newline
		{\footnotesize Ielasa vārdu no RAM reģistrā \texttt{rD}} \\
	\midrule \pagebreak[3]
	\multicolumn{4}{c}{Aritmētika, loģika un bitu bīdes (un \mnem{AR} operācijas saīsnes)}\\
	\midrule
	\mnem{AR} & 	\instr{10}{}{}{}{XXXX}{XXX}{X}\instr{}{}{}{}{XXX}{XXX}{} & \texttt{kA, kS, rD, rS} &
		{\footnotesize Aritmētikas/bīdes instrukcija.} \\ \midrule
	\rule{0pt}{1em}\mnem{ADD} & \instr{10}{0000}{000}{X}{XXX}{XXX}{} & \texttt{rD, rS} &
		\verb|rD = rD + rS| \\ \midrule
	\rule{0pt}{1em}\mnem{SUB} & \instr{10}{0001}{000}{X}{XXX}{XXX}{} & \texttt{rD, rS} &
		\verb|rD = rD - rS| \\ \midrule
	\rule{0pt}{1em}\mnem{INC} & \instr{10}{1000}{000}{X}{XXX}{}{XXX} & \texttt{rD} &
		\verb|rD = rD + 1| \\ \midrule
	\rule{0pt}{1em}\mnem{DEC} & \instr{10}{1001}{000}{X}{XXX}{}{XXX} & \texttt{rD} &
		\verb|rD = rD - 1| \\ \midrule
	\rule{0pt}{1em}\mnem{AND} & \instr{10}{0010}{000}{X}{XXX}{XXX}{} & \texttt{rD, rS} &
		\verb|rD = rD & rS| \\ \midrule
	\rule{0pt}{1em}\mnem{OR} & \instr{10}{0011}{000}{X}{XXX}{XXX}{} & \texttt{rD, rS} &
		\verb+rD = rD | rS+ \\ \midrule
	\rule{0pt}{1em}\mnem{XOR} & \instr{10}{0100}{000}{X}{XXX}{XXX}{} & \texttt{rD, rS} &
		\verb|rD = rD ^ rS| \\ \nopagebreak \midrule
	\rule{0pt}{1em}\mnem{NOT} & \instr{10}{1010}{000}{X}{XXX}{}{XXX} & \texttt{rD} &
		\verb|rD = ~rD| \\ \midrule
	\rule{0pt}{1em}\mnem{CLR} & \instr{10}{1111}{000}{X}{XXX}{}{XXX} & \texttt{rD} &
		\verb|rD = 0| \\ \midrule
	\rule{0pt}{1em}\mnem{MOV} & \instr{10}{0101}{000}{X}{XXX}{XXX}{} & \texttt{rD, rS} &
		\verb|rD = rS| \\ \midrule
	\rule{0pt}{1em}\mnem{LSL} & \instr{10}{0101}{001}{X}{XXX}{XXX}{} & \texttt{rD} &
		\verb|rD = rS * 2| \newline
		{\footnotesize Loģiskā kreisā bīde (reizina ar 2)} \\ \midrule
	\rule{0pt}{1em}\mnem{LSR} & \instr{10}{0101}{010}{X}{XXX}{XXX}{} & \texttt{rD} &
		\texttt{rD = rS / 2} \newline
		{\footnotesize Loģiskā labā bīde (dala ar 2)} \\ \midrule
	\rule{0pt}{1em}\mnem{ROL} & \instr{10}{0101}{011}{X}{XXX}{XXX}{} & \texttt{rD} &
		{\footnotesize Bitu rotācija pa kreisi} \\ \midrule
	\rule{0pt}{1em}\mnem{ROR} & \instr{10}{0101}{100}{X}{XXX}{XXX}{} & \texttt{rD} &
		{\footnotesize Bitu rotācija pa labi} \\ \nopagebreak
	\midrule \pagebreak[3]
	\multicolumn{4}{c}{Plūsmas kontroles instrukcijas}\\
	\midrule
	\mnem{NOP} & 	\instr{00}{}{}{00000000000000}{}{}{} & nav &
		{\footnotesize „Ne-operācija”} \\ \midrule
	\mnem{HLT} & 	\instr{11}{0011}{}{XXXX}{}{}{XXXXXX} & nav &
		{\footnotesize Apstādina procesora darbību} \\ \midrule
	\mnem{JMP} & 	\instr{11}{0010}{}{XXXX}{}{}{XXXXXX} \newline
					\instr{}{}{}{}{}{XXXXXXXXXXXXXXXX}{} & \texttt{L} &
		\texttt{PC = L} \newline
		{\footnotesize Beznosacījuma lēciens.} \\ \midrule
	\mnem{BREQ} & 	\instr{11}{1000}{}{XXXX}{XXX}{XXX}{} \newline
					\instr{}{}{}{}{}{XXXXXXXXXXXXXXXX}{} & \texttt{rD, rS, L} &
		\texttt{if(rD==rS) PC = L} \\ \midrule
	\mnem{BRNQ} & 	\instr{11}{1011}{}{XXXX}{XXX}{XXX}{} \newline
					\instr{}{}{}{}{}{XXXXXXXXXXXXXXXX}{} & \texttt{rD, rS, L} &
		\texttt{if(rD!=rS) PC = L} \\ \midrule
	\mnem{BRGT} & 	\instr{11}{1010}{}{XXXX}{XXX}{XXX}{} \newline
					\instr{}{}{}{}{}{XXXXXXXXXXXXXXXX}{} & \texttt{rD, rS, L} &
		\texttt{if(rD>rS) PC = L} \\ \midrule
	\mnem{BRGE} & 	\instr{11}{1001}{}{XXXX}{XXX}{XXX}{} \newline
					\instr{}{}{}{}{}{XXXXXXXXXXXXXXXX}{} & \texttt{rD, rS, L} &
		\texttt{if(rD>=rS) PC = L} \\ \midrule
	\mnem{BRLT} & 	\multicolumn{2}{c}{subst.~\texttt{\textbf{BRGE} rS, rD, L}} &
		\texttt{if(rD<rS) PC = L}\\ \midrule
	\mnem{BRLE} & 	\multicolumn{2}{c}{subst.~\texttt{\textbf{BRGT} rS, rD, L}} &
		\texttt{if(rD<=rS) PC = L}\\
	\bottomrule
	\caption*{\fboxrule=0.75pt \framebox{\footnotesize
		\begin{tabular}{ll}
			\multicolumn{2}{c}{Mašīnkoda krāsu apzīmējumi} \\
			\textcolor{purple}{\rule[-2pt]{1em}{1em}} Operāciju grupas kods &
			\textcolor{blue}{\rule[-2pt]{1em}{1em}} \textcolor{cyan}{\rule[-2pt]{1em}{1em}}
				Operācijas „vektora” kodi \\[2pt]
			\textcolor{lightgray}{\rule[-2pt]{1em}{1em}} Ignorētie biti &
			\textcolor{OliveGreen}{\rule[-2pt]{1em}{1em}} \textcolor{Green}{\rule[-2pt]{1em}{1em}}
				Argumentu biti \\
		\end{tabular}
		}}
\end{longtable}
\normalsize

Kā redzams instrukciju tabulā, instrukcijas
ir sadalītas grupās vai nu pēc nozīmes, vai pēc izpildes līdzības. Šo
instrukciju realizācijas detaļas tiks apskatītas turpmākajās apakšnodaļās.

\subsubsection{\mnem{AR} instrukcija} \label{sec:AR}
	Lai gan \mnem{AR} instrukcijas apzīmējums ir saīsinājums no „aritmētika”
	un tā ir instrukcijas primārā nozīme, \mnem{AR} implementē visas
	aritmētiskās, loģiskās, bīdes operācijas, kā arī reģistru 
	apmaiņas \mnem{MOV} instrukciju. (sk.~\ref{tbl:instructions}.~tabulu)
	Šāda implementācija izmantota tāpēc, ka visas šīs instrukcijas tiek
	pārvadītas pa to pašu signālceļu. Tādējādi kontroles iekārta tiek
	vienkāršota, jo atsevišķās operācijas nav nepieciešams izšķirt.
	
	Kontroles signāli aritmētiskajai un bīdes ierīcei,
	kā argumenti jeb vektora kodi,
	tiek nodoti tieši no instrukcijas vārda, un kontroles iekārtā netiek
	interpretēti (jeb ir „necaurspīdīgi”). Izņēmums, gan ir 
	unārās un bezargumentu operācijas
	\mnem{INC}, \mnem{DEC}, \mnem{NOT} un \mnem{CLR}, kuru ALU
	vektora koda vecākais bits (\texttt{=1}) tiek interpretēts, lai izlaistu
	otrā operanda ielasi, tā samazinot nepieciešamo takts ciklu skaitu
	instrukcijas izpildei. (sk.~\ref{kb:ARdecode}.~koda bloku)
	
	\begin{singlespace}
		\lstinputlisting[language={[qucs]VHDL},%float=pb,%
		                caption={\mnem{AR} instrukcijas dekodēšana. (\texttt{control2.vhd})},%
		                label=kb:ARdecode,%
		                firstnumber=150]
			{code/gen/ardecode-snippet.vhd}
	\end{singlespace}


\subsubsection{Nosacījuma zarošanās instrukcijas} \label{sec:branching}
	Nosacījuma zarošanās ir vitāla īpašība pilnīgam procesoram. Programmēšanas
	valodu \texttt{if()} konstrukcija nav iedomājama bez
	nosacījuma zarošanās.
	
	Izstrādātais procesors par nosacījumu var izmantot divu reģistru
	vienā\-dības vai nevienādības, kur reģistru saturs tiek interpretēts, kā
	bezzīmes veseli skaitļi (t.i.~\texttt{unsigned~integer} tips).
	Šai salīdzināšanai atbilst \mnem{BRxx} instrukcijas
	(sk.~\ref{tbl:instructions}.~tabulu), un pašu salīdzināšanu veic komparators
	(sk.~\ref{sec:comp}.~nod.) kuram tiek pievadīti salīdzināmie operandi.
	
	\noindent Aparatūras līmenī realizētas četras zarošanās instrukcijas:
	\begin{itemize}
		\item \mnem{BREQ} — zaroties, ja operandi ir vienādi;
		\item \mnem{BRNQ} — zaroties, ja operandi nav vienādi;
		\item \mnem{BRGE} — zaroties, ja pirmais operands ir lielāks
			vai vienāds ar otro;
		\item \mnem{BRGT} — zaroties, ja pirmais operands ir stigri lielāks
			par otro.
	\end{itemize}
	Šo instrukciju operāciju kodi nav izvēlēti gluži patvaļīgi, tie tiek
	izmantoti loģiskā izteiksmē kopā ar komparatora signāliem, lai noteiktu
	zarošanās izpildes nosacījumus.
	
	\begin{figure}[thp]
		\centering
		%\def\svgwidth{7cm}
		\def\svgscale{1.5}
		{\ttfamily\input{img/karnaugh.pdf_tex}}\\
		\(
			f = \mathtt{\overline{A}E + BG + AG + AB\overline{E}}
		\)\\[1ex]
		(apzīmējumu nozīmi sk.~\ref{kb:branchTest}.~kodā)
		\caption{Zarošanās nosacījuma \termEn{Karnaugh} karte un formula.}
		\label{fig:branch-karnaugh}
	\end{figure}
	
	\begin{singlespace}
		\lstinputlisting[language={[qucs]VHDL},%float=pb,%
		                caption={Zarošanās nosacījuma pārbaudes funkcija (\texttt{control2.vhd})},%
		                label=kb:branchTest,%
		                linerange={61-69},firstnumber=61,
		                emph={state,branchTest},%
		                breaklines,breakatwhitespace,
		                basicstyle=\ttfamily\scriptsize]
			{code/control2.vhd}
	\end{singlespace}
	
	Atlikušās zarošanās instrukcijas \mnem{BRLT} un \mnem{BRLE} ir
	programmatūras līmeņa substitūcijas, jo to ekvivalentus var iegūt
	izmantojot attiecīgi \mnem{BRGE} un \mnem{BRGT} instrukcijas,
	apmainot salīdzināmos operandus vietām.
	
	
