\subsection{Instrukciju kopa} \label{sec:instrSet}
%Instruckiju kopa ir izstrādāta no jauna un piedāvā vienkāršotu,
%mini\-mālu kopu līdzīgi RISC (\termEn{Reduced Instruction Set Computing})
%filo\-so\-fijai. Salīdzinot ar Perija realizāciju, ievērojami samazinātas
%nosacījuma zarošanās instrukcijas un instrukcijas pieejamie 16 biti
%tiek efektīvāk izmantoti ieviešot vektorizētus jeb dažāda garuma
%operāciju kodus. (detalizētākam aprakstam sk.~\ref{sec:instrSet}~nod.)

Instruckiju kopa ir izstrādāta no jauna un piedāvā vienkāršotu kopu, kas
satur pamata darbības ar bezzīmes veseliem (\texttt{unsigned integer} tipa)
skaitļiem, RAM ielasīšanas un ierakstes instrukcijas, un zarošanās
instrukcijas (sk.~\ref{tbl:instructions}~tabulu).

Instrukcijas operāciju kodi ir dažāda garuma, jeb vektoriski\footnote{%
	Termins aizgūts no programmēšanas paņēmiena, kur viena funkcija
	satur dažādu operāciju kopu (vektoru), un pēc nodotā argumenta tiek
	izpildīta viena operācija (mainot pārējo argumentu nozīmi).},
t.i.~operāciju kods sadalīts ,,grupas'' kodā un ,,vektora'' kodā, kur
grupas kods ir konstanta garuma (2~biti), bet vektora kods ir konstanta
garuma tikai piederošās grupas ietvaros. Strikti ņemot, varētu uzskatīt, ka
vektora kods ir arguments, bet ir piederīgs operācijas
kodam, jo abas tā daļas tiek interpretētas dekodēšanas solī.\footnote{%
	Izņemot \mnem{AR} instrukciju (sk.~\ref{sec:AR}~nod.).}

\begin{singlespace}\small
\begin{longtable}[c]{lp{20ex}lp{0.36\textwidth}}
	%\centering
	\caption{Instrukciju tabula.}\label{tbl:instructions}\\
	\toprule
	\textbf{Apz.} & \textbf{Mašīnkods} & \textbf{Argumenti} & \textbf{Operācija} \\
	\midrule \endfirsthead
	\caption[]{\nameref{tbl:instructions}~(turpinājums).}\\
	%\toprule
	\midrule
	\textbf{Apz.} & \textbf{Mašīnkods} & \textbf{Argumenti} & \textbf{Operācija} \\
	\midrule \endhead
	\multicolumn{4}{c}{Atmiņas datu apmaiņas instrukcijas}\\
	\midrule
	\mnem{LD} & 	\instr{01}{00}{}{XXXXXX}{XXX}{XXX}{} & \texttt{rD, rS} &
		\texttt{rD = *(rS)} \newline
		{\footnotesize Ielasa vārdu no RAM reģistrā \texttt{rD}} \\ \midrule
	\mnem{LDI} & 	\instr{01}{01}{}{XXXXXX}{XXX}{}{XXX} \newline
					\instr{}{}{}{}{}{XXXXXXXXXXXXXXXX}{} & \texttt{rD, C} &
		\texttt{rD = C} \newline
		{\footnotesize Ielasa konstanti reģistrā \texttt{rD}} \\ \midrule
	\mnem{ST} & 	\instr{01}{11}{}{XXXXXX}{XXX}{XXX}{} & \texttt{rD, rS} &
		\texttt{*(rD) = rS} \newline
		{\footnotesize Ielasa vārdu no RAM reģistrā \texttt{rD}} \\
	\midrule \pagebreak[3]
	\multicolumn{4}{c}{Aritmētikās, loģikās un bitu bīdes instrukcijas (un \mnem{AR} operācijas saīsnes)}\\
	\midrule
	\mnem{AR} & 	\instr{10}{}{}{}{XXXX}{XXX}{X}\instr{}{}{}{}{XXX}{XXX}{} & \texttt{kA, kS, rD, rS} &
		{\footnotesize Aritmētikas/bīdes instrukcija.} \\ \midrule
	\rule{0pt}{1em}\mnem{ADD} & \instr{10}{0000}{000}{X}{XXX}{XXX}{} & \texttt{rD, rS} &
		\verb|rD = rD + rS| \\ \midrule
	\rule{0pt}{1em}\mnem{SUB} & \instr{10}{0001}{000}{X}{XXX}{XXX}{} & \texttt{rD, rS} &
		\verb|rD = rD - rS| \\ \midrule
	\rule{0pt}{1em}\mnem{INC} & \instr{10}{1000}{000}{X}{XXX}{}{XXX} & \texttt{rD} &
		\verb|rD = rD + 1| \\ \midrule
	\rule{0pt}{1em}\mnem{DEC} & \instr{10}{1001}{000}{X}{XXX}{}{XXX} & \texttt{rD} &
		\verb|rD = rD - 1| \\ \midrule
	\rule{0pt}{1em}\mnem{AND} & \instr{10}{0010}{000}{X}{XXX}{XXX}{} & \texttt{rD, rS} &
		\verb|rD = rD & rS| \\ \midrule
	\rule{0pt}{1em}\mnem{OR} & \instr{10}{0011}{000}{X}{XXX}{XXX}{} & \texttt{rD, rS} &
		\verb+rD = rD | rS+ \\ \midrule
	\rule{0pt}{1em}\mnem{XOR} & \instr{10}{0100}{000}{X}{XXX}{XXX}{} & \texttt{rD, rS} &
		\verb|rD = rD ^ rS| \\ \nopagebreak \midrule
	\rule{0pt}{1em}\mnem{NOT} & \instr{10}{1010}{000}{X}{XXX}{}{XXX} & \texttt{rD} &
		\verb|rD = ~rD| \\ \midrule
	\rule{0pt}{1em}\mnem{CLR} & \instr{10}{1111}{000}{X}{XXX}{}{XXX} & \texttt{rD} &
		\verb|rD = 0| \\ \midrule
	\rule{0pt}{1em}\mnem{MOV} & \instr{10}{0101}{000}{X}{XXX}{XXX}{} & \texttt{rD, rS} &
		\verb|rD = rS| \\ \midrule
	\rule{0pt}{1em}\mnem{LSL} & \instr{10}{0101}{001}{X}{XXX}{XXX}{} & \texttt{rD} &
		\verb|rD = rS * 2| \newline
		{\footnotesize loģiskā kreisā bīde (reizina ar 2)} \\ \midrule
	\rule{0pt}{1em}\mnem{LSR} & \instr{10}{0101}{010}{X}{XXX}{XXX}{} & \texttt{rD} &
		\texttt{rD = rS / 2} \newline
		{\footnotesize loģiskā labā bīde (dala ar 2)} \\ \midrule
	\rule{0pt}{1em}\mnem{ROL} & \instr{10}{0101}{011}{X}{XXX}{XXX}{} & \texttt{rD} &
		{\footnotesize bitu rotācija pa kreisi} \\ \midrule
	\rule{0pt}{1em}\mnem{ROR} & \instr{10}{0101}{100}{X}{XXX}{XXX}{} & \texttt{rD} &
		{\footnotesize bitu rotācija pa labi} \\ \nopagebreak
	\midrule \pagebreak[3]
	\multicolumn{4}{c}{Plūsmas kontroles instrukcijas}\\
	\midrule
	\mnem{NOP} & 	\instr{00}{}{}{00000000000000}{}{}{} & nav &
		{\footnotesize tukša operācija} \\ \midrule
	\mnem{HLT} & 	\instr{11}{0011}{}{XXXX}{}{}{XXXXXX} & nav &
		{\footnotesize apstādina procesora darbību} \\ \midrule
	\mnem{JMP} & 	\instr{11}{0010}{}{XXXX}{}{}{XXXXXX} \newline
					\instr{}{}{}{}{XXXXXXXXXXXXXXXX}{}{} & \texttt{L} &
		\texttt{PC = L} \newline
		{\footnotesize beznosacījuma lēciens.} \\ \midrule
	\mnem{BREQ} & 	\instr{11}{1000}{}{XXXX}{XXX}{XXX}{} \newline
					\instr{}{}{}{}{XXXXXXXXXXXXXXXX}{}{} & \texttt{rD, rS, L} &
		\texttt{if(rD==rS) PC = L} \\ \midrule
	\mnem{BRNQ} & 	\instr{11}{1011}{}{XXXX}{XXX}{XXX}{} \newline
					\instr{}{}{}{}{XXXXXXXXXXXXXXXX}{}{} & \texttt{rD, rS, L} &
		\texttt{if(rD!=rS) PC = L} \\ \midrule
	\mnem{BRGT} & 	\instr{11}{1010}{}{XXXX}{XXX}{XXX}{} \newline
					\instr{}{}{}{}{XXXXXXXXXXXXXXXX}{}{} & \texttt{rD, rS, L} &
		\texttt{if(rD>rS) PC = L} \\ \midrule
	\mnem{BRGE} & 	\instr{11}{1001}{}{XXXX}{XXX}{XXX}{} \newline
					\instr{}{}{}{}{XXXXXXXXXXXXXXXX}{}{} & \texttt{rD, rS, L} &
		\texttt{if(rD>=rS) PC = L} \\ \midrule
	\mnem{BRLT} & 	\multicolumn{2}{c}{subst.~\texttt{\textbf{BRGE} rS, rD, L}} &
		\texttt{if(rD<rS) PC = L}\\ \midrule
	\mnem{BRLE} & 	\multicolumn{2}{c}{subst.~\texttt{\textbf{BRGT} rS, rD, L}} &
		\texttt{if(rD<=rS) PC = L}\\
	\bottomrule
	\caption*{\fboxrule=0.75pt \framebox{\footnotesize
		\begin{tabular}{ll}
			\multicolumn{2}{c}{Mašīnkoda krāsu apzīmējumi} \\
			\textcolor{purple}{\rule[-2pt]{1em}{1em}} Operāciju grupas kods &
			\textcolor{blue}{\rule[-2pt]{1em}{1em}} \textcolor{cyan}{\rule[-2pt]{1em}{1em}}
				Operācijas vektora kods \\[2pt]
			\textcolor{lightgray}{\rule[-2pt]{1em}{1em}} Ignorētie biti &
			\textcolor{OliveGreen}{\rule[-2pt]{1em}{1em}} \textcolor{Green}{\rule[-2pt]{1em}{1em}}
				Argumentu biti \\
		\end{tabular}
		}}
\end{longtable}
\end{singlespace}
\normalsize

%\pagebreak
Instrukcijas
ir sadalītas grupās vai nu pēc nozīmes, vai pēc izpildes līdzības. Šo
instrukciju realizācijas detaļas tiks apskatītas turpmākajās apakšnodaļās.

\subsubsection{\mnem{AR} instrukcija} \label{sec:AR}
	Lai gan \mnem{AR} instrukcijas apzīmējums ir saīsinājums no „aritmētika”
	un tā ir instrukcijas primārā nozīme, \mnem{AR} implementē visas
	aritmētiskās, loģiskās, bīdes operā\-cijas, kā arī reģistru 
	apmaiņas \mnem{MOV} instrukciju (sk.~\ref{tbl:instructions}~tabulu).
	Šāda imple\-men\-tā\-cija izmantota tāpēc, ka visas šīs instrukcijas tiek
	pārvadītas pa to pašu signālceļu. Tādējādi kontroles iekārta tiek
	vienkāršota, jo atsevišķās operācijas nav nepieciešams izšķirt.
	
	Kontroles signāli aritmētiskajai un bīdes ierīcei kā argumenti jeb vektora kodi,
	tiek nodoti tieši no instrukcijas vārda, un kontroles iekārtā netiek
	interpretēti (jeb ir „necaurspīdīgi”). Izņēmums, gan ir vektora koda
	vecākais bits, kurš tiek interpretēts un pie augsta stāvokļa (loģiskā |1|)
	tiek apieta otrā operanda (\texttt{rS}) ielase samazinot nepieciešamo
	takts ciklu skaitu instrukcijas izpildei (sk.~\ref{kb:ARdecode}~koda bloku).
	Šis bits ir |1|	unārājām un bezargumentu instrukcijām
	\mnem{INC}, \mnem{DEC}, \mnem{NOT} un \mnem{CLR}.%
	\footnote{Instrukcijas \mnem{LSL}, \mnem{LSR}, \mnem{ROL} un \mnem{ROR}
		nav īsti unāras, bet asemblējot šīs saīsnes izmanto to argumentu gan kā
		\texttt{rS}, gan kā \texttt{rD}.}
	
	\begin{singlespace}
		\lstinputlisting[language={[qucs]VHDL},%float=pb,%
		                caption={\mnem{AR} instrukcijas dekodēšana (izgriezums).},%
		                label=kb:ARdecode,%
		                firstnumber=150]
			{code/gen/ardecode-snippet.vhd}
	\end{singlespace}
	
	\mnem{AR} instrukcijas saīsnes nenosedz visas iespējamās darbības, kuras
	iespējamas ar \mnem{AR}. Tā kā ALU un Bitu bīdes loģiskā ierīce ir
	atsevišķas komponentes uz viena signālceļa, tās darbības ir izpildāmas
	vienlaicīgi. Tādējādi, izmantojot \mnem{AR} pilno formu, iespējams
	kombinēt aritmētiskās un bīdes operācijas (piem. saskaitīt un veikt bīdi)
	izpildei vienā instrukcijā. Jāņem vērā, ka Bitu bīdes loģiskā ierīce
	atrodas aiz ALU (sk.~\ref{fig:cpu-rev3}~att.),
	tādēļ bitu bīde vienmēr tiek izpildīta aritmētiskās darbības rezultātam.


\subsubsection{Nosacījuma zarošanās instrukcijas} \label{sec:branching}
	Nosacījuma zarošanās ir vitāla īpašība procesoram. Programmēšanas
	valodu \texttt{if()} konstrukcija nav iedomājama bez
	nosacījuma zarošanās.
	
	Izstrādātais procesors par zarošanās nosacījumu var izmantot divu reģistru
	satura apzīmēto bezzīmes veselo skaitļu salīdzinājumu.
	Šai salīdzināšanai atbilst \mnem{BRxx} instrukcijas
	(sk.~\ref{tbl:instructions}~tabulu), un pašu salīdzināšanu veic komparators
	(sk.~\ref{sec:comp}~nod.) kuram tiek pievadīti salīdzināmie operandi.
	
	\pagebreak[2]
	Aparatūras līmenī realizētas četras zarošanās instrukcijas:
	\begin{itemize}
		\item \mnem{BREQ} — zaroties, ja operandi ir vienādi;
		\item \mnem{BRNQ} — zaroties, ja operandi nav vienādi;
		\item \mnem{BRGE} — zaroties, ja pirmais operands ir lielāks
			vai vienāds ar otro;
		\item \mnem{BRGT} — zaroties, ja pirmais operands ir stingri lielāks
			par otro.
	\end{itemize}
	
	Zarošās nosacījumā pārbaudei tiek izmantoti šo instrukciju operāciju
	kodu divi jaunākie biti, kuri turpmāk šajā nodaļā apzīmēti ar
	\texttt{A} (otrs jaunākais bits) un \texttt{B} (jaunākais bits).
	
	Šo četru instrukciju operāciju kodi nav izvēlēti gluži patvaļīgi,
	bet pielāgoti tā, lai to \texttt{A} un \texttt{B} biti,
	kopā ar komparatora signāliem |eq| un |gr|
	(apz.~\texttt{E} un \texttt{G}),
	veidotu pēc iespējas vienkāršāku loģisko izteiksmi,
	kas izsaka zarošanās nosacījuma izpildi (sk.~\ref{fig:branch-karnaugh}~att.).
	
	\begin{figure}[thp]
		\centering
		%\def\svgwidth{7cm}
		\def\svgscale{1.5}
		{\ttfamily\input{img/karnaugh.pdf_tex}}\\
		\(
			f = \mathtt{\overline{A}E + BG + AG + AB\overline{E}}
		\)\\ %[1ex]
		%(apzīmējumu nozīmi sk.~\ref{kb:branchTest}.~kodā)
		\caption{Zarošanās nosacījuma Karno karte un formula.}
		\label{fig:branch-karnaugh}
	\end{figure}
	
	Zarošanās nosacījuma pārbaude implementēta kontroles iekārtā kā
	VHDL funkcija (sk.~\ref{kb:branchTest}~pirmkodu).
	
	\begin{singlespace}
		\lstinputlisting[language={[qucs]VHDL},%float=pb,%
		                caption={Zarošanās nosacījuma pārbaudes funkcija (izgriezums).},%
		                label=kb:branchTest,%
		                linerange={61-69},firstnumber=61,
		                emph={state,branchTest},%
		                breaklines,breakatwhitespace,
		                basicstyle=\ttfamily\scriptsize]
			{code/control2.vhd}
	\end{singlespace}
	
	Atlikušās zarošanās instrukcijas \mnem{BRLT} un \mnem{BRLE} ir
	programmatūras līmeņa substitūcijas, jo to ekvivalentus var iegūt
	izmantojot attiecīgi \mnem{BRGE} un \mnem{BRGT} instrukcijas,
	apmainot salīdzināmos operandus vietām assemblēšanas brīdī.
	
	
