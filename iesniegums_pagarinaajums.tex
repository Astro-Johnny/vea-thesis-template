\documentclass[10pt,a4paper]{article}
\usepackage{polyglossia}	% XeTeX multi-lingual support
\usepackage{fontspec}		% The XeTeX font spec package
\usepackage{xunicode}		% XeTeX Unicode character support
\usepackage{xltxtra}		% Umm something else XeTeX

\usepackage{amsmath}	% Math stuff
\usepackage[dvipsnames]{xcolor}
\usepackage{booktabs}	% Package for nicer tables

% Polyglossia settings
\setmainlanguage{latvian}

\usepackage[top=25mm, bottom=25mm, left=30mm, right=20mm]{geometry}


% FONTS (Empty \setmainfont gives you new Computer Modern)
% Deja Vu Family (rather good but wide chars)
%\setmainfont{DejaVu Serif}
%\setsansfont{DejaVu Sans}

% Liberation Family (my current favourite)
%\setmainfont{Liberation Serif}
%\setsansfont{Liberation Sans}

% Monos
%\setmonofont{DejaVu Sans Mono}

%\newfontfamily\TitleFont[Color=4A6161]{Liberation Serif}

% External common image library
\graphicspath{{/home/johnlm/Development/resources/tex_graphics_lib/}}

%Don't use hypentenation in this document
\hyphenpenalty=10000

% Set rubberband values
%\setlength{\parskip}{1ex plus 3ex minus 1ex}	% Space between paragraphs
%\setlength{\lineskip}{4pt plus 1pt minus 2pt}	% Space between lines

% Chardump: „” em— en– fig‒ “”
\begin{document}
	\sloppy
	\pagestyle{empty}
	\begin{minipage}[t]{0.47\textwidth}
		\begin{flushleft}
			\textbf{Ventspils Augstskolas}\\
			Informāciju Tehnoloģiju fakultātes\\
			Bakalaura studiju programmas „Elektronika”\\
			3.~kursa students \textbf{Jānis Šmēdiņš}\\
			p.k.~\texttt{040989-11652}\\
			matr.n.~\texttt{2009120280}\\
			mob.t.~\texttt{+371 29937190}
		\end{flushleft}
	\end{minipage}
	\begin{minipage}[t]{0.47\textwidth}
		\begin{flushright}
			\textbf{Ventspils Augstskolas}\\
			Informāciju Tehnoloģiju\\
			fakultātes dekānei\\
			asoc.~prof.~\textbf{G.~Hiļķevičai}
		\end{flushright}
	\end{minipage}\\[3em]
	
	\vspace{1cm}
	\begin{center}
		\Large\scshape Iesniegums
	\end{center}
	
	%\noindent
	Lūdzu pagarināt bakalaura darba nodošanas termiņu 
	līdz 2012.~gada 25.~maiju saistībā ar 
	nepieciešamību izstrādes prototipam saskaņot aiztures starp modeļiem
	lai sasniegtu sintezējamu rezultātu.
	
	
	\vspace{2cm}
	
	\begin{minipage}[t]{0.32\textwidth}
		\raggedright Saskaņots\\[3ex]
		Darba vadītājs:\\[3ex]
		Students:%\\[3ex]
	\end{minipage}
	\begin{minipage}[t]{0.32\textwidth}
		\centering \hfill \\[3ex]
		\color{gray}\rule[-2pt]{10em}{1pt}\\[3ex]
		\color{gray}\rule[-2pt]{10em}{1pt}
	\end{minipage}
	\begin{minipage}[t]{0.32\textwidth}
		\raggedleft \today \\[3ex]
		\scshape 
		G.~Gaigals\\[3ex]
		J.~Šmēdiņš
	\end{minipage}
		
\end{document}
