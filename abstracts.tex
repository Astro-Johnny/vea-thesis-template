%% Abstracts
% The 'abstract' enviroment makes more problems than it is useful

\phantomsection\addcontentsline{toc}{section}{\abstractname}
\abstitlestyle{\abstractname} % The abstract title (centered)
% Abstract text goes here
\noindent%
\begin{tabularx}{\textwidth}{lX}
	\textbf{Darba nosaukums:} & 
		\textit{Attēlu raksturpunktu pāru noteikšanas algoritmu
		        ātrdarbības izpēte dažādām aparatūras platformām}\\[1ex]
	\textbf{Darba autors:} & Jānis Šmēdiņš\\[1ex]
	\textbf{Darba vadītājs:} & Dr.~sc.~comp.~Kaspars Sudars\\[1ex]
	\textbf{Darba apjoms:} & 43~lpp., 1~tabula, 21~attēls, 1~pirmkoda izdruka,
	                         28~biblogrāfiskie avoti, 6~pielikumi\\[1ex]
	\textbf{Atslēgas vārdi:} & 
		skaitļošanas aparatūras platformas,
		raksturpunktu pāru noteikšanas ātrdarbība,
		ORB algoritms,
		FAST algoritms
		%algoritmu ātrdarbības testēšana,
\end{tabularx}

\vspace{1em}
Šī maģistra darba mērķis ir analizēt attēlu raksturpunktu pāru algoritmu
ātrdarbību CPU, GPU un FPGA platformās, novērtējot iespējamos
ātrdarbības uzlabojums. No vairākiem pieejamiem algoritmiem, analīzei
izvēlēts ORB algoritms.

ORB algoritms ir nozīmīgs algoritms, tā salīdzinoši zemās skaitļošanas
kompleksitātes dēļ, kas ļauj to potenciāli izmantot sistēmās ar ierobežotu
veiktspēju, galvenokārt iegultajās sistēmās (\termEn{embedded systems})
attēlu reāllaika apstrādei.

Darbā veikts CPU, GPU un FPGA aparatūras platformu salīdzinājums, kā arī
veikta ORB algoritma ātrdarbības salīdzinājums šīm platformām.
Padziļināta analīze tiek veikta FAST algoritmam,
kas ir izmantots ORB algoritmā kā komponente raksturpunktu detektēšanai.
Darbā izstrādāta FPGA implementācija FAST algoritmam, izveidots
ORB algoritma implementācijas modelis un 
pētījuma laikā tika konstatēts potenciālais
ātrdarbības uzlabojums tieši FPGA platformai.



%~ ORB tiek izdalīts pa daļām, kur katra komponente tiek
%~ apskatīta atsevišķi un empīriski vai teorētiski salīdzināta to ār

%~ Darba galvenais mērķis ir analizēt algoritmus un aparatūras platformas,
%~ potenciāli lokalizējot iespējas ātrdarbības uzlabošanai. Mērķa sasniegšanai
%~ izvirzītie uzdevumi ir:

\clearpage
\begin{english} % The English abstract
	\phantomsection\addcontentsline{toc}{section}{\abstractname}
	\abstitlestyle{\abstractname} % The abstract title (centered)
	% Abstract text goes here
	\noindent%
	\begin{tabularx}{\textwidth}{lX}
		\textbf{Title:} & 
			\textit{Performance analysis of keypoint matching algorithms on
			        different hardware platforms}\\[1ex]
		\textbf{Author:} & Jānis Šmēdiņš\\[1ex]
		\textbf{Supervisor:} & Dr.~sc.~comp.~Kaspars Sudars\\[1ex]
		\textbf{Extent:} & 43~pages, 1~table, 21~figures, 1~listing,
	                       28~bibliographical references, 6~appendices\\[1ex]
		\textbf{Keywords:} &
			computing hardware,
			keypoint matching performance,
			ORB algorithm,
			FAST algorithm
	\end{tabularx}
	
	\vspace{1em}
	
	The purpose of this master's thesis is performance analysis of
	keypoint matching algorithms on CPU, GPU and FPGA
	hardware platforms to determine
	the potential performance improvements. Out of many available algorithms
	the ORB was chosen for tests and analysis.
	
	ORB's main advantage is its computational efficiency compared to other
	algorithms of same class. This is especially important for 
	real-time image data processing on reduced
	performance systems such as embedded systems.
	
	An analysis and comparison of CPU, GPU and FPGA hardware platforms was
	carried out, with emphasis on performance effects on algorithm
	implementations. The ORB algorithm performance on these
	platforms is compared. A separate analysis was performed for FAST algorithm,
	which is used in ORB as keypoint detection component.
	An FPGA implementation was developed for FAST algorithm and
	an FPGA implementation
	model defined for ORB. The analysis revealed significant performance
	improvements by using FPGA technology for algorithm implementation.
\end{english}

\clearpage
\begin{russian} % The Russian abstract
	\phantomsection\addcontentsline{toc}{section}{\abstractname}
	\abstitlestyle{\abstractname} % The abstract title (centered)
	% Abstract text goes here
	\noindent%
	\begin{tabularx}{\textwidth}{lX}
		\textbf{Название работы:} & 
			\textit{Исследование быстродействия алгоритмов определения пар
			характерных точек изображений для различных аппаратных платформ.}\\[1ex]
		\textbf{Автор работы:} & Jānis Šmēdiņš (транслит.~Янис Шмэдиньш)\\[1ex]
		%\textbf{Руководитель работы:}
		\textbf{Руководитель:} & Dr.~sc.~comp.~Kaspars Sudars 
			(транслит.~Каспарс Сударс)\\[1ex]
		\textbf{Размер:} & 43~стр., 1~таблица, 21~изображений, 1~листинг,
	                       28~библиографических источников, 6~приложений\\[1ex]
		\textbf{Ключевые слова:} & 
			вычислительные платформы,
			быстродействие определения пар характерных точек,
			алгоритм ORB,
			алгоритм FAST
	\end{tabularx}
	
	\vspace{1em}
	Целью настоящей работы является анализ быстродействия отобранного ORB
	алгоритма, применяемого для определения пар характерных точек изображений
	в среде платформ CPU, GPU и FPGA, и сравнительная оценка возможного
	улучшения его быстродействия.
	
	Алгоритм ORB --- это важный алгоритм ввиду его сравнительно невысокой
	вычислительной сложности, что позволяет его  потенциально использовать
	в системах с ограниченной производительностью, главным образом в
	встроенных системах (\termEn{embedded systems})
	обработки изображений реального времени.
	
	В работе проведено сравнение как самих аппаратных платформ
	CPU, GPU и FPGA, так и быстродействия ORB алгоритма на этих платформах.
	Проводится углубленный анализ алгоритма FAST, используемого в ORB
	алгоритме в качестве компоненты определения характерных точек.
	В работе реализовано FPGA исполнение алгоритма FAST, разработана
	модель исполнения ORB алгоритма и в результате исследования
	констатировано потенциальное увеличение его быстродействия именно в
	случае использования платформы FPGA.
\end{russian}
