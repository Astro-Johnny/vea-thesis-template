\subsection{Komponentes}
Kā redzams procesora uzbūves shēmās
(\ref{fig:controlPipeline}.~un \ref{fig:aluPipeline}.~attēls), procesors
sastāv no dažādām apakš-komponentēm. Lai panāktu darba portabilitāti,%
\footnote{Iespējamu projekta izmantošanu dažādos izstrādes rīkos.}
komponentes aprakstītas RTL (\termEn{Register Transfer Level}) stila
VHDL apraksta kodā.

\subsubsection{Reģistri}
	Viena no vienkāršākajām komponentēm ir reģistrs. Tā funkcija ir uzglabāt
	viena vārda datus — šajā gadījumā 16 bitu vārda — un atjaunot to pēc
	pieprasījuma (ar \texttt{clk} signālu).
	
	\begin{figure}[bh]
		\centering
		%\def\svgwidth{7cm}
		\def\svgscale{1.25}
		{\ttfamily\scriptsize\input{img/sub-reg.pdf_tex}}
		\caption{Reģistrs.}
		\label{fig:reg}
	\end{figure}
	
	\noindent Procesorā ir vairāki atsevišķi reģistri ar savu nozīmi:
	\begin{description}
		\item[\texttt{PC} — Programmskaitītājs] \hfill \\
			Reģistrs, kas uzglabā adresi aktīvajai programmas pozīcijai,
			t.i.~parasti izpildāmās instrukcijas adresi. Secīgās programmas
			izpildes laikā \texttt{PC} tiek palielināts par 1%
			\footnote{Konkrētāk — \texttt{PC} tiek palielināts par
				izpildīto instrukcijas vārdu skaitu. (1 vai 2)},
			savukārt lēcieni (zarošanās)
			programmā realizēti pārrakstot programmskaitītāja saturu.
		\item[\texttt{instrReg} — Instrukciju reģistrs] \hfill \\
			Reģistrs, kurā ieraksta izpildāmo instrukcijas kodu.
			Nepieciešams, lai kontrole stāvokļa mašīna „neaizmirstu”
			izpildāmo operāciju un tās argumentus izpildot mikrokoda soļus.
		\item[\texttt{addrReg} — Atmiņas adresācijas reģistrs] \hfill \\
			Reģistrs, kurā ieraksta adresējamo atmiņas adresi tās nolasei
			vai ierakstei. Nepieciešams lai izvairītos no
			cirkulārās loģikas ielasot datus no atmiņas.
		\item[\texttt{opReg} — Operanda reģistrs] \hfill \\
			Reģistrs, kas uzglabā vienu no aritmētiskās/loģiskās darbības
			ope\-ran-diem, kamēr tiek iegūts otrs operands.
		\item[\texttt{outReg} — Izejas reģistrs] \hfill \\
			Reģistrs, kurā ieraksta aritmētiskās/loģiskās darbības rezultātu.
			Nepieciešams lai izvairītos no cirkulārās loģikas
			ierakstot rezultātu operatīvajos reģistros.
	\end{description}
	
	\begin{singlespace}
		\lstinputlisting[language={[qucs]VHDL},%float=pb,%
		                caption={Reģistra VHDL apraksts. (\texttt{reg.vhd})},%
		                label=kb:reg]
			{code/reg.vhd}
	\end{singlespace}

\pagebreak[3]
\subsubsection{Operatīvo reģistru masīvs}
	Operatīvie reģistri ir adresējama reģistru kopa, kura, atšķirībā no
	iepriekš apskatītajiem procesora specializētajiem reģistriem,
	ir pieejama	programmētājam tiešai modifikācijai.\\
	Tos sauc arī par vispārēja pielietojuma reģistriem, jo tiem nav noteikta specializēta
	nozīme, tā vietā programmētājs tos lieto pēc vajadzības, piem.~mainīgo
	uzglabāšanai.
	
	\begin{figure}[thp]
		\centering
		%\def\svgwidth{7cm}
		\def\svgscale{1.25}
		{\ttfamily\scriptsize\input{img/sub-regArray.pdf_tex}}
		\caption{Reģistru masīvs.}
		\label{fig:regArray}
	\end{figure}
	\begin{singlespace}
		\lstinputlisting[language={[qucs]VHDL},%float=pb,%
		                caption={Reģistru masīva VHDL apraksts. (\texttt{regarray2.vhd})},%
		                label=kb:regArray,%
		                emph={t_ram}]
			{code/regarray2.vhd}
	\end{singlespace}
	
	\pagebreak[2]
	Šajā realizācijā, reģistru masīvs vairāk līdzinās 8 vārdu operatīvajai
	atmiņai, bet netiek realizēta rakstīšanas vai lasīšanas režīma
	pārslēgšana. Tā vietā ir ieejas un izejas pieslēgvietas, kur
	ierakstītie dati tiek saglabāti un uzreiz izlikti uz izvadi.

\subsubsection{Multipleksors}
	Komponentēm, kurām datu šinā nepieciešams saņemt datus no dažādiem
	devējiem tiek multipleksētas, t.i.~devēju ieejas signāls tiek pārslēgts
	pēc nepieciešamības.
	
	Šim nolūkam izmantots multipleksors. Atšķirībā no citām procesora
	komponentēm multipleksors izveidots ar \termEn{Actel Libero}
	programmatūras pieejamo multipleksora makrosu, kura instancējamā
	entītijas definīcija parādīta \ref{kb:mux}.~koda blokā.
	
	\begin{figure}[thp]
		\centering
		%\def\svgwidth{7cm}
		\def\svgscale{1.25}
		{\ttfamily\scriptsize\input{img/sub-mux.pdf_tex}}
		\caption{2 ieeju multipleksors.}
		\label{fig:mux2}
	\end{figure}
	
	\begin{singlespace}
		\lstinputlisting[language={[qucs]VHDL},%float=pb,%
		                 caption={Multipleksora makrosa entītijas definīcija.},%
		                 label=kb:mux,%
		                 linerange={8-12},firstnumber=8]
			{code/gen/muxiitis.vhd}
	\end{singlespace}

\pagebreak[3]
\subsubsection{Aritmētiski loģiskā ierīce}
	Aritmētiski loģiskā ierīce jeb ALU ir viena no galvenajām,
	procesora definējošām
	komponentēm. Tās uzdevums ir veikt doto datu apstrādi, kā arī
	palielināt programmskaitītāju secīgas programmas izpildes nodrošināšanai.
	
	\begin{figure}[thp]
		\centering
		%\def\svgwidth{7cm}
		\def\svgscale{1.25}
		{\ttfamily\scriptsize\input{img/sub-alu.pdf_tex}}
		\caption{Aritmētiski loģiskā ierīce.}
		\label{fig:alu}
	\end{figure}
	
	Lai gan klasiski bīdes operācijas arī tiek realizētas Aritmētiski
	loģiskajā ierīcē, šajā gadījumā ALU veic tikai aritmētiskās un bitu
	loģikas darbības, atstājot bīdes operācijas „bitu bīdes loģiskjai ierīcei”
	(sk.~\ref{sec:shifter}.~nod.). Šāda sadalīšana ļauj veikt aritmētiskās
	un bīdes operācijas vienlaicīgi, uz ko balstās \mnem{AR} instrukcija
	(sk.~\ref{sec:AR}.~nod.~\pageref{sec:AR}.~lpp.)
	
	ALU realizēts ar asinhronu darbību, t.i.~izejas vērtība tiek izmainīta
	uzreiz bez kontroles signāla (takts) pievadīšanas.
	
	\begin{singlespace}
		\lstinputlisting[language={[qucs]VHDL},%float=pb,%
		                caption={ALU VHDL apraksts. (\texttt{alu.vhd})},%
		                label=kb:alu]
			{code/alu.vhd}
	\end{singlespace}

\pagebreak[3]
\subsubsection{Bitu bīdes loģiskā ierīce} \label{sec:shifter}
	Tā sauktā bitu bīdes loģiskā ierīce realizē bitu bīdes operācijas kuras
	šī procesora realizācijā ir izdalītas atsevišķi no ALU. Līdzīgi ALU
	realizācijai — darbojas asinhroni.

	\begin{figure}[thb]
		\centering
		%\def\svgwidth{7cm}
		\def\svgscale{1.25}
		{\ttfamily\scriptsize\input{img/sub-shift.pdf_tex}}
		\caption{Bitu bīdes loģiskā ierīce.}
		\label{fig:shift}
	\end{figure}
	
	\begin{singlespace}
		\lstinputlisting[language={[qucs]VHDL},%float=pb,%
		                caption={Bīdes ierīces VHDL apraksts. (\texttt{shifter.vhd})},%
		                label=kb:shifter,%
		                tabsize=3%,% TO REDUCE THE CODE WIDTH
		                ]
			{code/shifter.vhd}
	\end{singlespace}

%\clearpage
\pagebreak[3]
\subsubsection{Komparators} \label{sec:comp}
	Komparators jeb salīdzinātājs ir komponente, kas salīdzina 
	divus operandus un izvada tā vienādības vai nevienādības signālu.
	
	Šī komparatora implementācija ir minimizēta, kombinacionāla un nesatur
	kontroles signālus. Izvadīti tiek tikai divi biti, no kuriem viens
	norāda uz operandu vienādību vai nevienādību, savukārt otrs norāda
	vai operands \texttt{a} ir lielāks par \texttt{b} vai nav. Ar šo
	informāciju ir pilnīgi pietiekami, lai būtu iepējams realizēt visas
	nosacījuma zarošanās (\mnem{BRxx}) instrukcijas.
	\begin{figure}[thp]
		\centering
		%\def\svgwidth{7cm}
		\def\svgscale{1.25}
		{\ttfamily\scriptsize\input{img/sub-comp.pdf_tex}}
		\caption{Komparators.}
		\label{fig:comp}
	\end{figure}
	
	\begin{singlespace}
		\lstinputlisting[language={[qucs]VHDL},%float=pb,%
		                caption={Komparatora VHDL apraksts. (\texttt{comp.vhd})},%
		                label=kb:comp]
			{code/comp.vhd}
	\end{singlespace}

\pagebreak[3]
\subsubsection{Kontroles iekārta}
	Viena no procesora definējošām un iespējams sarežģītākajām komponentēm
	— kontroles iekārta ir tā kas nodrošina mikro-operāciju izpildi.
	Kontroles iekārta realizēta kā stāvokļa mašīna, kur katrs stāvoklis
	atbilst mikro-operācijai, un tātad tās VHDL apraksts ir uzskatāms par
	realizētā procesora mikrokodu.
	Kontroles iekārtas VHDL apraksts ir garš un sarežģīts, un tā daļas tiks
	izskatītas instrukciju darbības aprakstos.\\
	(Pilno kodu sk.~pielikumā~\ref{appx:control} \pageref{appx:control}.~lpp.)
	\begin{figure}[h!]
		\centering
		%\def\svgwidth{7cm}
		\def\svgscale{1.1}
		{\ttfamily\scriptsize\input{img/sub-control.pdf_tex}}
		\caption{Kontroles iekārta.}
		\label{fig:control}
	\end{figure}
