\subsection{Izstrādātie palīglīdzekļi}\label{appx:tools}
\setcounter{lstlisting}{0} %Reset listings counter for the (sub)appendix

Autors darba gaitā ir izstrādājis palīglīdzekļus, kuri
tiešā veidā neattiecas uz darba tēmu, bet sekmējuši darba izstrādi.
Šie palīglīdzekļi ir potenciāli noderīgi izmantošanai citos projektos
neatkarīgi no šī maģistra darba.

\subsubsection*{Piesātinājuma aritmētikas VHDL bibliotēka}
Veselo skaitļu (\termEn{integer}) aritmētikai datorsistēmās
(un citur elektronikā), praktiski vienmēr, ir ierobežots
reprezentējamo skaitļu apgabals. Šādā gadījumā var izdalīt divu veidu
aritmētikas tipus, ko nosaka matemātisko darbību definētie rezultāti, ja
tie nonāk ārpus reprezentējamā apgabala.
Tipiski datorsistēmās tiek izmantota ,,modulo aritmētika'', kad rezultāts
tiek atgriezts reprezentējamā apgabalā ,,apejot riņķi'' nonākot atpakaļ
šajā apgabalā.
Savukārt, ,,Piesātinājuma aritmētika'' ir otrs variants, kad rezultāts, operācijas
rezultātā ieņem galējo vērtību, kāds reprezentējams definētajā apgabalā.

Uzskatāmības pēc, ņemsim piemēru $240+20$ darbībai ar reprezentācijas
apgabalu $[0 ; 256)$. Modulo aritmētikā šīs darbības rezultāts ir:
\[
	(240 + 20) \mod 256 = 260 \mod 256 = 4
\]
savukārt piesātinājuma aritmētikā:
\[
	\max\left( {\min (240+20, 255)}, 0 \right) =
		\max\left({\min(260, 255)}, 0\right) = 255
\]

\begin{singlespace}
\lstinputlisting[language={[qucs]VHDL},%float=hb,%
	                 caption={Piesātinājuma aritmētikas VHDL pakotne.},%
	                 breaklines,breakatwhitespace,%
	                 mathescape=true,%
	                 tabsize=2,%
	                 label=kb:satarith]
		{code/saturation/saturation_arith.vhdl}
\end{singlespace}

Autora izstrādātā pakotne realizē piesātinājuma aritmētikas saskaitīšanas un
atņemšanas operācijas VHDL vidē ar dažāda garuma \texttt{unsigned} tipa
mainīgajiem, pēc līdzības ar IEEE~1076.3 standarta \texttt{numeric\_std}
aritmētiskajiem operatoriem.

Autors šo pakotni darbā izmanto veicot operācijas ar pikseļu intensitātes
vērtībām, kur, piemēram, divu intensitāšu summai pārsniedzot
reprezentējamo apgabalu ir vēlams, lai summas vērtība ieņem lielāko
reprezentējamo, nevis kādu nekorekti zemāku vērtību no apgabala.


\begin{singlespace}
	\lstinputlisting[language={[qucs]VHDL},%float=thb,%
	                 caption={Piesātinājuma aritmētikas VHDL pakotnes implementācija.},%
	                 breaklines,breakatwhitespace,%
	                 mathescape=true,%
	                 tabsize=2,%
	                 label=kb:satarith-body]
		{code/saturation/saturation_arith-body.vhdl}
\end{singlespace}


\subsubsection*{VHDL attēlu ielādes bibliotēka simulācijām}
Autors, veicot ar VHDL aprakstītās aparatūras darbības simulāciju šajā darbā
saskārās ar problēmu attēlu datu nodošanu simulatoram. Problēmas risinājumam
tika izveidota bibliotēka, kas definē \texttt{image\_t} tipu attēlam un
funkcijas attēlu ielādei un pikseļu datu izgūšanai.

Attēlu ielādes funkcijas implementētas C++ valodā piesaistoties CVD 
bibliotēkai. C++ koda piesaistei VHDL izmantota GHDL simulatora specifiska
saskarne.

\pagebreak
\begin{singlespace}
	\lstinputlisting[language={[qucs]VHDL},%float=thb,%
	                 caption={VHDL attēla tipa un funkciju pakotne.},%
	                 breaklines,breakatwhitespace,%
	                 mathescape=true,%
	                 tabsize=2,%
	                 label=kb:imageio]
		{code/imagelib/imageio.vhdl}
	
	\lstinputlisting[language={C++},%float=thb,%
	                 caption={Funkciju implementācija C++ valodā.},%
	                 breaklines,breakatwhitespace,%
	                 firstline=8,firstnumber=8,%
	                 showlines=false,%
	                 inputencoding=iso-8859-13,%
	                 keywords={[2]size,push_back},
	                 keywords={[3]std,vector,string,exception},
	                 %tabsize=2,%
	                 label=kb:imageio-cpp]
		{code/imageio_impl.cpp}
\end{singlespace}

%iso-8859-13
