\subsection{Revīziju vēsture} \label{sec:sys-revs}
%\todo

\begin{description}
	\item[Rev.~01] \hfill \\
		Kā RAM izmantota atmiņas realizācija ar 
		trīs-stāvokļu, divvirzienu datu šinu.
	\item[Rev.~02] \hfill \\
		Līdzīgi revīzijas izmaiņām procesora uzbūvē, arī atmiņa pārveidota,
		likvidējot trīs-stāvokļu šinu un tā vietā divas vienvirziena šinas
		datu ieejai un izejai. Kontroles signāli atstāti bez izmaiņām,
		tādējādi atmiņa vēljoprojām „emulē” trīs-stāvokļu datu apmaiņu,
		saglabājot procesora iespēju strādāt ar divvirzienu šinas atmiņu
		(papildus izmantojot virziena maiņas buferus).
	\item[Rev.~03] \hfill \\
		Šī revīzija veic fundamentālas izmaiņas sistēmas perifērijā,
		pievienojot patstāvīgo atmiņu un realizējot adrešu telpas
		piekārtošanu datu ievades/izvades ierīcēm ar MMU palīdzību.
		Šai arī paredz sistēmas \termEn{boot} procesu, kur izpildes
		programma tiek pārvietota izpildei uz operatīvo atmiņu no
		patstāvīgās atmiņas.
\end{description}
