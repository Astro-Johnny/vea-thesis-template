\subsubsection{Raksturpunktu atlase pēc Harisa mēra}
ORB izmanto FAST9 algoritma noteikto raksturpunktu pēcapstrādi novērtējot tos
pēc Harisa (\termEn{Harris}) stūru mēra un atlasot $N$ skaitu
raksturpunktus ar augstākajiem Harisa mēra rādītājiem~\cite{ORB}.
FAST detektora slieksnis $t$ tiek izvēlēts pietiekami zems, lai
raksturpunktu kopā būtu vismaz $N$ skaits elementu un $t$ tiek samazināts,
ja tā nav~\cite{ORB}.
Šī ir otrreizēja raksturpunktu atlase, jo tiek veikta pēc FAST 
lokālo maksimumu atlases.

%~ Harisa stūra mērs balstās uz starpību kvadrātu summu starp attēla
%~ apgabalu un no tā nobīdītiem apgabaliem, kura aprēķināšanai
%~ Haris\cite{Harris} definē aproksimācijas tenzoru~\cite{FAST}:
%~ \[
	%~ \vb{H} =
	%~ \begin{pmatrix}
		%~ \overline{{I_x'}^2} & \overline{{I_x'}{I_y'}}\\
		%~ \overline{{I_x'}{I_y'}} & \overline{{I_y'}^2}
	%~ \end{pmatrix}
%~ \]
%~ kur $I_x'$ un $I_y'$ ir attēla parciālie atvasinājumi (pēc $x$ un $y$),\\
%~ un
Harisa stūru mērs tiek izmantots Harisa stūru detektorā~\cite{Harris}\cite{FAST},
kuru aprēķina ar:
\begin{equation}
	V_H = |\vb{H}| - k \cdot {(H_{11}+H_{22})}^2
\end{equation}
kur\\
$\vb{H}$ ir Harisa definēts struktūras tenzors\cite{Harris}\cite{FAST} un\\
$k$ ir konstante kas kontrolē stūra vai malas jutību.

Tenzors $\vb{H}$ iekļauj viduvējošanu (summēšanu) noteiktā
logā~\cite{FAST}. Lai gan avots \cite{ORB} nenorāda konkrētos parametrus,
autors analizējot ORB OpenCV implementācijas pirmkodu\cite{OpenCV-src}
secina, ka viduvējošanai izmantots $7 \times 7$ kvadrāta formas logs un
izmantotais $k=0.04$.

Ņemot vērā, ka Harisa stūru mērs $V_H$ tiek noteikts punktiem
$\vec{p} \in \hat{F_9}(\vb{I}, \vec{p}, t)$, kas ir apakškopa no attēla
$\vb{I}$, pretstatā Harisa stūru detektoram, kas $V_H$ nosaka visiem attēla
punktiem~\cite{Harris}, var arī uzskatīt, ka šajā gadījumā FAST funkcionē kā
filtrs Harisa stūru detektoram.
