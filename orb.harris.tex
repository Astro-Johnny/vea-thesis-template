\subsubsection{Raksturpunktu atlase pēc Harisa mēra}
ORB izmanto FAST9 algoritma noteikto raksturpunktu pēcapstrādi novērtējot tos
pēc Harisa (\termEn{Harris}) stūru mēra un atlasot $N$ skaitu
raksturpunktus ar augstākajiem Harisa mēra rādītājiem~\cite{ORB}.
FAST detektora slieksnis $t$ tiek izvēlēts pietiekami zems, lai
raksturpunktu kopā būtu vismaz $N$ skaits elementu un $t$ tiek samazināts,
ja tā nav~\cite{ORB}.
Šī ir otrreizēja raksturpunktu atlase, jo tiek veikta pēc FAST 
lokālo maksimumu atlases.

%~ Harisa stūra mērs balstās uz starpību kvadrātu summu starp attēla
%~ apgabalu un no tā nobīdītiem apgabaliem, kura aprēķināšanai
%~ Haris\cite{Harris} definē aproksimācijas tenzoru~\cite{FAST}:
%~ \[
	%~ \vb{H} =
	%~ \begin{pmatrix}
		%~ \overline{{I_x'}^2} & \overline{{I_x'}{I_y'}}\\
		%~ \overline{{I_x'}{I_y'}} & \overline{{I_y'}^2}
	%~ \end{pmatrix}
%~ \]
%~ kur $I_x'$ un $I_y'$ ir attēla parciālie atvasinājumi (pēc $x$ un $y$),\\
%~ un
Harisa stūru mērs tiek izmantots Harisa stūru detektorā~\cite{Harris}\cite{FAST},
kuru aprēķina ar:
\begin{equation}
	V_H = |\vb{H}| - k \cdot {(H_{11}+H_{22})}^2
\end{equation}
kur\\
$\vb{H}$ ir Harisa matrica\cite{Harris}\cite{FAST}, un\\
%$H_{11}$ un $H_{22}$ ir $\vb{H}$ matricas elementi, un\\
$k$ ir konstante, kas kontrolē stūra vai malas jutību.

Harisa matrica $\vb{H}$, saīsinātā formā un izvērstā formā,
tiek pierakstīta kā~\cite{Harris}\cite{FAST}:
\begin{equation}\label{eq:harris}
	\vb{H} =
		\begin{pmatrix}
			\langle I_x^2 \rangle & \langle I_xI_y \rangle \\
			\langle I_xI_y \rangle & \langle I_y^2 \rangle
		\end{pmatrix}
		=
		\sum_{\vec{r}} W(\vec{r})
			\begin{pmatrix}
				I_x(\vec{p}+\vec{r})^2 & I_x(\vec{p}+\vec{r}) I_y(\vec{p}+\vec{r}) \\
				I_x(\vec{p}+\vec{r}) I_y(\vec{p}+\vec{r}) & I_y(\vec{p}+\vec{r})^2
			\end{pmatrix}
\end{equation}
kur\\
$I_x$ un $I_y$ ir parciālie atvasinājumi attēlam $\vb{I}$,\\
stūra iekavas $\langle {\,} \rangle$ apzīmē summēšanu noteiktā logā
(saīsinātā forma),\\
$W$ ir izvēlētais logs (kā funkcija), un\\
$\vec{r}$ ir loga pārvietojums no centra punkta $\vec{p}$, kurā tiek noteikta
intensitāšu starpība (atvasinājums).

%~ Harisa stūra detektors balstās uz noteikta attēla loga un tā pārvietojumu
%~ pašlīdzības noteikšanu ar starpību kvadrātu summu, ko
%~ aproksimē \eqref{eq:harris} definīcija~\cite{Harris}.

Hariss~un~Stefens\cite{Harris} (\termEn{Harris~and~Stephen})
rekomendē izmantot Gausa matricu par logu
$W$, bet avots~\cite{ORB}, definējot ORB, izmantotos parametrus nenorāda.
Autors analizējot ORB OpenCV implementācijas pirmkodu\cite{OpenCV-src}
secina, ka ORB implementācijā tiek izmantots $7 \times 7$ kvadrāta formas
logs (ar vienmērīgu sadalījumu) un izmantotais $k=0.04$.

Ņemot vērā, ka Harisa stūru mērs $V_H$ tiek noteikts punktiem
$\vec{p} \in \hat{F_9}(\vb{I}, \vec{p}, t)$, kas ir apakškopa no attēla
$\vb{I}$, pretstatā Harisa stūru detektoram, kas $V_H$ nosaka visiem attēla
punktiem~\cite{Harris}, var arī uzskatīt, ka šajā gadījumā FAST funkcionē kā
filtrs Harisa stūru detektoram.
