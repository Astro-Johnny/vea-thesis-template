\subsubsection{oFAST: raksturpunkta virziens} \label{sec:ofast}
oFAST ir ORB algoritma raksturpunktu virziena noteikšanas metode.
Neskatoties uz Rublē~u.c.\cite{ORB} piedēvēto nosaukumu, oFAST tiešā veidā
nemodificē FAST algoritmu un tādēļ autors to uzskata par atsevišķu ORB
algoritma komponenti.

Raksturpunktu virziena informācija ir nepieciešama
ORB rotācijas invariances nodrošināšanai~\cite{ORB}
(sk.~\ref{sec:rbrief-def}~nod.).
Tā noteikšanai izmanto
,,intensitātes centroīdu''\cite{Rosin}\cite{ORB} (jeb ,,masas'' centru, pēc fizikālās
līdzības), kas ir definēta kā:
\begin{equation}
	\vec{C} = \left( \frac{m_{10}}{m_{00}},\; \frac{m_{01}}{m_{00}} \right)
\end{equation}
kur $m_{00}$, $m_{01}$, $m_{10}$ ir attēla momenti, kurus aprēķina pēc:
\begin{equation}
	m_{pq} = \sum_{x,y} x^p y^p I(x,y)
\end{equation}

oFAST virzienu uzdod kā leņķi, kas vienāds ar centroīdas $\vec{C}$
lenķi pret $x$ asi (tās pozitīvo virzienu), ko aprēķina ar~\cite{ORB}:
\begin{equation}
	\theta = \mathrm{atan2}(m_{01}, m_{10})
\end{equation}
kur $\mathrm{atan2}()$ ir divu argumentu arktangenss (kas arī nosaka kvadrantu).

Lai šo virziena mēru padarītu maksimāli rotācijas invariantu,
momentu noteikšanai izvēlas riņķa formas attēla apgabalu ar diametru, kas
vienāds ar salāgotāja (kvadrāta formas) apgabala malas garumu $S$
(definēta~\ref{sec:brief}~nodaļā).
