\section{Mikrokontroliera realizācijas paraugs} \label{sec:uC}
	Iepriekšējā nodaļā apskatītais mikrokontroliera kodols (centrālā
	komponente), bez papildus komponentēm ir praktiski nefunkcionāls, jo
	nav iespējams uzglabāt un pievadīt izpildāmās instrukcijas un veikt
	datu apmaiņu ar perifērām ierīcēm.
	
	%Vitāli svarīga komponente, praktiski jebkurā datoriekārtā, 
	%ir operatīvā atmiņa (RAM), kur
	%glabāt izpildāmo programmu, kā arī pirmsapstrādes, pēcapstrādes un 
	%apstrādes laika datus.
	
	Šī nodaļa apskata kā izveidotais kodols iekļaujas
	mikrokontroliera uzbūvē, piedāvājot	parauga implementāciju
	mikrokontrolierim. 
	Paraug\-implementācija paredzēta mikrokontroliera kodola 
	darbības demontrēšanai, kā arī kalpo par dokumentētu pamatu 
	turmākajām, iespējams, specializētām implementācijām.
	
	\subsection{Implementācijas īpašības} \label{sec:design}
Izstrādājot mikrokontrolieri tika pieņemti vairāki lēmumi par tā uzbūvi,
veidojot specifikāciju kopu uz kuras
balstās izstrādātātās mikrokontroliera komponentes. \todo \\
\begin{figure}[bhp]
	\centering
	\def\svgwidth{\textwidth}
	{\input{img/uC-sheem.pdf_tex}}
	\caption{Augšējā līmeņa blokshēma (rev.~03).}
	\label{fig:top-rev3}
\end{figure}
Šim mikrokontrolierim ir sekojošas uzbūves īpašības:
\begin{description}
	\item[Vienota adrešu telpa] \hfill \\
		Mikrokontroliera kodols komunicē ar ārējām ierīcēm tikai caur
		atmiņas saskarni, bet ne obligāti šai ierīcei arī jābūt atmiņai.
		Tā vietā kopējai adrešu telpai tiek piekārtotas vairākas ierīces
		izdalot noteiktus atmiņas apgabalus. Adrešu dekoderis veido saskarni
		multipleksējot ierīces atkarībā no pieprasītās adreses.
		(sk.~\ref{sec:mmu}~nod.)\pagebreak[1]
	\item[Aparatūras kontrolēta SPI saskarne] \hfill \\
		Atšķirībā no citiem mikrokontrolieriem ar SPI saskarni
		[\todo ], % FIXME: Citation needed
		šis mikro\-kontrolieris nodrošina aparatūras kontrolētu datu pārraidi,
		atšķirībā no programmatūras kontrolētu pārraidi ar
		\termEn{bit-banging} metodi%
		\footnote{Metode kurā takts signāls tiek ģenerēts izpildot programmas
			instrukcijas, pārraidot vienu bitu vienā vai vairākos kodola takts ciklos.}.
		Tādējādi tiek vienkāršots programmkods SPI pārraidēm, un atbrīvots
		kodols citu darbību izpildei.
	\item[] \todo
\end{description}

	%\subsection{Simulētā shēma (rev.~02)}
Sākotnējā darba augšējā līmeņa sistēmas gala shēma paredzēta, kā
paškomplektējoša, tikai ar procesoru un atmiņu, 
kur no ārpuses tiek tikai dota takts (\texttt{CLOCK})
un atiestatīšanas signāls ($\overline{\texttt{RESET}}$). Darbības pārbaudei
tiek izmantotas izstrādes platformas piedāvātās diodes.

\begin{figure}[bhp]
	\centering
	\def\svgwidth{\textwidth}
	{\ttfamily\tiny\input{img/top-rev2.pdf_tex}}
	\caption{Augšējā līmeņa shēma (rev.~02).}
	\label{fig:top-rev2}
\end{figure}

Šī shēma tika veiksmīgi simulēta pirms sintēzes
(rezultātus sk.~\ref{appx:simulation}.~pielikumā),
bet tā nav korekti sintezējama, jo paredz RAM sākotnējos datus,
kurus sintēzes rīks ignorē.
Realitātē RAM dati tiek pazaudēti tikko tiek noņemts barošanas spriegums un
tātad pēc šādas shēmas nav iespējams saglabāt izpildāmo programmu.

\pagebreak[3]
%\subsubsection{Operatīvā atmiņa}
	Operatīvā atmiņa šeit realizēta ar divām vienvirziena datu apmaiņas
	šinām. Kontroles signāli izmantoti līdzīgi klasiskai trīs-stāvokļu
	divvirzienu datu šinas atmiņai, pielāgojoties procesora atmiņas
	saskarnei.
	
	\begin{singlespace}
		\lstinputlisting[language={[qucs]VHDL},%float=p,
		                caption={RAM VHDL entītija.},%
		                label=kb:ram-entity,%
		                linerange={7-13},firstnumber=7,
		                breaklines,breakatwhitespace]
			{code/mem.twoport.vhd}
	
		\lstinputlisting[language={[qucs]VHDL},%float=p,
		                caption={RAM VHDL arhitektūras apraksts (izgriezums).},%
		                label=kb:ram-trimmed,%
		                linerange={85-99},firstnumber=85,%
		                breaklines,breakatwhitespace]
			{code/mem.twoport.vhd}
	\end{singlespace}
	
	\noindent Pilno kodu skatīt \ref{appx:ram-code}.~pielikumā.
 \clearpage %\pagebreak[3]
	\subsection{Papildinātā shēma (rev.~03)}
Šī apakšnodaļa apraksta sistēmas papildinājumus funkcionālas, sintezējamas
sistēmas realizācijai. Šīs revīzijas implementācija \textbf{nav pabeigta},
tādēļ šī apakšnodaļa ir uzskatāma par \textbf{turpmākā darba dokumentāciju},
nevis realizētu implementāciju.

Trešā sistēmas revīzija pievieno patstāvīgās atmiņas elementus sākotnējās
programmas un datu uzglabāšanai. Šīs atmiņas ierīces piekārtotas kopējai
adresācijas telpai, tādējādi rodas nepieciešamība pēc adreses dekodēšanas
loģikas, kuru nodrošina MMU (\termEn{Memory map unit}).



\subsubsection{Boot ROM}
	\termEn{Boot ROM} ir tikai nolasāmā atmiņa, kura paredzēta
	\termEn{bootstrap} procesa programmas uzglabāšanai.
	Šīs programmas uzdevums ir ielādēt reālo izpildes programmu no
	patstāvīgās atmiņas — šajā implementācijā konkrēti 
	no SPI \termEn{Flash} atmiņas.
	
	Par \termEn{Boot ROM} kalpos \termEn{Actel Fusion} piedāvātais
	\termEn{FlashROM} makross.\citeet{FlashROM}
	Maksimāli pieejamais ROM garums uz izmantojamās platformas ir
	1 kilobits jeb 64 vārdi.\citeet{FusionGuide}

\subsubsection{Adrešu telpai piekārtotā ievade/izvade}
	Papildus adresācijas telpā tiks piekārtoti speciālie reģistri dažādu
	perifēro ierīču datu apmaiņai un konfigurācijai.
	
	Uz doto brīdi (un kā redzams blokshēmā — \ref{fig:top-rev3}.~att.)
	paredzēta tikai SPI saskarne. Tam paredzēts viens pārraides un viens
	konfigurācijas reģistrs. Tam paredzēta iespēja nokonfigurēt datu
	apmaiņu pa baitam (8 biti) vai pa vārdam (16 biti). Datu pārraide tiek
	uzsākta uzreiz pēc jauno datu ierakstes reģistrā (ja SPI iespējots) un
	tiek pārraidīti bez procesora līdzdalības.
	
	\begin{figure}[thp]
		\centering
		%\def\svgwidth{7cm}
		\def\svgscale{1.25}
		{\ttfamily\scriptsize\input{img/sub-spi.pdf_tex}}
		\caption{SPI saskarnes ierīce.}
		\label{fig:spi}
	\end{figure}
	
	SPI saskarne nepieciešama komunikācijai ar SPI \termEn{Flash} atmiņu,
	kura pieejama uz izstrādes platformas,\cite[43.~lpp.]{FusionGuide}
	kur tiks glabāta galvenā izpildāmā programma, kura \termEn{boot} procesa
	laikā tiks pārvietota uz RAM.

\subsubsection{MMU — adrešu dekoderis} \label{sec:mmu}
	MMU jeb adrešu dekoderis ir ierīce, kas atbild par adrešu telpas
	pārdalīšanu dažādām ierīcēm, kā arī nodrošināt korektu komunikāciju
	ar šīm ierīcēm.
	
	\begin{figure}[thp]
		\centering
		\def\svgwidth{0.75\textwidth}
		{\ttfamily\small\input{img/remap.pdf_tex}}
		\caption{Adrešu telpas sadalījums.}
		\label{fig:memory-map}
	\end{figure}
	
	MMU tādējādi ar procesoru veido tādu pašu saskarni kā RAM un aparatūras
	līmenī ir pilnībā „caursīdīga”, savukārt piekārtojamo ierīču saskarne
	ir implementācijas definēta.
	
	\termEn{Boot ROM} nepieciešams piekārtot ar sākumu adresē \texttt{0x0000},
	lai pēc atiestatīšanas procesors sāktu izpildīt \termEn{boot} programmu.
 \pagebreak[3]
	
	% TODO: Revision history in appendices
	%\subsection{Revīziju vēsture} \label{sec:sys-revs}
%\todo

\begin{description}
	\item[Rev.~01] \hfill \\
		Kā RAM izmantota atmiņas realizācija ar 
		trīs-stāvokļu, divvirzienu datu šinu.
	\item[Rev.~02] \hfill \\
		Līdzīgi revīzijas izmaiņām procesora uzbūvē, arī atmiņa pārveidota,
		likvidējot trīs-stāvokļu šinu un tā vietā divas vienvirziena šinas
		datu ieejai un izejai. Kontroles signāli atstāti bez izmaiņām,
		tādējādi atmiņa vēljoprojām „emulē” trīs-stāvokļu datu apmaiņu,
		saglabājot procesora iespēju strādāt ar divvirzienu šinas atmiņu
		(papildus izmantojot virziena maiņas buferus).
	\item[Rev.~03] \hfill \\
		Šī revīzija veic fundamentālas izmaiņas sistēmas perifērijā,
		pievienojot patstāvīgo atmiņu un realizējot adrešu telpas
		piekārtošanu datu ievades/izvades ierīcēm ar MMU palīdzību.
		Šai arī paredz sistēmas \termEn{boot} procesu, kur izpildes
		programma tiek pārvietota izpildei uz operatīvo atmiņu no
		patstāvīgās atmiņas.
\end{description}

