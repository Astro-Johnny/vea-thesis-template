\subsection{Revīziju vēsture} \label{sec:cpu-revs}
%\todo

\begin{description}
	\item[Rev.~01] \hfill \\
		Par pamatu ņemta Perija grāmatas\citeet{Perry-VHDL} procesora
		realizācija. Atsevišķas komponentes izveidotas ļoti līdzīgi, bet
		kontroles iekārtas modelis un tam piekārtotā instrukciju tabula
		pilnībā izstrādāta no jauna.
	\item[Rev.~02] \hfill \\
		\termEn{Actel Fusion} FPGA nepiedāvā trīs-stāvokļu loģiku.\citeet{FusionFAQ}
		Tā rezultātā
		datu apmaiņas šina, pārveidota no kopējas arbitējamas 
		\termEn{multidrop} realizācijas pārveidota uz sazarotu vienvirziena
		pārraides ķēdi. Šāds solis pavildus ļāva arī likvidēt trīs-stāvokļu
		buferus un samazināt kontroles iekārtas signālu skaitu.
	\item[Rev.~03] \hfill \\
		Šī revīzija nav ienesusi fundamentālu izmaiņu procesora uzbūvē.
		(Izmaiņas sistēmas perifērijā skatīt
			\ref{sec:sys-revs}.~nod.~\pageref{sec:sys-revs}.~lpp.)
\end{description}
