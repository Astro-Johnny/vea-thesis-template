\subsection{Heterogēnas sistēmas} \label{sec:heterogenous}
Par heterogēnu sistēmu sauc sistēmu, uz kuras ir pieejamas vairākas
skaitļošanas ierīces. Vistipiskākā heterogēnā sistēma uz CPU pamata ar
GPGPU spējīgu video adapteri (ar GPU), kam atbilst vairums datoru, kas
ražoti kopš 2010.~gada.
GPU šajā kontekstā vienmēr var uzskatīt par ,,palīgprocesoru'',
jo šī platforma nevar funkcionēt autonomi, pretstatā CPU un FPGA.

Retāk kombinē CPU un FPGA vienā sistēmā, kas vairumā gadījumu realizēta,
datoram pieslēdzot perifērijas karti ar FPGA pie kādas no datora
pamatplates perifērijas saskarnēm, visbiežāk --- pie PCI Express
saskarnes. OpenCL ir arī radījis pirmo standartizēto programmēšanas
saskarni CPU un FPGA kombinācijai. %TODO: Citation needed!

\begin{figure}[tbh]
	\centering
	\def\svgscale{1.4}
	{\input{img/full-hetero-system.pdf_tex}}
	\caption{,,Pilnas'' heterogēnas sistēmas modelis.}
	\label{fig:hetero-sys}
\end{figure}

Protams, ir iespējama arī visu trīs platformu kombinācija vienā sistēmā, kā
shematiski ilustrēts \ref{fig:hetero-sys}~attēlā. Arī CPU procesors var
būt (un bieži ir) vairāku kodolu procesors, veidojot augstas paralelitātes
sistēmu ar ļoti augstu potenciālo skaitļošanas jaudu.

Programmējot heterogēnas sistēmas ir iespējams dalīt algoritmu daļās
izpildei dažādās platformās. Tomēr jāņem vērā, ka datu pārraide starp
platformām ir būtisks ātrdarbības ierobežojums, jo starpplatformu 
datu pārraides ātrums, praktiski vienmēr, ir ievērojami zemāks nekā
platformu skaitļošanas ātrums.


% TODO?: Nodaļa vai apakšnodaļa par OpenCL?
