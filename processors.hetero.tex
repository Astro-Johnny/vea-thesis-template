\subsection{Heterogēnas sistēmas} \label{sec:heterogenous}
Par heterogēnu sistēmu sauc sistēmu, uz kuras ir pieejamas vairākas
skaitļošanas ierīces. Vistipiskākā heterogēnā sistēma uz CPU pamata ar
GPGPU spējīgu video adapteri (ar GPU), kam atbilst vairums datoru, kas
ražoti kopš 2010.~gada.
GPU šajā kontekstā vienmēr var uzskatīt par ,,palīgprocesoru'',
jo šī platforma nevar funkcionēt autonomi, pretstatā CPU un FPGA.

Retāk kombinē CPU un FPGA vienā sistēmā, kas vairumā gadījumu realizēta,
datoram pieslēdzot perifērijas karti ar FPGA pie kādas no datora
pamatplates perifērijas saskarnēm, visbiežāk --- pie PCI Express
saskarnes.

Protams, ir iespējama arī visu trīs platformu kombinācija vienā sistēmā, kā
shematiski ilustrēts \ref{fig:hetero-sys}~attēlā. Arī CPU procesors var
būt (un bieži ir) vairāku kodolu procesors, kopumā veidojot augstas paralelitātes
sistēmu ar ļoti augstu potenciālo skaitļošanas jaudu.

\begin{figure}[tbh]
	\centering
	\def\svgscale{1.3}
	{\small\input{img/full-hetero-system.pdf_tex}}
	\caption{,,Pilnas'' heterogēnas sistēmas modelis.}
	\label{fig:hetero-sys}
\end{figure}

Programmējot heterogēnas sistēmas ir iespējams dalīt algoritmu daļās
izpildei dažādās platformās. Tomēr jāņem vērā, ka datu pārraide starp
platformām ir būtisks ātrdarbības ierobežojums, jo starpplatformu 
datu pārraides ātrums, praktiski vienmēr, ir ievērojami zemāks nekā
platformu skaitļošanas ātrums.

%\subsubsection{OpenCL}
Heterogēnu sistēmu programmēšanai ir pieejama OpenCL programmēšanas vide.
OpenCL definē uz C programmēšanas valodas bāzētu valodu (OpenCL C), kurā
apraksta apakšprogrammas izpildei heterogēnās sistēmās~\cite{OpenCL-book}. 

OpenCL apakšprogrammas tiek kompilētas izpildes laikā, izmantojot mērķa
platformas ražotāja rīku kopu. Šādas rīku kopas ir pieejamas gan CPU, gan GPU,
gan FPGA platformām~\cite{OpenCL-book}.

