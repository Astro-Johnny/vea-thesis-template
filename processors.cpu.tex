\subsection{Centrālais procesors} \label{sec:cpu}
Centrālais procesors jeb CPU (no angļu \termEn{central processing unit}), ir
praktiski visās datorsistēmās --- gan personālajos datoros (PC), gan serveros,
gan iekļautajās sistēmās. CPU kalpo par galveno vadības komponenti un, vairumā
gadījumu, arī kā galvenais skaitļošanas resurss.

CPU arhitektūra ir vēsturiski attīstīta kopš 20.~gadsimta vidus~%
\cite{Flynn-arch}\cite{von-Neumann}.
Klasiskas arhitektūras CPU raksturo ātra secīgu instrukciju izpilde, 
bet kam nepiemīt ne datu, ne uzdevumu paralelitāte~\cite{Owens-GPU}.
Šādi CPU aritmētiskās instrukcijas izpilda ar vienu datu vienību --- 
operandu vai operandu pāri (piem.,~divu skaitļu saskaitīšanu)~\cite{Flynn-arch},
kā tas vienkāršoti parādīts \ref{fig:cpu-arch}~attēlā.
Līdz 21.~gadsimtam veiktspējas palielināšanai pamatā bija 
instrukcijas izpildes laika saīsināšana vienkārši 
palielinot takts signāla frekvenci
un/vai optimizējot izpildes signāltraktu (\termEn{pipeline})~\cite{Flynn-arch}.
20.~gadsimta 90-o~gadu beigās plaši pieejamo CPU arhitektūrā tika ieviestas
vektoru jeb SIMD (\termEn{single instruction, multiple data}) 
instrukcijas, kas ieviesa zināmu datu paralelitāti jo ar šo instrukciju
palīdzību, varēja izdarīt darbības ar vairāk datu vienībām (skaitļu vektoru)
vienlaikus~\cite{SIMD}.
Savukārt, sākoties 21.~gadsimtam, \newTerm{vairāku kodolu} procesori
sāka kļūt komerciāli pieejami, kas nodrošināja arī uzdevumu paralelitāti.

\begin{figure}[tbh]
	\centering
	\def\svgscale{1.2}
	{\input{img/CPU-arch.pdf_tex}}
	\caption{Skaitļošanas resursi CPU arhitektūrā.}
	\label{fig:cpu-arch}
\end{figure}

\phantomsection\label{sec:cache}
Palielinot CPU instrukciju izpildes ātrumu par
,,vājāko ķēdes posmu'' kļuva datu atgūšana no operatīvās atmiņas (RAM) jeb
tās latentums.
Šo problēmu jau 60-ajos gados risināja radot
\termTech{kešatmiņu} (angļu \termEn{cache})~\cite[473.~lpp.]{Patterson},
kuras pamatprincips ir mazākas ietilpības, bet zemāka latentuma (ātrākas)
atmiņas izmantošana, lai uzglabātu datu apakškopas kopiju no
RAM ar kuru, potenciāli, tūlītēji tiks veiktas darbības.
\cite{Flynn-arch}\cite{Patterson2}\cite{Patterson}\cite{Cache}

\termTech{Kešatmiņa} būtiski uzlabo CPU aprīkotas sistēmas vispārējo
ātrdarbību, bet tās trūkumi sistēmas paralelitātē kļuva acīmredzami
izstrādājot vairāku kodolu CPU~\cite{Fatahalian}\cite{Owens-GPU}\cite{Cache}.
Vairāku procesoru vai vairāku procesora kodolu%
\footnote{Turpmāk nodaļas tekstā minēti kā atsevišķi procesori.}
sistēmas nodrošina
\termTech{kešatmiņas} \newTerm{koherenci}
(angļu \termEn{cache coherence}), t.i.,~visām vienas datu vienības
kopijām \termTech{kešatmiņā(s)} pēc izmaiņas vienādi jāatspoguļo tās
jaunākā vērtība, kā arī jānodrošina, ka šo vērtību vienlaicīgi 
izmainīt drīkst tikai viens no procesoriem.
\begin{figure}[tbh]
	\centering
	\def\svgscale{1.2}
	{\input{img/snoop-cache-bottleneck.pdf_tex}}
	\caption{\termTech{Kešatmiņas} koherence vairāku procesoru sistēmā.}
	\label{fig:snoop-bottleneck}
\end{figure}
Šādā sistēmā, koherences nodrošināšanai, nepieciešama papildus loģika.
Pie tam, tipiskā implementācijā, šī komunikācija
jānodrošina procesoram ,,katram ar katru''.
Tas nozīmē, ka pie $N$ skaita
procesoru nepieciešama komunikācija $\frac{N^2-N}{2}$ skaitam procesoru pāru
(sk.~\ref{fig:snoop-bottleneck}~att.), tādējādi koherences nodrošināšana ir
kavējošais faktors liela skaita procesoru sistēmās.
Šādu koherences nodrošināšanas protokola modeli dēvē ,,okšķērējošo''
(\termEn{snooping}) modeli, kur izmaiņas par datu vienības vērtības izmaiņu
tiek apraidītas visu procesoru \termTech{kešatmiņas}.
\cite{Cache}

Eksistē arī alternatīvs modelis --- direktorija (\termEn{directory}) modelis,
kur centrāli tiek \termTech{kešatmiņas} datu vienībām piedēvēts ,,īpašnieks''
un komunikācija notiek tikai starp šo ,,īpašnieku'' un procesoru, kurš
pieprasa pieeju datu vienībai. Šādi tiek samazināts komunikācijas apjoms pie
liela procesoru skaita, bet modelis ir komplicētāks, kas prasa papildus
atmiņu direktorijam un atmiņas transakcijas izpildes laiks ir garāks.
\cite{Cache}

Savukārt, būtiski citādāka pieeja ir izmantota grafiskā procesora (GPU)
uzbūvē. GPU definē citādu atmiņas izmantošanas
modeli, kas padara \termTech{kešatmiņas} koherenci mazsvarīgu,
atbrīvojot arhitektūras uzbūvi no koherences nodrošināšanas loģikas un
sekmējot paralelitāti. GPU arhitektūra plašāk apskatīta \ref{sec:gpu}~nodaļā.




