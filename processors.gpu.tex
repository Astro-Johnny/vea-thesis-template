\subsection{Grafiskais procesors} \label{sec:gpu}
Grafiskais procesors jeb GPU (no angļu \termEn{graphics processing unit})
ir specializēta skaitļošanas iekārta, kura izstrādāta
un attīstīta divdimensiju un trīsdimensiju attēlu atveidošanai un apstrādei
to izvadei uz displeja.
Grafiskos procesorus izvieto:
\begin{itemize}
	\item uz \newTerm{video kartēm} --- kopā ar tam speciāli paredzētu
		atmiņu (VRAM) --- kuras var pieslēgt PC \newTerm{mātes platei};
	\item tieši uz mātes plates, kur GPU var izmantot speciāli paredzētu
		atmiņu un/vai koplietot (ar CPU) datora operatīvo atmiņu (RAM);
	\item iestrādājot vienā mikroshēmā ar CPU (tipiski jaunos klēpjdatoros);
	\item arvien biežāk, iekļautajās sistēmās iestrādājot SoC
		(angļu \termEn{system-on-chip}) mikroshēmās.
\end{itemize}

\begin{figure}[tbh]
	\centering
	\def\svgscale{1.2}
	{\input{img/GPU-arch.pdf_tex}}
	\caption{Skaitļošanas resursi GPU arhitektūrā.}
	\label{fig:gpu-arch}
\end{figure}
