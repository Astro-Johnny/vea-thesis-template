\subsection{Grafiskais procesors} \label{sec:gpu}
Grafiskais procesors jeb GPU (no angļu \termEn{graphics processing unit})
ir specializēta skaitļošanas iekārta, kura izstrādāta
un attīstīta divdimensiju un trīsdimensiju attēlu atveidošanai un apstrādei
to izvadei uz displeja.
Grafiskos procesorus izvieto:
\begin{itemize}
	\item uz \newTerm{video kartēm} --- kopā ar tam speciāli paredzētu
		atmiņu (VRAM) --- kuras var pieslēgt PC \newTerm{mātes platei};
	\item tieši uz mātes plates, kur GPU var izmantot speciāli paredzētu
		atmiņu un/vai koplietot (ar CPU) datora operatīvo atmiņu (RAM);
	\item iestrādājot vienā mikroshēmā ar CPU (tipiski jaunos klēpjdatoros);
	\item arvien biežāk, iekļautajās sistēmās iestrādājot SoC
		(angļu \termEn{system-on-chip}) mikroshēmās.
\end{itemize}

Lai gan GPU idejiski nav izstrādāts, lai veiktu vispārējus skaitļošanas
uzdevumus, GPU arhitektūras attīstības tendences pavēra šādu iespēju un GPU
kļuva nozīmīga skaitļošanas platforma augstās veiktspējas dēļ, ko,
galvenokārt, nodrošina GPU arhitektūras izteiktā paralelitāte.

Sākotnēji GPU arhitektūras pamatā bija vairāku pakāpju signāltrakts, kur
katra pakāpe veica fiksētu funkciju ar lielu apjomu datu. Katra pakāpe
signāltraktā varēja darboties vienlaicīgi, tādējādi GPU arhitektūrai
piemita gan uzdevumu, gan datu paralelitāte no tās pirmsākumiem.
Programmējamība GPU arhitektūrā parādījās programmējamas \newTerm{ēnotāju}
(\termEn{pixel shader} un \termEn{vertex shader}) pakāpes,
kuras iepriekš arī bija fiksētas funkcijas. Šādam signāltraktam bija būtiska
problēma ar slodzes sadalīšanu, jo slodze dažādās pakāpēs bija atkarīga
no datiem un ēnotāju programmējuma. Šo problēmu risināja izstrādājot
,,vienotu ēnotāju arhitektūru'' (\termEn{unified shader architecture}),
kuras pamatā ir liels skaits programmējami, paralēli
\newTerm{straumes procesori} (\termEn{stream processors}), kuru lomu
signāltraktā, kurš tagad vairs nav fiksēts aparatūras līmeni, var mainīt.
GPU straumes procesori izmanto SIMD instrukcijas darbībām ar skaitļu
vektoriem (sk.~\ref{fig:gpu-arch}~att.), pie tam katrs no šiem straumes procesoriem
arī izmanto \newTerm{vairākpavedienošanu} (\termEn{multithreading}) aparatūras
līmenī. Ņemot vērā lielo skaitu%
\footnote{AMD Radeon HD7990 ir 4096 straumes procesori.
	\url{http://www.amd.com/en-us/products/graphics/desktop/7000/7990}}
šādu SIMD straumes procesoru, GPU var attīstīt ļoti lielu datu caurlaidspēju
(\termEn{throughput}).
\cite{Fatahalian}\cite{Owens-GPU}

\begin{figure}[tbh]
	\centering
	\def\svgscale{1.2}
	{\input{img/GPU-arch.pdf_tex}}
	\caption{Skaitļošanas resursi GPU arhitektūrā.}
	\label{fig:gpu-arch}
\end{figure}

GPU atmiņas izmantošanas modeli arī definē tā pamatuzdevums --- atveidot jeb
\termTech{rasterizēt} attēlus vadoties pēc trīsdimensiju objektu datiem.
Šie dati signāltraktā tiek transformēti un apstrādāti attēla iegūšanai,
bet ieejas dati \termTech{rasterizējot} netiek modificēti
un iegūtais attēls (vairumā gadījumu) pēc tā izvades uz displeja neietekmē
nākamo attēlu, respektīvi, esošie, koplietojamie dati netiek pārrakstīti,
bet tiek radīti jauni dati no tiem. Tas atbrīvo GPU no atmiņas koherences
problēmas, kāda ir CPU arhitektūrā (sk.~\pageref{sec:cache}~lpp.).

Vēl viens būtisks paralelitātes aspekts atmiņas izmantošanas modelī ir
uzsvars uz datu caurlaidspēju, nevis zemu atmiņas latentumu.
,,\termTech{Kešatmiņas}'' loma GPU arhitektūrā arī novirzīt slodzi no
pamatatmiņas, kas, galvenokārt, ir tikai lasāmā
(\termEn{read-only}) \termTech{kešatmiņa}%
\footnote{GPU arhitektūrā speciālām datu grupām izmanto arī
	rakstāmu (\termEn{read/write}) \termTech{kešatmiņu}~\cite{Owens-GPU}.}
\cite{Fatahalian}. Augsto atmiņas latentumu SIMD straumes procesori kompensē
ar vairākpavedienošanu. Katrs SIMD straumes procesors uzglabā informāciju
par vairākiem%
\footnote{NVIDIA GeForce GTX 280 atbalsta līdz 128 pavedieniem uz katru
	straumes procesoru \cite{Fatahalian}.}
izpildes pavedieniem (\termEn{threads}), kur aktīvais pavediens, kas izdara
pieprasījumu no atmiņas un ir spiests gaidīt, tiek aizvietots ar
citu pavedienu, kurš ir tūlītēji izpildāms~\cite{Fatahalian}. Tādējādi tiek samazināts
gaidīšanas laiks un tiek efektīvāk izmantoti skaitļošanas resursi.


