\documentclass[xetex]{beamer} % beamer = presentation type
\usepackage{fontspec}		% The XeTeX font spec package
\usepackage{xunicode}		% XeTeX Unicode character support
\usepackage{xltxtra}		% Umm something else XeTeX
\usepackage{url}

\usepackage{booktabs}	% Package for nicer tables

\usepackage{polyglossia}	% XeTeX multi-lingual support
\setmainlanguage{latvian}
\setotherlanguages{english,russian}

\newcommand{\termEn}[1]{\textenglish{\itshape {#1}}} % inline English

%\newfontfamily\japfont{IPAPMincho}
%\newfontfamily\japfont{IPAPGothic}
%\setmainfont{DejaVu Sans}
%\setsansfont{DejaVu Sans}
%\newfontfamily\IPA[Scale=MatchUppercase]{DejaVu Sans}


\author{Jānis Šmēdiņš}
\title[Mikrokontroliera kodola izstrāde]
	{Iekļautās sistēmas mikrokontroliera kodola izstrāde}
%\titlegraphic{\includegraphics[height=0.5\textheight]{img/Thistle.pdf}}
%\date{November 30, 2011}

%\usetheme{Dresden}	% My usual
\usetheme{Warsaw}	% Similar to Dresden

%\usecolortheme{whale}	% Usually same as default

%\usefonttheme{structuresmallcapsserif}	% Use small caps for titles etc.

% External common image library
\graphicspath{{/home/johnlm/Development/resources/tex_graphics_lib/}{img/}}

% symbol dump : “ ” —

\begin{document}
	% Slide 1 : Titlepage
	\begin{frame}
		%\centering\includegraphics[height=0.4\textheight]{Thistle}\\
		\titlepage 	% The default generated title slide
	\end{frame}
	
	\begin{frame}{Paraugierīces uzbūve}
		~ % FIX: To avoid putting graphic into frame title
		{\def\svgwidth{\textwidth} \small \input{img/uC-sheem.pdf_tex}}
	\end{frame}
	
	\begin{frame}{Darba izstrāde}
		\begin{itemize}
			\item Komponentes tiek aprakstītas VHDL valodā.
			\item Pirms sintēzes tiek veikta ierīces simulācija
				darbības korektuma novērtēšanai.
			\item Ierīce tiek realizēta sintezējot VHDL aprakstus
				FPGA platformai.
		\end{itemize}
	\end{frame}
	
	\begin{frame}{Progress}
		\large Izdarītais:
		\begin{itemize}\small
			\item Kodola komponenšu VHDL apraksti;
			\item Kodola darbības simulācija;
			\item Vairums perifēro komponenšu VHDL apraksti;
			\item \termEn{Boot} processa programma.
		\end{itemize}
		\large Darāmais:
		\begin{itemize}\small
			\item SPI un GPIO saskarnes VHDL apraksts;
			\item Fiziskā realizācija uz FPGA testa platformas.
		\end{itemize}
	\end{frame}
	
	% Slide Last : Thank you!
	\begin{frame}
		\begin{center}
			\Large Paldies par uzmanību!\\[2em]
			%\normalsize It's now time for your questions.
		\end{center}
	\end{frame}
	
\end{document}
% OLDDDDDDDDDDDDDDDDDDDDDDDDDDDDDDDDDDDDDDDDDDDDDDDDDDDDDDDDDDDDDDDDDDDDDDD
	
	
	
	\begin{frame}{Languages}
		\begin{columns}
		\column{0.25\textwidth}
			{\def\svgwidth{\textwidth}
			 \input{img/languages.pdf_tex}}
		\column{0.7\textwidth}
				\begin{itemize}
					\item \textbf{English} is the \textit{de facto}
						official language.\\[1em]
					\item (Lowland) Scots is often considered a language in
						it's own right, but it's distinction from English
						is rather fuzzy.\\[1em]
					\item \textbf{Scottish Gaelic} (or Highland Scots) is
						a recognized language used by 1\% of the population.
				\end{itemize}
		\end{columns}
	\end{frame}
	
	%\section{Culture}
	\begin{frame}{Vocabulary of Scottish English}
		Scottish English has retained some archaic features of English
		and it has numerous loanwords from Gaelic.
		
		\begin{table} \begin{tabular}{lcll}
			\toprule
			\textbf{Scottish E.} & \textbf{IPA} 
				& \textbf{English} & \textbf{Usage}\\
			\midrule
			\textbf{aye} & \IPA{[aɪ]} & yes & {\footnotesize Widespread}\\
			\textbf{nae} & \IPA{[neː]} & no & {\footnotesize Uncommon}\\
			\midrule
			\textbf{loch} & \IPA{[lɔx]} & lake & {\footnotesize Widespread}\\
			\midrule
			\textbf{bonnie} & \IPA{[bɔnɪ]} & beautiful, graceful
				& {\footnotesize Fairly common}\\
			\textbf{wee} & \IPA{[wiː]} & little, small
				& {\footnotesize Common}\\
			\textbf{wabbit} & \IPA{[ˈwæbɪt]} & exhausted
				& {\footnotesize Fairly common}\\
			\bottomrule
		\end{tabular}\end{table}
		
		\textbf{NOTE}: The Scottish vocabulary is highly region dependent.
	\end{frame}
	
	\begin{frame}{Scottish Culture “in a nutshell”}
		\begin{columns}
			\column{0.4\textwidth}
				\includegraphics[width=\textwidth]{ConneryKilt-edit}
			\column{0.58\textwidth}
				\small
				Scottish have strong sense of nationality. The turbulent
				history with English has played major role in Scottish
				culture.
				\begin{itemize} %\small
					\item Tartan Kilt is the traditional garment of Scots.
					\item Highland Pipe music is associatted with Scots.
					\item Football (Soccer) is hugely popular.
					\item There is animosity towards English.
				\end{itemize}
		\end{columns}
	\end{frame}
	
	\begin{frame}{Etiquette}
		\begin{itemize}
			\item Most common informal greeting is “\textbf{Hello}” or
				“\textbf{Hiya}”, \linebreak[2] while “How do you do”
				would serve better in formal cases.
			\item \textbf{Men shake hands} upon meeting. In other cases
				handshake may be exchanged with woman extending her hand
				first.
			\item \textbf{Politeness} is highly valued.
			\item \textbf{Punctuality} is expected.
		\end{itemize}
	\end{frame}
	
	% TABOOS/DONTS HERE????????
	
	\begin{frame}{Scottish Cuisine}
		\begin{columns}
			\column{0.5\textwidth}
				\begin{block}{Haggis, neeps and tatties}
					\includegraphics[width=\textwidth]{Haggis-neeps-and-tatties}
				\end{block}
			\column{0.5\textwidth}
				\begin{block}{Scottish breakfast}
					\includegraphics[width=\textwidth]{Scottish-breakfast-1024px}
				\end{block}
		\end{columns}
		\begin{itemize}\small
			\item Scottish foods are rich in fats and protein.
			\item Fast food has become popular in late
				20\textsuperscript{th} century.
		\end{itemize}
	\end{frame}
	
	\begin{frame}{Festivals and Holidays}
		\begin{columns}
			\column{0.3\textwidth}
				\includegraphics[width=\textwidth]{EdinburghNYE-800px}
			\column{0.7\textwidth}
				\begin{itemize}
					\item 30\textsuperscript{th} Nov — 
						\textbf{St.~Andrews Day}
					\item 25\textsuperscript{th} Jan — 
						\textbf{Burns' Night}
					\item 31\textsuperscript{st} Oct — 
						\textbf{All Hallows' Eve} (Haloween)\\[2ex]
					\item There are a large number regional
						\textbf{Scottish Highland Game and Dance festivals}
					\item Most ubiqutous western holidays are celebrated as well
						\begin{itemize}
							\item New Year
							\item Christmas
							\item Easter
							\item Labour Day
							\item etc.
						\end{itemize}
				\end{itemize}
		\end{columns}
	\end{frame}
	
	%\begin{frame}{Caveats}
	%	\begin{itemize}
	%		\item Scotland is quite cosmopolitan country.
	%	\end{itemize}
	%\end{frame}
	
	\begin{frame}{References}
		\begin{itemize} \small
			\item \texttt{culturecrossing.net}, \textit{Scotland}\\
				{\scriptsize \url{http://www.culturecrossing.net/basics_business_student.php?id=242}}
			\item eHow, \textit{Business Etiquette in Scotland}\\
				{\scriptsize \url{http://www.ehow.com/about_5286041_business-etiquette-scotland.html}}
			\item Wikimedia foundation, \textit{Scotland}
				{\scriptsize \url{http://en.wikipedia.org/wiki/Scotland}}
			\item Wikimedia foundation, \textit{Scottish English}
				{\scriptsize \url{http://en.wiktionary.org/wiki/Category:Scottish_English}}
		\end{itemize}
	\end{frame}
	
	
	
	
	
	% Slide last-1 : References
	\begin{frame}{References}
		\begin{itemize}
			\normalsize
			\item {\japfont 山崎照彦} (2005),
				  {\small{\japfont\itshape カラーTFT液晶ディスプレイ}\\
				  (translit. Yamasaki Teruhiko (2005), \textit{Colour TFT LCD})}
			\item Toshiba corp.~(2000), \textit{LTM10C210 Product Information}
			\item Wikimedia foundation, \textit{Liquid crystal display}
				  from \url{http://en.wikipedia.org/wiki/LCD}
			\item Wikimedia foundation, \textit{Thin-film transistor}
				  from \url{http://en.wikipedia.org/wiki/Thin-film_transistor}
		\end{itemize}
	\end{frame}
