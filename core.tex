\section{Mikrokontroliera kodols} \label{sec:cpu}
	Mikrokontroliera kodols, pēc savas būtības un uzbūves, ir procesors,
	kurš veic kalkulācijas un datu apmaiņu, nodrošinot mikrokontroliera
	programmas izpildi.
	
	Šajā darbā izstrādātais mikrokontroliera kodols ir 
	Fon-Neimaņa (\termEn{von~Neumann}) arhitektūras pro\-ce\-sors, kurš
	izmanto 16 bitu instrukcijas%
		\footnote{Instrukcijas var būt $N$-skaita 16 bitu vārdi. 
		Šajā darbā realizētas viena un divu vārdu instrukcijas.},
	kā arī ,,atomiski'' adresējamā atmiņas vienība, kurai pieder unikāla
	adrese, ir 16 bitu plata. Kā Fon-Neimaņa procesoram, datu un
	progrmmkoda atmiņa ir kopēja, un vienādi adresējama.\citeet{von-Neumann}
	
	Kodola uzbūve ir daļēji balstīta uz D.~Perija (\termEn{D.~Perry}) grāmatas%
	\citeet{Perry-VHDL} piedāvāto procesora reali\-zā\-ciju, 
	bet ar ievērojamām izmaiņām, galvenokārt pilnībā no jauna izstrādāta 
	instrukciju kopa un vienvirziena,
	cirkulāra datu apmaiņas šina atšķirībā no kopējas,
	trīs\-stāvokļu arbitējamas datu šinas. Galvenais šāda risinājuma iemesls
	ir izmantojamās \termEn{Actel} FPGA darba platformas trīs\-stāvokļu
	loģikas atbalsta \mbox{trūkums.\cite[18.~lpp.]{FusionFAQ}}
	
	Instruckiju kopa ir izstrādāta no jauna un piedāvā vienkāršotu,
	mini\-mālu kopu līdzīgi RISC (\termEn{Reduced Instruction Set Computing})
	filo\-so\-fijai. Salīdzinot ar Perija realizāciju, ievērojami samazinātas
	nosacījuma zarošanās instrukcijas un instrukcijas pieejamie 16 biti
	tiek efektīvāk izmantoti ieviešot vektorizētus jeb dažāda garuma
	operāciju kodus. (detalizētākam aprakstam sk.~\ref{sec:instrSet}~nod.)
	%\todo
	
	
	
	%Tā kā \termEn{Actel} FPGA risinājums neatbalsta trīs stāvokļu
	%loģiku\cite[18.~lpp.]{FusionFAQ},
	%tad kopēju iekšējo datu šinu nav iespējams
	%izveidot. Tā vietā izveidota vienvirziena, nosacīti cirkulāra datu
	%apmaiņas līnija. Datu līnijas sazarojumi dažādu komponenšu 
	%komunikācijai realizēti ar multipleksoru palīdzību.
	
	%\begin{figure}[bhp]
	%	\centering
	%	\def\svgwidth{\textwidth}
	%	{\ttfamily\scriptsize\input{img/control.pdf_tex}}
	%	\caption{Procesora kontrole un atmiņas saskarne.}
	%	\label{fig:controlPipeline}
	%\end{figure}
	%
	%\begin{figure}[thp]
	%	\centering
	%	\def\svgwidth{\textwidth}
	%	{\ttfamily\footnotesize\input{img/aluPipeline.pdf_tex}}
	%	\caption{Aritmētikas un loģikas signālceļš un uzbūve.}
	%	\label{fig:aluPipeline}
	%\end{figure}
	
	\begin{figure}[thb]
		\centering
		\def\svgwidth{\textwidth}
		{\ttfamily\footnotesize\input{img/top-rev3-detail.pdf_tex}}
		\caption{Procesora uzbūve.}
		\label{fig:cpu-rev3}
	\end{figure}
	
	%\pagebreak[4]
	%\clearpage
	% TODO: Atmiņas izdalījums, 'von Neumann bottleneck' un kā DMA palīdz
	\subsection{Iekšējās datu šinas uzbūve} \label{sec:databus}
Izveidotajam procesoram ir sazarota, cirkulāra datu apmaiņas šina atšķirībā no kopējas,
trīsstāvokļu arbitējamas datu šinas. Šādai šinas uzbūvei ir vairākas
priekšrocības:
\begin{description}
	\item[Nav trīsstāvokļu loģika] \hfill \\
		Trīsstāvokļu loģikas trūkums atbrīvo no nepieciešamības izmantot
		trīsstāvokļu komponentes procesora iekšējā uzbūvē,
		samazinot realizācijas kompleksitāti.
	\item[Elektriskā drošība] \hfill \\ %TODO: Šitam citādāku nosaukumu?
		Nevienai no komponentēm pie šinas nevar tikt savienotas izejas,
		likvidējot iespējamo kaitējumu arbitācijas kļūdu rezultātā.
	\item[Mazāk vadības signālu] \hfill \\
		Šādai implementācijai nav nepieciešami nolases un rakstīšanas
		vadības signālu pāri katrai pievienotai ierīcei. Vienvirziena datu
		apmaiņa ļauj tikai pievadīt rakstīšanas vadības signālu%
		\footnote{Lai uzsvērtu šo vadības signālu nozīmi reģistru satura
			atjaunošanā, to nosaukumiem pievienots -\texttt{Updt} piedēklis.}%
		, vai pat nevienu signālu kombinacionālo komponenšu gadījumā.
\end{description}

Viens no šādas šinas uzbūves trūkumiem var tikt uzskatīta nepieciešamība
multipleksēt signālus, bet pēc visu iespējamo datu apmaiņas scenāriju analīzes,
implementējamās instrukciju kopas ietvaros, tika noskaidrots ka --- 
pat ne tuvu --- katrai komponentei nepieciešama datu apmaiņa ar visām pārējām.
Konkrēti, tikai piecām komponentēm nepieciešama datu nolase no vairāk
nekā vienas citas.

%\todo % TODO: Tabula ar nepieciešamajiem savienojumiem
\begin{table}[hb]
	\centering
	\caption{Komponentes kurām nepieciešama ieejas multipleksēšana.}
	\label{tbl:muxes}
	\begin{tabular}{ll}
		\toprule
		Komponente & Ienākošie (multipleksējamie) dati\\ 
		\midrule
		Programmskaitītājs & Atmiņas ielase; ALU signālceļa dati\\
		Reģistru masīvs & Atmiņas ielase; ALU signālceļa dati\\
		ALU (pirmais operands) & Programmskaitītājs; Reģistru masīvs\\
		Komparators (pirmais operands) & Programmskaitītājs; Reģistru masīvs\\
		Adresācijas reģistrs & ALU signālceļa dati; Reģistru masīvs\\
		\bottomrule
	\end{tabular}
\end{table}

Apvienojot vienādos multipleksorus iegūstam optimizētu datu apmaiņas šinu
ar tikai trijiem \texttt{32>16} multipleksoriem. (sk.~\ref{fig:cpu-rev3}~att.)

%Galvenais šāda risinājuma iemesls
%ir izmantojamās \termEn{Actel} FPGA darba platformas trīs stāvokļu
%loģikas atbalsta \mbox{trūkums.\cite[18.~lpp.]{FusionFAQ}}
 \clearpage 
	\subsection{Procesora VHDL bibliotēka}
\todo
%\begin{singlespace}
	\lstinputlisting[language={[qucs]VHDL},float=h,%
					 caption={Procesora \texttt{cpu\_lib} pakas definīcija. (\texttt{cpu\_lib.vhd})},%
					 label=kb:cpulib,
					 basicstyle=\ttfamily\scriptsize]
		{code/cpu_lib.vhd}
%\end{singlespace}
	\clearpage
	\subsection{Komponentes}
Izstrādātais procesors (mikrokontroliera kodols) sastāv no vairākām
apakš\-kom\-po\-nentēm (sk.~\ref{fig:cpu-rev3}~att., \pageref{fig:cpu-rev3}~lpp.). 
Šo komponenšu uzbūve un darbība aprakstīta sintēzei
derīgā ,,RTL stila'' VHDL apraksta kodā izmantojot tikai standarta un
paša izveidotas pakotnes,
šādi panākot augstu implementācijas portabilitāti%
	\footnote{Iespējamu projekta izmantošanu dažādos izstrādes rīkos.}.
Lai izvairītos no ,,skriešanās problēmām'' (\termEn{race condition})
simulācijas laikā un arī simulētu signālu aizkavēšanos,
VHDL aprakstos visai bieži izmantota novēlotās piešķiršanas konstrukcija,
kuru sintēzes rīks ignorē.

Komponenšu VHDL aprakstos piesaukti tipi un konstantes no paša definētas
\texttt{cpu\_lib} pakotnes, kas redzama sekojošā koda blokā.
\begin{singlespace}
\lstinputlisting[language={[qucs]VHDL},%float=b,%
                 caption={Procesora \texttt{cpu\_lib} pakas definīcija (\texttt{cpu\_lib.vhd}).},%
                 label=kb:cpulib,basicstyle=\ttfamily\scriptsize]
	{code/cpu_lib.vhd}
\end{singlespace}
\pagebreak[3]

\FloatBarrier
\subsubsection{Reģistri}
	Viena no vienkāršākajām komponentēm ir reģistrs.
	Tā funkcija ir uzglabāt	viena mašīn\-vārdu datus 
	(šajā gadījumā 16 bitu vārda) un pārrakstīt to pēc
	pieprasījuma. Implementētais reģistrs ir sinhrons un dinamisks, un tā
	dati tiek pārrakstīti \texttt{clk} signālam pārejot no zema uz augstu
	stāvokli.
	\begin{figure}[bh]
		\centering
		%\def\svgwidth{7cm}
		\def\svgscale{1.25}
		{\ttfamily\scriptsize\input{img/sub-reg.pdf_tex}}
		\caption{Reģistrs.}
		\label{fig:reg}
	\end{figure}
	
	Reģistra VHDL apraksts ir visai triviāls un, ar nelielām izmaiņām
	(sk.~\ref{kb:reg}~pirmkodu),
	izmantots D.~Perija paraugs.\cite[321.~lpp.]{Perry-VHDL}
	%\begin{singlespace}
	% NOTE: No Need for singlespace when floated!
		\lstinputlisting[language={[qucs]VHDL},float=thb,%
		                caption={Reģistra VHDL apraksts. (\texttt{reg.vhd})},%
		                label=kb:reg]
			{code/reg.vhd}
	%\end{singlespace}
	
	\pagebreak[2]
	Lai gan reģistrs ir vienkārša ierīce, to plaši izmanto procesora
	realizācijā, uzglabājot ļoti dažādas nozīmes datus.
	Šī darba ietvaros izstrādātajā procesorā ir sekojoši speciālās nozīmes reģistri:
	\pagebreak[1]
	\begin{itemize}
		\item \textbf{Programmskaitītājs} (\texttt{PC}):
			Reģistrs, kas uzglabā adresi aktīvajai programmas pozīcijai,
			t.i.~parasti izpildāmās instrukcijas adresi. Secīgās programmas
			izpildes laikā \texttt{PC} tiek inkrementēts%
			\footnote{Konkrētāk — \texttt{PC} tiek palielināts par
				izpildīto instrukcijas vārdu skaitu (1 vai 2).},
			savukārt lēcieni (zarošanās)
			programmā realizēti pārrakstot programmskaitītāja saturu.\pagebreak[2]
		\item \textbf{Izpildāmās instrukcijas reģistrs} (\texttt{instrReg})
			ieraksta izpildāmās instrukcijas kodu.
			Nepieciešams, lai kontroles stāvokļa mašīna „neaizmirstu”
			izpildāmo operāciju un tās argumentus izpildot
			turpmākās mikro\-operācijas%
			\footnote{Mikrooperācijas ir soļi kas jāveic vienas instrukcijas
				izpildei. Kontroles iekārtas stāvokļi atbilst mikrooperācijām
				(sk.~kodu pielikumā \ref{appx:control}).}.
			Divu vārdu instrukcijām saglabāts
			tiek tikai pirmais vārds, jo tikai tajā ir dekodējamais
			operācijas kods. Otrais vārds satur tikai argumentus.
		\item \textbf{Atmiņas adresācijas reģistrs} (\texttt{addrReg})
			uzglabā adresējamo atmiņas adresi tās nolasei
			vai ierakstei. Nepieciešams, lai saglabātu adresi izpildot
			\mnem{LD} un \mnem{ST} instrukcijas (sk.~\ref{sec:instrSet}~nod.),
			%jo tikai viens no diviem iesaistītajiem operatīvajiem reģistriem
			jo tikai viens no diviem iesaistītajiem vispārējas nozīmes reģistriem
			(sk.~\ref{sec:regArray}~nod.) ir nolasāms vai pārrakstāms
			vienā mikrooperācijas solī.
		\item \textbf{Operanda reģistrs} (\texttt{opReg})
			uzglabā vienu no aritmētiskās/loģiskās darbības
			ope\-ran\-diem. Nepieciešams, jo ALU neuzglabā pievadītos
			operandus\footnote{ALU ir realizēta kā kombinacionāla shēma.}
			un abi operandi nolasāmi no tā paša datu avota.
			(par ALU sk.~\ref{sec:alu}~nod., \pageref{sec:alu}~lpp.)
		\item \textbf{Rezultāta reģistrs} (\texttt{outReg}),
			kurā ieraksta aritmētiskās/loģiskās darbības rezultātu.
			Tā kā ALU datu avots var būt (un parasti ir) arī rezultāta
			ierakstes mērķis, rezultāta reģistrs novērš 
			,,skriešanās problēmu'', kas būtu
			iespējama vienlaicīgi lasot un pārrakstot datu avotu.
	\end{itemize}

\pagebreak[3]
\FloatBarrier
%\subsubsection{Operatīvo reģistru masīvs}
\subsubsection{Vispārējas nozīmes reģistru masīvs} \label{sec:regArray}
	Vispārējas nozīmes reģistri ir reģistru kopa, kura, atšķirībā no
	iepriekš apskatītajiem procesora specializētajiem reģistriem,
	ir pieejama	programmētājam tiešai modifikācijai.
	Šiem reģistriem nav noteikta specializēta nozīme,
	tā vietā programmētājs tos lieto pēc vajadzības, piem.~izpildāmās
	programmas mainīgo uzglabāšanai.
	\begin{figure}[th]
		\centering
		%\def\svgwidth{7cm}
		\def\svgscale{1.25}
		{\ttfamily\scriptsize\input{img/sub-regArray.pdf_tex}}
		\caption{Reģistru masīvs.}
		\label{fig:regArray}
	\end{figure}
	
	Vispārējas nozīmes reģistru masīvs ir izpild\-programmas datu apstrādes
	krustpunkts. Tas ir tādēļ ka izstrādātais procesors, 
	līdzīgi vairumam RISC tipa arhi\-tek\-tūru,
	izmanto \termEn{Load/Store} principu, t.i.~visas aritmētiskās darbības
	tiek veiktas ar vispārējas nozīmes reģistriem un datu apmaiņa ar atmiņu (RAM)
	notiek tikai caur šiem reģistriem izmantojot datu apmaiņas
	instrukcijas\cite[11.~lpp.]{Flynn-arch}, šajā gadījumā \mnem{LD} un \mnem{ST}.
	
	%\begin{singlespace}
		\lstinputlisting[language={[qucs]VHDL},float=thb,%
		                caption={Reģistru masīva VHDL apraksts. (\texttt{regarray2.vhd})},%
		                label=kb:regArray,%
		                emph={t_ram}]
			{code/regarray2.vhd}
	%\end{singlespace}
	%\pagebreak[2]
	Realizācijā, kura redzama \ref{kb:regArray} pirmkoda blokā, 
	reģistru masīvs vairāk līdzinās 8 vārdu operatīvajai
	atmiņai, bet netiek realizēta rakstīšanas vai lasīšanas režīma
	pārslēgšana. Tā vietā ir ieejas un izejas pieslēgvietas un
	ierakstītie dati tiek saglabāti un uzreiz izlikti uz izvadi.
	
	Reģistru skaitu masīvā ierobežo pieejamais bitu skaits operācijas
	koda vārdā reģistra adreses\footnote{Vispārējas nozīmes reģistriem ir
		sava, neatkarīga adrešu telpa.}
 	uzglabāšanai. Konkrēti \mnem{AR} instrukcijai pieejami 3 biti reģistra
 	adresei, kas ierobežo vispārējās nozīmes reģistru skaitu līdz 8
 	(unikālo adrešu skaits 3 bitos).

\pagebreak[3]
\FloatBarrier
\subsubsection{Multipleksors}
	Multipleksors praktiski ir elektroniski kontrolēts ,,slēdzis'',
	kas pārslēdz divas vai vairāk ieejas uz vienu izeju. Ieejas un izejas
	ne obligāti ir viens signālvads (bits). Šajā gadījumā tiek pārslēgtas
	16 bitu platas datu maģistrāles, kur
	multipleksors tiek izmantots datu šinā, noslēdzot sazarojumu,
	lai pārslēgtu datu ieejas avotu aiz tā sekojošai komponentei
	(vai komponentēm).
	\begin{figure}[bhp]
		\centering
		%\def\svgwidth{7cm}
		\def\svgscale{1.25}
		{\ttfamily\scriptsize\input{img/sub-mux.pdf_tex}}
		\caption{2 ieeju multipleksors.}
		\label{fig:mux2}
	\end{figure}
	
	Multipleksors realizēts ar ļoti vienkāršu izejas vārda nosacījuma
	pārslēgšanu (sk.~\ref{kb:mux}~pirmkodu).
	%\begin{singlespace}
		\lstinputlisting[language={[qucs]VHDL},float=thb,%
		                 caption={Multipleksora VHDL apraksts (\texttt{mux.vhd}).},%
		                 label=kb:mux]
			{code/mux.vhd}
	%\end{singlespace}

\pagebreak[3]
\FloatBarrier
\subsubsection{Aritmētiski loģiskā ierīce} \label{sec:alu}
	Aritmētiski loģiskā ierīce jeb ALU ir viena no galvenajām
	procesora komponentēm. Tās uzdevums ir veikt doto datu apstrādi un,
	šī procesora realizācijā, arī
	palielināt programmskaitītāju secīgas programmas izpildes nodrošināšanai.
	\begin{figure}[thp]
		\centering
		%\def\svgwidth{7cm}
		\def\svgscale{1.25}
		{\ttfamily\scriptsize\input{img/sub-alu.pdf_tex}}
		\caption{Aritmētiski loģiskā ierīce.}
		\label{fig:alu}
	\end{figure}
	
	Lai gan klasiski bīdes operācijas arī tiek realizētas ALU,
	šajā gadījumā ALU veic tikai aritmētiskās un bitu
	loģikas darbības, atstājot bīdes operācijas „Bitu bīdes loģiskjai ierīcei”
	(sk.~\ref{sec:shifter}~nod.). Šāda sadalīšana ļauj veikt aritmētiskās
	un bīdes operācijas vienlaicīgi, uz ko balstās \mnem{AR} instrukcija
	(sk.~\ref{sec:AR}~nod.~\pageref{sec:AR}~lpp.).
	
	ALU realizēta kā kombinacionāla shēma ar asinhronu darbību,
	t.i.~izejas vērtība tiek izmainīta uzreiz,
	bez kontroles signāla (takts) pievadīšanas, un izejas vērtība ir tikai
	atkarīga no pievadīto operandu vērtībām, kuru vērtības iekšēji ALU
	netiek saglabātas.
	Lai būtu iespējams veikt darbības ar diviem
	operandiem no operatīvo reģistru masīva, tiek izmantots \texttt{opReg}
	reģistrs viena operanda uzglabāšanai, lai otru varētu nodot tieši.
	(sk.~\ref{fig:cpu-rev3}~att., \pageref{fig:cpu-rev3}~lpp.)
	
	\begin{singlespace}
		\lstinputlisting[language={[qucs]VHDL},%float=th,%
		                caption={ALU VHDL apraksts. (\texttt{alu.vhd})},%
		                label=kb:alu]
			{code/alu.vhd}
	\end{singlespace}
	
	ALU veic operācijas ar 16 bitu vārdiem, kuri reprezentē bezzīmes
	skaitļus. ALU VHDL implementācija izmanto \texttt{std\_logic\_unsigned}
	pakotnē	definētās aritmētiskās operācijas,
	ievērojami vienkāršojot ALU VHDL aprakstu (sk.~\ref{kb:alu}~pirmkodu).

\pagebreak[3]
\FloatBarrier
\subsubsection{Bitu bīdes loģiskā ierīce} \label{sec:shifter}
	Ierīce, kura šeit nosaukta par ,,Bitu bīdes loģisko ierīci'',
	realizē bitu bīdes operācijas, kuras šī procesora realizācijā ir
	izdalītas atsevišķi no ALU. 

	\begin{figure}[h!]
		\centering
		%\def\svgwidth{7cm}
		\def\svgscale{1.25}
		{\ttfamily\scriptsize\input{img/sub-shift.pdf_tex}}
		\caption{Bitu bīdes loģiskā ierīce.}
		\label{fig:shift}
	\end{figure}
	
	Realizācija, līdzīgi ALU, ir kombinacionāla un asinhrona. Ieejošie dati
	tiek pārbīdīti atkarībā pēc |sel| signāla stāvokļa 
	(sk.~\ref{kb:shifter}~pirmkodu).
	
	%\begin{singlespace}
		\lstinputlisting[language={[qucs]VHDL},float=th,%
		                caption={Bīdes ierīces VHDL apraksts. (\texttt{shifter.vhd})},%
		                label=kb:shifter]
			{code/shifter.vhd}
	%\end{singlespace}
	
	

%\clearpage
\pagebreak[3]
\FloatBarrier
\subsubsection{Komparators} \label{sec:comp}
	Komparators jeb salīdzinātājs ir komponente, kas salīdzina 
	divus operandus --- 16 bitu vārdus, kas tiek interpretēti kā bezzīmes
	skaitļi.
	Šī komparatora implementācija ir minimizēta, kombinacionāla un nesatur
	kontroles signālus (salīdzinājumam ar D.~Perry realizāciju sk.~\todo{}~nod.).
	Izvadīti tiek tikai divi loģiskie signāli, no kuriem |eq|
	norāda uz operandu vienādību (loģiskais |1|) vai nevienādību (loģiskā |0|),
	savukārt |gr| norāda vai operands \texttt{a} ir lielāks (|1|)
	par \texttt{b} vai nav (|0|). Ar šo
	informāciju ir pilnīgi pietiekami, lai būtu iepējams realizēt visas
	nosacījuma zarošanās (\mnem{BRxx}) instrukcijas
	(sk.~\ref{sec:branching}~nod.~\pageref{sec:branching}~lpp.).
	\begin{figure}[bhp]
		\centering
		%\def\svgwidth{7cm}
		\def\svgscale{1.25}
		{\ttfamily\scriptsize\input{img/sub-comp.pdf_tex}}
		\caption{Komparators.}
		\label{fig:comp}
	\end{figure}
	
	%\begin{singlespace}
		\lstinputlisting[language={[qucs]VHDL},float=thbp,%
		                %basicstyle=\ttfamily\scriptsize,%
		                caption={Komparatora VHDL apraksts. (\texttt{comp.vhd})},%
		                label=kb:comp]
			{code/comp.vhd}
	%\end{singlespace}

%\clearpage
%\FloatBarrier
\pagebreak[3]
\subsubsection{Kontroles iekārta} \label{sec:control}
	Viena no galvenajām procesora komponentēm ir kontroles iekārta, kas,
	iespējams, ir procesora sarežģītākā komponente. Kontroles iekārta
	nodrošina mikrooperāciju izpildi, kas kontrolē datu plūsmu starp
	procesora iekšējām komponentēm un datu apmaiņu ar RAM.
	\begin{figure}[hp]
		\centering
		%\def\svgwidth{7cm}
		\def\svgscale{1.1}
		{\ttfamily\scriptsize\input{img/sub-control.pdf_tex}}
		\caption{Kontroles iekārta.}
		\label{fig:control}
	\end{figure}
	
	Kontroles iekārta realizēta kā stāvokļa mašīna, kur katrs stāvoklis
	atbilst vienai mikrooperācijai, un, tātad, tās VHDL apraksts var tikt
	uzskatīts par analogu mikrokodam. Viena mikrooperācija tiek
	izpildīta vienā takts ciklā.
	
	Kontroles iekārta instrukcijas izpilda pa vienai, izdarot sekojošas
	soļus katras instrukcijas izpildē:
	\begin{enumerate}
		\item \textbf{instrukcijas ielase} (1 mikroop.) nolasa operācijas kodu no RAM
			un saglabā to turpmākai dekodēšanai;
		\item \textbf{instrukcijas dekodēšana} (1 mikroop.) pēc operācijas
			koda nosaka nepieciešamās mikrooperācijas konkrētai instrukcijas
			izpildei; %\footnote{Konkrēti tikai jānosaka viena nākamā izpildāmā
				%mikrooperācija, jo aiz tās sekojošas mikroop.~izpildās jau
				%noteiktā kārtībā};
		\item \textbf{datu ielase un operācijas izpilde} (0--7 mikroop.)
			veic instrukcijas specifiskās operācijas, kas var iekļaut
			nevienu, vienu vai vairākas no sekojošām operācijām:
			\begin{itemize}
				\item datu ielase no RAM;\footnote{%
					Datu nolase var aizņemt vairākus takts ciklus līdz
					atmiņa ir gatava datu pārraidei (\texttt{ready} tiek pacelts).}
				\item datu ierakste RAM;
				\item aritmētiskā un/vai bīdes operācija;
				\item zarošanās nosacījuma pārbaude;
				\item programmskaitītāja pārrakstīšana zarošanās izpildei;
			\end{itemize}\pagebreak[1]
		\item \textbf{programmskaitītāja inkrementācija} veic sagatavošanos
			nākamās instrukcijas izpildes ciklam.
	\end{enumerate}
	
	Kontroles iekārtas kods ir pārāk apjomīgs un komplicēts, lai to
	apskatītu šeit, tādēļ atsevišķas tā daļas tiks izskatītas turpmākajās
	nodaļās un pilnais kods iekļauts pielikumā~\ref{appx:control}
 \pagebreak[3]
	\subsection{Instrukciju kopa} \label{sec:instrSet}
%Instruckiju kopa ir izstrādāta no jauna un piedāvā vienkāršotu,
%mini\-mālu kopu līdzīgi RISC (\termEn{Reduced Instruction Set Computing})
%filo\-so\-fijai. Salīdzinot ar Perija realizāciju, ievērojami samazinātas
%nosacījuma zarošanās instrukcijas un instrukcijas pieejamie 16 biti
%tiek efektīvāk izmantoti ieviešot vektorizētus jeb dažāda garuma
%operāciju kodus. (detalizētākam aprakstam sk.~\ref{sec:instrSet}~nod.)

Instruckiju kopa ir izstrādāta no jauna un piedāvā vienkāršotu kopu, kas
satur pamata darbības ar bezzīmes veseliem (\texttt{unsigned integer} tipa)
skaitļiem, RAM ielasīšanas un ierakstes instrukcijas, un zarošanās
instrukcijas (sk.~\ref{tbl:instructions}~tabulu).

Instrukcijas operāciju kodi ir dažāda garuma, jeb vektoriski\footnote{%
	Termins aizgūts no programmēšanas paņēmiena, kur viena funkcija
	satur dažādu operāciju kopu (vektoru), un pēc nodotā argumenta tiek
	izpildīta viena operācija (mainot pārējo argumentu nozīmi).},
t.i.~operāciju kods sadalīts ,,grupas'' kodā un ,,vektora'' kodā, kur
grupas kods ir konstanta garuma (2~biti), bet vektora kods ir konstanta
garuma tikai piederošās grupas ietvaros. Strikti ņemot, varētu uzskatīt, ka
vektora kods ir arguments, bet uzskatāms par piederīgu operācijas
kodam, jo abas tā daļas tiek interpretētas dekodēšanas solī.\footnote{%
	Izņemot \mnem{AR} instrukciju (sk.~\ref{sec:AR}~nod.).}

\begin{singlespace}\small
\begin{longtable}[c]{lp{20ex}lp{0.36\textwidth}}
	%\centering
	\caption{Instrukciju tabula.}\label{tbl:instructions}\\
	\toprule
	\textbf{Apz.} & \textbf{Mašīnkods} & \textbf{Argumenti} & \textbf{Operācija} \\
	\midrule \endfirsthead
	\caption[]{\nameref{tbl:instructions}~(turpinājums).}\\
	%\toprule
	\midrule
	\textbf{Apz.} & \textbf{Mašīnkods} & \textbf{Argumenti} & \textbf{Operācija} \\
	\midrule \endhead
	\multicolumn{4}{c}{Atmiņas datu apmaiņas instrukcijas}\\
	\midrule
	\mnem{LD} & 	\instr{01}{00}{}{XXXXXX}{XXX}{XXX}{} & \texttt{rD, rS} &
		\texttt{rD = *(rS)} \newline
		{\footnotesize Ielasa vārdu no RAM reģistrā \texttt{rD}} \\ \midrule
	\mnem{LDI} & 	\instr{01}{01}{}{XXXXXX}{XXX}{}{XXX} \newline
					\instr{}{}{}{}{}{XXXXXXXXXXXXXXXX}{} & \texttt{rD, C} &
		\texttt{rD = C} \newline
		{\footnotesize Ielasa konstanti reģistrā \texttt{rD}} \\ \midrule
	\mnem{ST} & 	\instr{01}{11}{}{XXXXXX}{XXX}{XXX}{} & \texttt{rD, rS} &
		\texttt{*(rD) = rS} \newline
		{\footnotesize Ielasa vārdu no RAM reģistrā \texttt{rD}} \\
	\midrule \pagebreak[3]
	\multicolumn{4}{c}{Aritmētikās, loģikās un bitu bīdes instrukcijas (un \mnem{AR} operācijas saīsnes)}\\
	\midrule
	\mnem{AR} & 	\instr{10}{}{}{}{XXXX}{XXX}{X}\instr{}{}{}{}{XXX}{XXX}{} & \texttt{kA, kS, rD, rS} &
		{\footnotesize Aritmētikas/bīdes instrukcija.} \\ \midrule
	\rule{0pt}{1em}\mnem{ADD} & \instr{10}{0000}{000}{X}{XXX}{XXX}{} & \texttt{rD, rS} &
		\verb|rD = rD + rS| \\ \midrule
	\rule{0pt}{1em}\mnem{SUB} & \instr{10}{0001}{000}{X}{XXX}{XXX}{} & \texttt{rD, rS} &
		\verb|rD = rD - rS| \\ \midrule
	\rule{0pt}{1em}\mnem{INC} & \instr{10}{1000}{000}{X}{XXX}{}{XXX} & \texttt{rD} &
		\verb|rD = rD + 1| \\ \midrule
	\rule{0pt}{1em}\mnem{DEC} & \instr{10}{1001}{000}{X}{XXX}{}{XXX} & \texttt{rD} &
		\verb|rD = rD - 1| \\ \midrule
	\rule{0pt}{1em}\mnem{AND} & \instr{10}{0010}{000}{X}{XXX}{XXX}{} & \texttt{rD, rS} &
		\verb|rD = rD & rS| \\ \midrule
	\rule{0pt}{1em}\mnem{OR} & \instr{10}{0011}{000}{X}{XXX}{XXX}{} & \texttt{rD, rS} &
		\verb+rD = rD | rS+ \\ \midrule
	\rule{0pt}{1em}\mnem{XOR} & \instr{10}{0100}{000}{X}{XXX}{XXX}{} & \texttt{rD, rS} &
		\verb|rD = rD ^ rS| \\ \nopagebreak \midrule
	\rule{0pt}{1em}\mnem{NOT} & \instr{10}{1010}{000}{X}{XXX}{}{XXX} & \texttt{rD} &
		\verb|rD = ~rD| \\ \midrule
	\rule{0pt}{1em}\mnem{CLR} & \instr{10}{1111}{000}{X}{XXX}{}{XXX} & \texttt{rD} &
		\verb|rD = 0| \\ \midrule
	\rule{0pt}{1em}\mnem{MOV} & \instr{10}{0101}{000}{X}{XXX}{XXX}{} & \texttt{rD, rS} &
		\verb|rD = rS| \\ \midrule
	\rule{0pt}{1em}\mnem{LSL} & \instr{10}{0101}{001}{X}{XXX}{XXX}{} & \texttt{rD} &
		\verb|rD = rS * 2| \newline
		{\footnotesize loģiskā kreisā bīde (reizina ar 2)} \\ \midrule
	\rule{0pt}{1em}\mnem{LSR} & \instr{10}{0101}{010}{X}{XXX}{XXX}{} & \texttt{rD} &
		\texttt{rD = rS / 2} \newline
		{\footnotesize loģiskā labā bīde (dala ar 2)} \\ \midrule
	\rule{0pt}{1em}\mnem{ROL} & \instr{10}{0101}{011}{X}{XXX}{XXX}{} & \texttt{rD} &
		{\footnotesize bitu rotācija pa kreisi} \\ \midrule
	\rule{0pt}{1em}\mnem{ROR} & \instr{10}{0101}{100}{X}{XXX}{XXX}{} & \texttt{rD} &
		{\footnotesize bitu rotācija pa labi} \\ \nopagebreak
	\midrule \pagebreak[3]
	\multicolumn{4}{c}{Plūsmas kontroles instrukcijas}\\
	\midrule
	\mnem{NOP} & 	\instr{00}{}{}{00000000000000}{}{}{} & nav &
		{\footnotesize tukša operācija} \\ \midrule
	\mnem{HLT} & 	\instr{11}{0011}{}{XXXX}{}{}{XXXXXX} & nav &
		{\footnotesize apstādina procesora darbību} \\ \midrule
	\mnem{JMP} & 	\instr{11}{0010}{}{XXXX}{}{}{XXXXXX} \newline
					\instr{}{}{}{}{XXXXXXXXXXXXXXXX}{}{} & \texttt{L} &
		\texttt{PC = L} \newline
		{\footnotesize beznosacījuma lēciens.} \\ \midrule
	\mnem{BREQ} & 	\instr{11}{1000}{}{XXXX}{XXX}{XXX}{} \newline
					\instr{}{}{}{}{XXXXXXXXXXXXXXXX}{}{} & \texttt{rD, rS, L} &
		\texttt{if(rD==rS) PC = L} \\ \midrule
	\mnem{BRNQ} & 	\instr{11}{1011}{}{XXXX}{XXX}{XXX}{} \newline
					\instr{}{}{}{}{XXXXXXXXXXXXXXXX}{}{} & \texttt{rD, rS, L} &
		\texttt{if(rD!=rS) PC = L} \\ \midrule
	\mnem{BRGT} & 	\instr{11}{1010}{}{XXXX}{XXX}{XXX}{} \newline
					\instr{}{}{}{}{XXXXXXXXXXXXXXXX}{}{} & \texttt{rD, rS, L} &
		\texttt{if(rD>rS) PC = L} \\ \midrule
	\mnem{BRGE} & 	\instr{11}{1001}{}{XXXX}{XXX}{XXX}{} \newline
					\instr{}{}{}{}{XXXXXXXXXXXXXXXX}{}{} & \texttt{rD, rS, L} &
		\texttt{if(rD>=rS) PC = L} \\ \midrule
	\mnem{BRLT} & 	\multicolumn{2}{c}{subst.~\texttt{\textbf{BRGE} rS, rD, L}} &
		\texttt{if(rD<rS) PC = L}\\ \midrule
	\mnem{BRLE} & 	\multicolumn{2}{c}{subst.~\texttt{\textbf{BRGT} rS, rD, L}} &
		\texttt{if(rD<=rS) PC = L}\\
	\bottomrule
	\caption*{\fboxrule=0.75pt \framebox{\footnotesize
		\begin{tabular}{ll}
			\multicolumn{2}{c}{Mašīnkoda krāsu apzīmējumi} \\
			\textcolor{purple}{\rule[-2pt]{1em}{1em}} Operāciju grupas kods &
			\textcolor{blue}{\rule[-2pt]{1em}{1em}} \textcolor{cyan}{\rule[-2pt]{1em}{1em}}
				Operācijas vektora kods \\[2pt]
			\textcolor{lightgray}{\rule[-2pt]{1em}{1em}} Ignorētie biti &
			\textcolor{OliveGreen}{\rule[-2pt]{1em}{1em}} \textcolor{Green}{\rule[-2pt]{1em}{1em}}
				Argumentu biti \\
		\end{tabular}
		}}
\end{longtable}
\end{singlespace}
\normalsize

%\pagebreak
Instrukcijas
ir sadalītas grupās vai nu pēc nozīmes, vai pēc izpildes līdzības. Šo
instrukciju realizācijas detaļas tiks apskatītas turpmākajās apakšnodaļās.

\subsubsection{\mnem{AR} instrukcija} \label{sec:AR}
	Lai gan \mnem{AR} instrukcijas apzīmējums ir saīsinājums no „aritmētika”
	un tā ir instrukcijas primārā nozīme, \mnem{AR} implementē visas
	aritmētiskās, loģiskās, bīdes operā\-cijas, kā arī reģistru 
	apmaiņas \mnem{MOV} instrukciju (sk.~\ref{tbl:instructions}~tabulu).
	Šāda imple\-men\-tā\-cija izmantota tāpēc, ka visas šīs instrukcijas tiek
	pārvadītas pa to pašu signālceļu. Tādējādi kontroles iekārta tiek
	vienkāršota, jo atsevišķās operācijas nav nepieciešams izšķirt.
	
	Kontroles signāli aritmētiskajai un bīdes ierīcei,
	kā argumenti jeb vektora kodi,
	tiek nodoti tieši no instrukcijas vārda, un kontroles iekārtā netiek
	interpretēti (jeb ir „necaurspīdīgi”). Izņēmums, gan ir vektora koda
	vecākais bits, kurš tiek interpretēts un pie augsta stāvokļa (loģiskā |1|)
	tiek apieta otrā operanda (\texttt{rS}) ielase samazinot nepieciešamo
	takts ciklu skaitu instrukcijas izpildei (sk.~\ref{kb:ARdecode}~koda bloku).
	Šis bits ir |1|	unārājām un bezargumentu instrukcijām
	\mnem{INC}, \mnem{DEC}, \mnem{NOT} un \mnem{CLR}.%
	\footnote{Instrukcijas \mnem{LSL}, \mnem{LSR}, \mnem{ROL} un \mnem{ROR}
		nav īsti unāras, bet asemblējot šīs saīsnes izmanto to argumentu gan kā
		\texttt{rS}, gan kā \texttt{rD}.}
	
	\begin{singlespace}
		\lstinputlisting[language={[qucs]VHDL},%float=pb,%
		                caption={\mnem{AR} instrukcijas dekodēšana (izgriezums).},%
		                label=kb:ARdecode,%
		                firstnumber=150]
			{code/gen/ardecode-snippet.vhd}
	\end{singlespace}
	
	\mnem{AR} instrukcijas saīsnes nenosedz visas iespējamās darbības, kuras
	iespējamas ar \mnem{AR}. Tā kā ALU un Bitu bīdes loģiskā ierīce ir
	atsevišķas komponentes uz viena signālceļa, tās darbības ir izpildāmas
	vienlaicīgi. Tādējādi, izmantojot \mnem{AR} pilno formu, iespējams
	kombinēt aritmētiskās un bīdes operācijas (piem. saskaitīt un veikt bīdi)
	izpildei vienā instrukcijā. Jāņem vērā, ka Bitu bīdes loģiskā ierīce
	atrodas aiz ALU (sk.~\ref{fig:cpu-rev3}~att.),
	tādēļ bitu bīde vienmēr tiek izpildīta aritmētiskās darbības rezultātam.


\subsubsection{Nosacījuma zarošanās instrukcijas} \label{sec:branching}
	Nosacījuma zarošanās ir vitāla īpašība procesoram. Programmēšanas
	valodu \texttt{if()} konstrukcija nav iedomājama bez
	nosacījuma zarošanās.
	
	Izstrādātais procesors par zarošanās nosacījumu var izmantot divu reģistru
	satura apzīmēto bezzīmes veselo skaitļu salīdzinājumu.
	Šai salīdzināšanai atbilst \mnem{BRxx} instrukcijas
	(sk.~\ref{tbl:instructions}~tabulu), un pašu salīdzināšanu veic komparators
	(sk.~\ref{sec:comp}~nod.) kuram tiek pievadīti salīdzināmie operandi.
	
	\pagebreak[2]
	Aparatūras līmenī realizētas četras zarošanās instrukcijas:
	\begin{itemize}
		\item \mnem{BREQ} — zaroties, ja operandi ir vienādi;
		\item \mnem{BRNQ} — zaroties, ja operandi nav vienādi;
		\item \mnem{BRGE} — zaroties, ja pirmais operands ir lielāks
			vai vienāds ar otro;
		\item \mnem{BRGT} — zaroties, ja pirmais operands ir stingri lielāks
			par otro.
	\end{itemize}
	
	Zarošās nosacījumā pārbaudei tiek izmantoti šo instrukciju operāciju
	kodu divi jaunākie biti, kuri turpmāk šajā nodaļā apzīmēti ar
	\texttt{A} (otrs jaunākais bits) un \texttt{B} (jaunākais bits).
	
	Šo četru instrukciju operāciju kodi nav izvēlēti gluži patvaļīgi,
	bet pielāgoti tā, lai to \texttt{A} un \texttt{B} biti,
	kopā ar komparatora signāliem |eq| un |gr|
	(apz.~\texttt{E} un \texttt{G}),
	veidotu pēc iespējas vienkāršāku loģisko izteiksmi,
	kas izsaka zarošanās nosacījuma izpildi (sk.~\ref{fig:branch-karnaugh}~att.).
	
	\begin{figure}[thp]
		\centering
		%\def\svgwidth{7cm}
		\def\svgscale{1.5}
		{\ttfamily\input{img/karnaugh.pdf_tex}}\\
		\(
			f = \mathtt{\overline{A}E + BG + AG + AB\overline{E}}
		\)\\ %[1ex]
		%(apzīmējumu nozīmi sk.~\ref{kb:branchTest}.~kodā)
		\caption{Zarošanās nosacījuma Karno karte un formula.}
		\label{fig:branch-karnaugh}
	\end{figure}
	
	Zarošanās nosacījuma pārbaude implementēta kontroles iekārtā kā
	VHDL funkcija (sk.~\ref{kb:branchTest}~pirmkodu).
	
	\begin{singlespace}
		\lstinputlisting[language={[qucs]VHDL},%float=pb,%
		                caption={Zarošanās nosacījuma pārbaudes funkcija (izgriezums).},%
		                label=kb:branchTest,%
		                linerange={61-69},firstnumber=61,
		                emph={state,branchTest},%
		                breaklines,breakatwhitespace,
		                basicstyle=\ttfamily\scriptsize]
			{code/control2.vhd}
	\end{singlespace}
	
	Atlikušās zarošanās instrukcijas \mnem{BRLT} un \mnem{BRLE} ir
	programmatūras līmeņa substitūcijas, jo to ekvivalentus var iegūt
	izmantojot attiecīgi \mnem{BRGE} un \mnem{BRGT} instrukcijas,
	apmainot salīdzināmos operandus vietām assemblēšanas brīdī.
	
	
 %\clearpage %\pagebreak[3]
	% Man vajag revīziju cēsturi?
	%\subsection{Revīziju vēsture} \label{sec:cpu-revs}
%\todo

\begin{description}
	\item[Rev.~01] \hfill \\
		Par pamatu ņemta Perija grāmatas\citeet{Perry-VHDL} procesora
		realizācija. Atsevišķas komponentes izveidotas ļoti līdzīgi, bet
		kontroles iekārtas modelis un tam piekārtotā instrukciju tabula
		pilnībā izstrādāta no jauna.
	\item[Rev.~02] \hfill \\
		\termEn{Actel Fusion} FPGA nepiedāvā trīs-stāvokļu loģiku.\citeet{FusionFAQ}
		Tā rezultātā
		datu apmaiņas šina, pārveidota no kopējas arbitējamas 
		\termEn{multidrop} realizācijas pārveidota uz sazarotu vienvirziena
		pārraides ķēdi. Šāds solis pavildus ļāva arī likvidēt trīs-stāvokļu
		buferus un samazināt kontroles iekārtas signālu skaitu.
	\item[Rev.~03] \hfill \\
		Šī revīzija nav ienesusi fundamentālu izmaiņu procesora uzbūvē.
		(Izmaiņas sistēmas perifērijā skatīt
			\ref{sec:sys-revs}.~nod.~\pageref{sec:sys-revs}.~lpp.)
\end{description}
 %\pagebreak[3]
