\section{Procesora uzbūve} \label{sec:cpu}
	Izstrādātais procesors ir Fon-Neimaņa (\termEn{von~Neumann})
	arhitektūras pro\-ce\-sors, kas ir
	daļēji balstīts uz D.~Perija (\termEn{D.~Perry}) grāmatas%
	\citeet{Perry-VHDL} piedāvāto reali\-zā\-ciju. Procesors izmanto 16 bitu
	instrukcijas%
	\footnote{Instrukcijas var būt $N$-skaita 16 bitu vārdi. Šajā darbā
		realizētas viena un divu vārdu instrukcijas.},
	adresāciju un atmiņas datu šinas.
	
	Instruckiju kopa ir izstrādāta no jauna un piedāvā vienkāršotu,
	mini\-mālu kopu līdzīgi RISC (\termEn{Reduced Instruction Set Computing})
	filo\-so\-fijai. Salīdzinot ar Perija realizāciju, ievērojami samazinātas
	nosacījuma zarošanās instrukcijas un instrukcijas pieejamie 16 biti
	tiek efektīvāk izmantoti ieviešot vektorizētus jeb dažāda garuma
	operāciju kodus.
	
	Tā kā \termEn{Actel} FPGA risinājums neatbalsta trīs stāvokļu
	loģiku\cite[18.~lpp.]{FusionFAQ},
	tad kopēju iekšējo datu šinu nav iespējams
	izveidot. Tā vietā izveidota vienvirziena, nosacīti cirkulāra datu
	apmaiņas līnija. Datu līnijas sazarojumi dažādu komponenšu 
	komunikācijai realizēti ar multipleksoru palīdzību.
	
	\begin{figure}[bhp]
		\centering
		\def\svgwidth{\textwidth}
		{\ttfamily\scriptsize\input{img/control.pdf_tex}}
		\caption{Procesora kontrole un atmiņas saskarne.}
		\label{fig:controlPipeline}
	\end{figure}
	
	\begin{figure}[thp]
		\centering
		\def\svgwidth{\textwidth}
		{\ttfamily\footnotesize\input{img/aluPipeline.pdf_tex}}
		\caption{Aritmētikas un loģikas signālceļš un uzbūve.}
		\label{fig:aluPipeline}
	\end{figure}
	
	\pagebreak[4]
	%\clearpage
	\subsection{Komponentes}
Kā redzams procesora uzbūves shēmās
(\ref{fig:controlPipeline}.~un \ref{fig:aluPipeline}.~attēls), procesors
sastāv no dažādām apakš-komponentēm. Lai panāktu darba portabilitāti,%
\footnote{Iespējamu projekta izmantošanu dažādos izstrādes rīkos.}
komponentes aprakstītas RTL (\termEn{Register Transfer Level}) stila
VHDL apraksta kodā.

\subsubsection{Reģistri}
	Viena no vienkāršākajām komponentēm ir reģistrs. Tā funkcija ir uzglabāt
	viena vārda datus — šajā gadījumā 16 bitu vārda — un atjaunot to pēc
	pieprasījuma (ar \texttt{clk} signālu).
	
	\begin{figure}[bh]
		\centering
		%\def\svgwidth{7cm}
		\def\svgscale{1.25}
		{\ttfamily\scriptsize\input{img/sub-reg.pdf_tex}}
		\caption{Reģistrs.}
		\label{fig:reg}
	\end{figure}
	
	\noindent Procesorā ir vairāki atsevišķi reģistri ar savu nozīmi:
	\begin{description}
		\item[\texttt{PC} — Programmskaitītājs] \hfill \\
			Reģistrs, kas uzglabā adresi aktīvajai programmas pozīcijai,
			t.i.~parasti izpildāmās instrukcijas adresi. Secīgās programmas
			izpildes laikā \texttt{PC} tiek palielināts par 1%
			\footnote{Konkrētāk — \texttt{PC} tiek palielināts par
				izpildīto instrukcijas vārdu skaitu. (1 vai 2)},
			savukārt lēcieni (zarošanās)
			programmā realizēti pārrakstot programmskaitītāja saturu.
		\item[\texttt{instrReg} — Instrukciju reģistrs] \hfill \\
			Reģistrs, kurā ieraksta izpildāmo instrukcijas kodu.
			Nepieciešams, lai kontrole stāvokļa mašīna „neaizmirstu”
			izpildāmo operāciju un tās argumentus izpildot mikrokoda soļus.
		\item[\texttt{addrReg} — Atmiņas adresācijas reģistrs] \hfill \\
			Reģistrs, kurā ieraksta adresējamo atmiņas adresi tās nolasei
			vai ierakstei. Nepieciešams lai izvairītos no
			cirkulārās loģikas ielasot datus no atmiņas.
		\item[\texttt{opReg} — Operanda reģistrs] \hfill \\
			Reģistrs, kas uzglabā vienu no aritmētiskās/loģiskās darbības
			ope\-ran-diem, kamēr tiek iegūts otrs operands.
		\item[\texttt{outReg} — Izejas reģistrs] \hfill \\
			Reģistrs, kurā ieraksta aritmētiskās/loģiskās darbības rezultātu.
			Nepieciešams lai izvairītos no cirkulārās loģikas
			ierakstot rezultātu operatīvajos reģistros.
	\end{description}
	
	\begin{singlespace}
		\lstinputlisting[language={[qucs]VHDL},%float=pb,%
		                caption={Reģistra VHDL apraksts. (\texttt{reg.vhd})},%
		                label=kb:reg]
			{code/reg.vhd}
	\end{singlespace}

\pagebreak[3]
\subsubsection{Operatīvo reģistru masīvs}
	Operatīvie reģistri ir adresējama reģistru kopa, kura, atšķirībā no
	iepriekš apskatītajiem procesora specializētajiem reģistriem,
	ir pieejama	programmētājam tiešai modifikācijai.\\
	Tos sauc arī par vispārēja pielietojuma reģistriem, jo tiem nav noteikta specializēta
	nozīme, tā vietā programmētājs tos lieto pēc vajadzības, piem.~mainīgo
	uzglabāšanai.
	
	\begin{figure}[thp]
		\centering
		%\def\svgwidth{7cm}
		\def\svgscale{1.25}
		{\ttfamily\scriptsize\input{img/sub-regArray.pdf_tex}}
		\caption{Reģistru masīvs.}
		\label{fig:regArray}
	\end{figure}
	\begin{singlespace}
		\lstinputlisting[language={[qucs]VHDL},%float=pb,%
		                caption={Reģistru masīva VHDL apraksts. (\texttt{regarray2.vhd})},%
		                label=kb:regArray,%
		                emph={t_ram}]
			{code/regarray2.vhd}
	\end{singlespace}
	
	\pagebreak[2]
	Šajā realizācijā, reģistru masīvs vairāk līdzinās 8 vārdu operatīvajai
	atmiņai, bet netiek realizēta rakstīšanas vai lasīšanas režīma
	pārslēgšana. Tā vietā ir ieejas un izejas pieslēgvietas, kur
	ierakstītie dati tiek saglabāti un uzreiz izlikti uz izvadi.

\subsubsection{Multipleksors}
	Komponentēm, kurām datu šinā nepieciešams saņemt datus no dažādiem
	devējiem tiek multipleksētas, t.i.~devēju ieejas signāls tiek pārslēgts
	pēc nepieciešamības.
	
	Šim nolūkam izmantots multipleksors. Atšķirībā no citām procesora
	komponentēm multipleksors izveidots ar \termEn{Actel Libero}
	programmatūras pieejamo multipleksora makrosu, kura instancējamā
	entītijas definīcija parādīta \ref{kb:mux}.~koda blokā.
	
	\begin{figure}[thp]
		\centering
		%\def\svgwidth{7cm}
		\def\svgscale{1.25}
		{\ttfamily\scriptsize\input{img/sub-mux.pdf_tex}}
		\caption{2 ieeju multipleksors.}
		\label{fig:mux2}
	\end{figure}
	
	\begin{singlespace}
		\lstinputlisting[language={[qucs]VHDL},%float=pb,%
		                 caption={Multipleksora makrosa entītijas definīcija.},%
		                 label=kb:mux,%
		                 linerange={8-12},firstnumber=8]
			{code/gen/muxiitis.vhd}
	\end{singlespace}

\pagebreak[3]
\subsubsection{Aritmētiski loģiskā ierīce}
	Aritmētiski loģiskā ierīce jeb ALU ir viena no galvenajām,
	procesora definējošām
	komponentēm. Tās uzdevums ir veikt doto datu apstrādi, kā arī
	palielināt programmskaitītāju secīgas programmas izpildes nodrošināšanai.
	
	\begin{figure}[thp]
		\centering
		%\def\svgwidth{7cm}
		\def\svgscale{1.25}
		{\ttfamily\scriptsize\input{img/sub-alu.pdf_tex}}
		\caption{Aritmētiski loģiskā ierīce.}
		\label{fig:alu}
	\end{figure}
	
	Lai gan klasiski bīdes operācijas arī tiek realizētas Aritmētiski
	loģiskajā ierīcē, šajā gadījumā ALU veic tikai aritmētiskās un bitu
	loģikas darbības, atstājot bīdes operācijas „bitu bīdes loģiskjai ierīcei”
	(sk.~\ref{sec:shifter}.~nod.). Šāda sadalīšana ļauj veikt aritmētiskās
	un bīdes operācijas vienlaicīgi, uz ko balstās \mnem{AR} instrukcija
	(sk.~\ref{sec:AR}.~nod.~\pageref{sec:AR}.~lpp.)
	
	ALU realizēts ar asinhronu darbību, t.i.~izejas vērtība tiek izmainīta
	uzreiz bez kontroles signāla (takts) pievadīšanas.
	
	\begin{singlespace}
		\lstinputlisting[language={[qucs]VHDL},%float=pb,%
		                caption={ALU VHDL apraksts. (\texttt{alu.vhd})},%
		                label=kb:alu]
			{code/alu.vhd}
	\end{singlespace}

\pagebreak[3]
\subsubsection{Bitu bīdes loģiskā ierīce} \label{sec:shifter}
	Tā sauktā bitu bīdes loģiskā ierīce realizē bitu bīdes operācijas kuras
	šī procesora realizācijā ir izdalītas atsevišķi no ALU. Līdzīgi ALU
	realizācijai — darbojas asinhroni.

	\begin{figure}[thb]
		\centering
		%\def\svgwidth{7cm}
		\def\svgscale{1.25}
		{\ttfamily\scriptsize\input{img/sub-shift.pdf_tex}}
		\caption{Bitu bīdes loģiskā ierīce.}
		\label{fig:shift}
	\end{figure}
	
	\begin{singlespace}
		\lstinputlisting[language={[qucs]VHDL},%float=pb,%
		                caption={Bīdes ierīces VHDL apraksts. (\texttt{shifter.vhd})},%
		                label=kb:shifter,%
		                tabsize=3%,% TO REDUCE THE CODE WIDTH
		                ]
			{code/shifter.vhd}
	\end{singlespace}

%\clearpage
\pagebreak[3]
\subsubsection{Komparators} \label{sec:comp}
	Komparators jeb salīdzinātājs ir komponente, kas salīdzina 
	divus operandus un izvada tā vienādības vai nevienādības signālu.
	
	Šī komparatora implementācija ir minimizēta, kombinacionāla un nesatur
	kontroles signālus. Izvadīti tiek tikai divi biti, no kuriem viens
	norāda uz operandu vienādību vai nevienādību, savukārt otrs norāda
	vai operands \texttt{a} ir lielāks par \texttt{b} vai nav. Ar šo
	informāciju ir pilnīgi pietiekami, lai būtu iepējams realizēt visas
	nosacījuma zarošanās (\mnem{BRxx}) instrukcijas.
	\begin{figure}[thp]
		\centering
		%\def\svgwidth{7cm}
		\def\svgscale{1.25}
		{\ttfamily\scriptsize\input{img/sub-comp.pdf_tex}}
		\caption{Komparators.}
		\label{fig:comp}
	\end{figure}
	
	\begin{singlespace}
		\lstinputlisting[language={[qucs]VHDL},%float=pb,%
		                caption={Komparatora VHDL apraksts. (\texttt{comp.vhd})},%
		                label=kb:comp]
			{code/comp.vhd}
	\end{singlespace}

\pagebreak[3]
\subsubsection{Kontroles iekārta}
	Viena no procesora definējošām un iespējams sarežģītākajām komponentēm
	— kontroles iekārta ir tā kas nodrošina mikro-operāciju izpildi.
	Kontroles iekārta realizēta kā stāvokļa mašīna, kur katrs stāvoklis
	atbilst mikro-operācijai, un tātad tās VHDL apraksts ir uzskatāms par
	realizētā procesora mikrokodu.
	Kontroles iekārtas VHDL apraksts ir garš un sarežģīts, un tā daļas tiks
	izskatītas instrukciju darbības aprakstos.\\
	(Pilno kodu sk.~pielikumā~\ref{appx:control} \pageref{appx:control}.~lpp.)
	\begin{figure}[h!]
		\centering
		%\def\svgwidth{7cm}
		\def\svgscale{1.1}
		{\ttfamily\scriptsize\input{img/sub-control.pdf_tex}}
		\caption{Kontroles iekārta.}
		\label{fig:control}
	\end{figure}
 %\pagebreak[3]
	% CPU LIB
		\clearpage 
		\subsection{Procesora VHDL bibliotēka}
		\begin{singlespace}
			\lstinputlisting[language={[qucs]VHDL},%float=pb,%
			                 caption={Procesora \texttt{cpu\_lib} pakas definīcija. (\texttt{cpu\_lib.vhd})},%
			                 label=kb:cpulib,
			                 basicstyle=\ttfamily\scriptsize]
				{code/cpu_lib.vhd}
		\end{singlespace}
		\clearpage
	\subsection{Instrukciju kopa} \label{sec:instrSet}
Instukciju kopa ir bāzēta uz RISC procesoru arhitektūrām, un
satur nedaudz vairāk par minimumu funkcionālam procesoram.
Instrukciju operāciju kodi ir vektoriski, lai efektīvi izmantotu pieejamo instrukcijas 
vārda garumu.

\begin{singlespace}\small
\begin{longtable}[c]{lp{20ex}lp{0.36\textwidth}}
	%\centering
	\caption{Instrukciju tabula.}\label{tbl:instructions}\\
	\toprule
	\textbf{Apz.} & \textbf{Mašīnkods} & \textbf{Argumenti} & \textbf{Operācija} \\
	\midrule \endfirsthead
	\caption[]{\nameref{tbl:instructions}~(turpinājums).}\\
	%\toprule
	\midrule
	\textbf{Apz.} & \textbf{Mašīnkods} & \textbf{Argumenti} & \textbf{Operācija} \\
	\midrule \endhead
	\multicolumn{4}{c}{Atmiņas datu apmaiņas instrukcijas}\\
	\midrule
	\mnem{LD} & 	\instr{01}{00}{}{XXXXXX}{XXX}{XXX}{} & \texttt{rD, rS} &
		\texttt{rD = *(rS)} \newline
		{\footnotesize Ielasa vārdu no RAM reģistrā \texttt{rD}} \\ \midrule
	\mnem{LDI} & 	\instr{01}{01}{}{XXXXXX}{XXX}{}{XXX} \newline
					\instr{}{}{}{}{}{XXXXXXXXXXXXXXXX}{} & \texttt{rD, C} &
		\texttt{rD = C} \newline
		{\footnotesize Ielasa konstanti reģistrā \texttt{rD}} \\ \midrule
	\mnem{ST} & 	\instr{01}{11}{}{XXXXXX}{XXX}{XXX}{} & \texttt{rD, rS} &
		\texttt{*(rD) = rS} \newline
		{\footnotesize Ielasa vārdu no RAM reģistrā \texttt{rD}} \\
	\midrule \pagebreak[3]
	\multicolumn{4}{c}{Aritmētika, loģika un bitu bīdes (un \mnem{AR} operācijas saīsnes)}\\
	\midrule
	\mnem{AR} & 	\instr{10}{}{}{}{XXXX}{XXX}{X}\instr{}{}{}{}{XXX}{XXX}{} & \texttt{kA, kS, rD, rS} &
		{\footnotesize Aritmētikas/bīdes instrukcija.} \\ \midrule
	\rule{0pt}{1em}\mnem{ADD} & \instr{10}{0000}{000}{X}{XXX}{XXX}{} & \texttt{rD, rS} &
		\verb|rD = rD + rS| \\ \midrule
	\rule{0pt}{1em}\mnem{SUB} & \instr{10}{0001}{000}{X}{XXX}{XXX}{} & \texttt{rD, rS} &
		\verb|rD = rD - rS| \\ \midrule
	\rule{0pt}{1em}\mnem{INC} & \instr{10}{1000}{000}{X}{XXX}{}{XXX} & \texttt{rD} &
		\verb|rD = rD + 1| \\ \midrule
	\rule{0pt}{1em}\mnem{DEC} & \instr{10}{1001}{000}{X}{XXX}{}{XXX} & \texttt{rD} &
		\verb|rD = rD - 1| \\ \midrule
	\rule{0pt}{1em}\mnem{AND} & \instr{10}{0010}{000}{X}{XXX}{XXX}{} & \texttt{rD, rS} &
		\verb|rD = rD & rS| \\ \midrule
	\rule{0pt}{1em}\mnem{OR} & \instr{10}{0011}{000}{X}{XXX}{XXX}{} & \texttt{rD, rS} &
		\verb+rD = rD | rS+ \\ \midrule
	\rule{0pt}{1em}\mnem{XOR} & \instr{10}{0100}{000}{X}{XXX}{XXX}{} & \texttt{rD, rS} &
		\verb|rD = rD ^ rS| \\ \nopagebreak \midrule
	\rule{0pt}{1em}\mnem{NOT} & \instr{10}{1010}{000}{X}{XXX}{}{XXX} & \texttt{rD} &
		\verb|rD = ~rD| \\ \midrule
	\rule{0pt}{1em}\mnem{CLR} & \instr{10}{1111}{000}{X}{XXX}{}{XXX} & \texttt{rD} &
		\verb|rD = 0| \\ \midrule
	\rule{0pt}{1em}\mnem{MOV} & \instr{10}{0101}{000}{X}{XXX}{XXX}{} & \texttt{rD, rS} &
		\verb|rD = rS| \\ \midrule
	\rule{0pt}{1em}\mnem{LSL} & \instr{10}{0101}{001}{X}{XXX}{XXX}{} & \texttt{rD} &
		\verb|rD = rS * 2| \newline
		{\footnotesize Loģiskā kreisā bīde (reizina ar 2)} \\ \midrule
	\rule{0pt}{1em}\mnem{LSR} & \instr{10}{0101}{010}{X}{XXX}{XXX}{} & \texttt{rD} &
		\texttt{rD = rS / 2} \newline
		{\footnotesize Loģiskā labā bīde (dala ar 2)} \\ \midrule
	\rule{0pt}{1em}\mnem{ROL} & \instr{10}{0101}{011}{X}{XXX}{XXX}{} & \texttt{rD} &
		{\footnotesize Bitu rotācija pa kreisi} \\ \midrule
	\rule{0pt}{1em}\mnem{ROR} & \instr{10}{0101}{100}{X}{XXX}{XXX}{} & \texttt{rD} &
		{\footnotesize Bitu rotācija pa labi} \\ \nopagebreak
	\midrule \pagebreak[3]
	\multicolumn{4}{c}{Plūsmas kontroles instrukcijas}\\
	\midrule
	\mnem{NOP} & 	\instr{00}{}{}{00000000000000}{}{}{} & nav &
		{\footnotesize „Ne-operācija”} \\ \midrule
	\mnem{HLT} & 	\instr{11}{0011}{}{XXXX}{}{}{XXXXXX} & nav &
		{\footnotesize Apstādina procesora darbību} \\ \midrule
	\mnem{JMP} & 	\instr{11}{0010}{}{XXXX}{}{}{XXXXXX} \newline
					\instr{}{}{}{}{XXXXXXXXXXXXXXXX}{}{} & \texttt{L} &
		\texttt{PC = L} \newline
		{\footnotesize Beznosacījuma lēciens.} \\ \midrule
	\mnem{BREQ} & 	\instr{11}{1000}{}{XXXX}{XXX}{XXX}{} \newline
					\instr{}{}{}{}{XXXXXXXXXXXXXXXX}{}{} & \texttt{rD, rS, L} &
		\texttt{if(rD==rS) PC = L} \\ \midrule
	\mnem{BRNQ} & 	\instr{11}{1011}{}{XXXX}{XXX}{XXX}{} \newline
					\instr{}{}{}{}{XXXXXXXXXXXXXXXX}{}{} & \texttt{rD, rS, L} &
		\texttt{if(rD!=rS) PC = L} \\ \midrule
	\mnem{BRGT} & 	\instr{11}{1010}{}{XXXX}{XXX}{XXX}{} \newline
					\instr{}{}{}{}{XXXXXXXXXXXXXXXX}{}{} & \texttt{rD, rS, L} &
		\texttt{if(rD>rS) PC = L} \\ \midrule
	\mnem{BRGE} & 	\instr{11}{1001}{}{XXXX}{XXX}{XXX}{} \newline
					\instr{}{}{}{}{XXXXXXXXXXXXXXXX}{}{} & \texttt{rD, rS, L} &
		\texttt{if(rD>=rS) PC = L} \\ \midrule
	\mnem{BRLT} & 	\multicolumn{2}{c}{subst.~\texttt{\textbf{BRGE} rS, rD, L}} &
		\texttt{if(rD<rS) PC = L}\\ \midrule
	\mnem{BRLE} & 	\multicolumn{2}{c}{subst.~\texttt{\textbf{BRGT} rS, rD, L}} &
		\texttt{if(rD<=rS) PC = L}\\
	\bottomrule
	\caption*{\fboxrule=0.75pt \framebox{\footnotesize
		\begin{tabular}{ll}
			\multicolumn{2}{c}{Mašīnkoda krāsu apzīmējumi} \\
			\textcolor{purple}{\rule[-2pt]{1em}{1em}} Operāciju grupas kods &
			\textcolor{blue}{\rule[-2pt]{1em}{1em}} \textcolor{cyan}{\rule[-2pt]{1em}{1em}}
				Operācijas „vektora” kodi \\[2pt]
			\textcolor{lightgray}{\rule[-2pt]{1em}{1em}} Ignorētie biti &
			\textcolor{OliveGreen}{\rule[-2pt]{1em}{1em}} \textcolor{Green}{\rule[-2pt]{1em}{1em}}
				Argumentu biti \\
		\end{tabular}
		}}
\end{longtable}
\end{singlespace}
\normalsize

\pagebreak
Kā redzams instrukciju tabulā, instrukcijas
ir sadalītas grupās vai nu pēc nozīmes, vai pēc izpildes līdzības. Šo
instrukciju realizācijas detaļas tiks apskatītas turpmākajās apakšnodaļās.

\subsubsection{\mnem{AR} instrukcija} \label{sec:AR}
	Lai gan \mnem{AR} instrukcijas apzīmējums ir saīsinājums no „aritmētika”
	un tā ir instrukcijas primārā nozīme, \mnem{AR} implementē visas
	aritmētiskās, loģiskās, bīdes operācijas, kā arī reģistru 
	apmaiņas \mnem{MOV} instrukciju. (sk.~\ref{tbl:instructions}.~tabulu)
	Šāda implementācija izmantota tāpēc, ka visas šīs instrukcijas tiek
	pārvadītas pa to pašu signālceļu. Tādējādi kontroles iekārta tiek
	vienkāršota, jo atsevišķās operācijas nav nepieciešams izšķirt.
	
	Kontroles signāli aritmētiskajai un bīdes ierīcei,
	kā argumenti jeb vektora kodi,
	tiek nodoti tieši no instrukcijas vārda, un kontroles iekārtā netiek
	interpretēti (jeb ir „necaurspīdīgi”). Izņēmums, gan ir 
	unārās un bezargumentu operācijas
	\mnem{INC}, \mnem{DEC}, \mnem{NOT} un \mnem{CLR}, kuru ALU
	vektora koda vecākais bits (\texttt{=1}) tiek interpretēts, lai izlaistu
	otrā operanda ielasi, tā samazinot nepieciešamo takts ciklu skaitu
	instrukcijas izpildei. (sk.~\ref{kb:ARdecode}.~koda bloku)
	
	\begin{singlespace}
		\lstinputlisting[language={[qucs]VHDL},%float=pb,%
		                caption={\mnem{AR} instrukcijas dekodēšana. (\texttt{control2.vhd})},%
		                label=kb:ARdecode,%
		                firstnumber=150]
			{code/gen/ardecode-snippet.vhd}
	\end{singlespace}


\subsubsection{Nosacījuma zarošanās instrukcijas} \label{sec:branching}
	Nosacījuma zarošanās ir vitāla īpašība pilnīgam procesoram. Programmēšanas
	valodu \texttt{if()} konstrukcija nav iedomājama bez
	nosacījuma zarošanās.
	
	Izstrādātais procesors par nosacījumu var izmantot divu reģistru
	vienā\-dības vai nevienādības, kur reģistru saturs tiek interpretēts, kā
	bezzīmes veseli skaitļi (t.i.~\texttt{unsigned~integer} tips).
	Šai salīdzināšanai atbilst \mnem{BRxx} instrukcijas
	(sk.~\ref{tbl:instructions}.~tabulu), un pašu salīdzināšanu veic komparators
	(sk.~\ref{sec:comp}.~nod.) kuram tiek pievadīti salīdzināmie operandi.
	
	\noindent Aparatūras līmenī realizētas četras zarošanās instrukcijas:
	\begin{itemize}
		\item \mnem{BREQ} — zaroties, ja operandi ir vienādi;
		\item \mnem{BRNQ} — zaroties, ja operandi nav vienādi;
		\item \mnem{BRGE} — zaroties, ja pirmais operands ir lielāks
			vai vienāds ar otro;
		\item \mnem{BRGT} — zaroties, ja pirmais operands ir stigri lielāks
			par otro.
	\end{itemize}
	Šo instrukciju operāciju kodi nav izvēlēti gluži patvaļīgi, tie tiek
	izmantoti loģiskā izteiksmē kopā ar komparatora signāliem, lai noteiktu
	zarošanās izpildes nosacījumus.
	
	\begin{figure}[thp]
		\centering
		%\def\svgwidth{7cm}
		\def\svgscale{1.5}
		{\ttfamily\input{img/karnaugh.pdf_tex}}\\
		\(
			f = \mathtt{\overline{A}E + BG + AG + AB\overline{E}}
		\)\\[1ex]
		(apzīmējumu nozīmi sk.~\ref{kb:branchTest}.~kodā)
		\caption{Zarošanās nosacījuma \termEn{Karnaugh} karte un formula.}
		\label{fig:branch-karnaugh}
	\end{figure}
	
	\begin{singlespace}
		\lstinputlisting[language={[qucs]VHDL},%float=pb,%
		                caption={Zarošanās nosacījuma pārbaudes funkcija (\texttt{control2.vhd})},%
		                label=kb:branchTest,%
		                linerange={61-69},firstnumber=61,
		                emph={state,branchTest},%
		                breaklines,breakatwhitespace,
		                basicstyle=\ttfamily\scriptsize]
			{code/control2.vhd}
	\end{singlespace}
	
	Atlikušās zarošanās instrukcijas \mnem{BRLT} un \mnem{BRLE} ir
	programmatūras līmeņa substitūcijas, jo to ekvivalentus var iegūt
	izmantojot attiecīgi \mnem{BRGE} un \mnem{BRGT} instrukcijas,
	apmainot salīdzināmos operandus vietām.
	
	
 \clearpage %\pagebreak[3]
	\subsection{Revīziju vēsture} \label{sec:cpu-revs}
%\todo

\begin{description}
	\item[Rev.~01] \hfill \\
		Par pamatu ņemta Perija grāmatas\citeet{Perry-VHDL} procesora
		realizācija. Atsevišķas komponentes izveidotas ļoti līdzīgi, bet
		kontroles iekārtas modelis un tam piekārtotā instrukciju tabula
		pilnībā izstrādāta no jauna.
	\item[Rev.~02] \hfill \\
		\termEn{Actel Fusion} FPGA nepiedāvā trīs-stāvokļu loģiku.\citeet{FusionFAQ}
		Tā rezultātā
		datu apmaiņas šina, pārveidota no kopējas arbitējamas 
		\termEn{multidrop} realizācijas pārveidota uz sazarotu vienvirziena
		pārraides ķēdi. Šāds solis pavildus ļāva arī likvidēt trīs-stāvokļu
		buferus un samazināt kontroles iekārtas signālu skaitu.
	\item[Rev.~03] \hfill \\
		Šī revīzija nav ienesusi fundamentālu izmaiņu procesora uzbūvē.
		(Izmaiņas sistēmas perifērijā skatīt
			\ref{sec:sys-revs}.~nod.~\pageref{sec:sys-revs}.~lpp.)
\end{description}
 %\pagebreak[3]
