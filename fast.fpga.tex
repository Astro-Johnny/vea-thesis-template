\subsubsection{FAST FPGA implementācijas modelis} \label{sec:fast-fpga}
FAST potenciālās ātrdarbības salīdzināšanai FPGA platformai, autors
izstrādājis savu implementācijas modeli. Pēc vairākiem uzbūves variantiem,
%TODO?: Atmestie modeļi?
autors par piemērotāko atzinis modeli, kas balstās atsevišķu apstrādes
vienību izmantoto resursu samazināšanu, un caurlaidspējas palielināšanu
instancējot lielu skaitu šo vienību.

\begin{figure}[tbh]
	\centering
	\def\svgwidth{\linewidth}
	{\small\input{img/fpga-model.pdf_tex}}
	\caption{Vienkāršota uzbūves shēma viena punkta apstrādes vienībai.}
	\label{fig:fast-fpga}
\end{figure}

Tā kā viena elementa --- punkta $\vec{p}$ piederības $F_9(\vb{A},t)$ ---
noteikšanas latentums nav prioritāte, apstrādes vienība, kā ilustrēts
\ref{fig:fast-fpga}~attēlā, resursu ekonomijas nolūkos, ir secīgas uzbūves.
Salīdzināšana ar apkārtnes punktiem notiek secīgi, to intensitātes vērtības
pārvietojot caur bīdes reģistru un ar skaitītājiem uzskaitot secīgu
gaišo un tumšo loka punktu skaitu. Šāda FAST-9 apstrādes vienība vienu punktu
var klasificēt $16+(n-1)$ jeb 24 takts ciklos, bet papildus takts cikli
var būt nepieciešami datu sagatavošanai.

Katra FAST punkta apstrādes vienība tad var tikt instancēta un vairāki
punkti klasificēti paralēli. Testiem tika izveidots
,,attēlu apgabalu procesors'', kas vienlaikus apstrādāja $m \times m$ attēla
apakškopu, kur ${(m-6)}^2$ punkta apstrādes instances klasificē attēla
apakškopas punktus. Sintēzes rezultāti uzrādīja CPU līdzvērtīgu datu
caurlaidspēju izmantojot tikai 18\% Virtex-6 resursu
(sk.~pielikumu~\ref{appx:test3}).

\TODO
