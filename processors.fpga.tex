\subsection{FPGA} \label{sec:fpga}
FPGA (angļu \termEn{field-programmable gate array}) jeb
pārprogrammējams loģisko elementu masīvs ir integrētā shēma, kura sastāv no
liela skaita programmējamiem loģiskajiem blokiem. Šie bloki ir pārprogrammējami
izmantojot aparatūras apraksta valodas (HDL) un ražotāja programmatūru,
ļaujot lietotājam ar FPGA realizēt vēlamo funkcionalitāti.
FPGA loģisko bloku un starpsavienojumu tīkla (\termEn{the interconnect})
uzbūve, loģisko bloku un citu resursu skaits ir ražotāja un
FPGA modeļa specifiska. \cite{JIS}

FPGA, kā skaitļošanas resurss, ir unikāls ar to, ka tā arhitektūras definēšana
ir ļoti elastīga un, vairumā gadījumu, arhitektūru pielāgo veicamajam
uzdevumam vai algoritmam, nevis otrādi, kā iepriekš apskatītajiem CPU un GPU.

\begin{figure}[tbh]
	\centering
	\def\svgscale{1.2}
	{\input{img/FPGA-arch.pdf_tex}}
	\caption{Fiksēta signāltrakta arhitektūra.}
	\label{fig:fpga-arch}
\end{figure}
Autorprāt, efektīvs un visai intuitīvs arhitektūras uzbūves modelis ir
fiksēta signāltrakta modelis. Kā ilustrēts \ref{fig:fpga-arch}~attēlā, šādā
modelī datu vienība --- tipiski, matrica vai vektors --- tiek virzīta caur
signāltraktam, kur tā tiek transformēta lai iegūtu rezultātu. Atšķirībā no
CPU un GPU, kur skaitļošanas algoritmus realizē ar secīgu instrukciju izpildi
starprezultātus no ALU (aritmētiski loģiskās ierīces)
atgriežot uzglabāšanai koplietojamos reģistros vai RAM,
FPGA, pēc signāltrakta modeļa, RAM vai koplietojamie reģistri nav
nepieciešami, jo starprezultāti tiek nākamajām pakāpēm nodoti tieši.
Šādi tiek efektīvi izmantoti FPGA resursi un sekmēta ātrdarbība.
%~ Nav nepieciešams dekodēt instrukcijas...

Pēc CPU analoģijas, var uzskatīt, ka šādā modelī FPGA, veic vienu
īpaši augstas kompleksitātes instrukciju kura tiek izpildīta noteikta,
iespējams mainīga, skaita takts ciklu laikā.

Ja ir pieejami FPGA resursi, tad var iegūt datu paralelitāti replicējot 
vairākus paralēlus signāltraktus, palielinot datu caurlaides spēju. Kā arī, ja ir
jāapstrādā datu straume, ievērojamu datu caurlaides spējas uzlabojumu var
iegūt ar ,,konveijera principu'' --- virzot jaunu datu vienību
signāltrakta pakāpē pirms iepriekšējā datu vienība ir šķērsojusi visu
signāltraktu, tādējādi signāltraktā, dažādās pakāpēs, vienlaikus
var atrasties vairākas datu vienības, līdzīgi CPU instrukciju ,,konveijeram''
(\termEn{instruction pipeline})~\cite{Flynn-arch}.

Papildus apsvērums realizācijai ir gadījumi, kad viena neatkarīgā datu vienība
--- ar ko saprot datu kopu, kuras apstrādei nav ārēju atkarību --- ir
ļoti liela, piem.,~vesels attēls (pikseļu kopa).
% TODO: Piemērs no praktiskā darba -- atgūt patchu pēc keypoint apstiprinājuma
\begin{figure}[tbh]
	\centering
	\def\svgscale{1.2}
	{\input{img/FPGA-arch2.pdf_tex}}
	\caption{Modificēta signāltrakta arhitektūra ar atliktu datu ielasi.}
	\label{fig:fpga-arch2}
\end{figure}
Šo situāciju var risināt apstrādājot datus pa daļām, ja konkrētai pakāpei
nepieciešama tikai datu vienības apakškopa. Veiktspējas palielināšanai datu
kopu vēlams glabāt lokālā atmiņā, kam noder FPGA specializētie atmiņas
resursi (RAM bloki). Šāda arhitektūras modifikācija ilustrēta
\ref{fig:fpga-arch2}~attēlā.

Atliktā datu ielase ir īpaši izdevīga, ja
atliktā datu daļa nav nepieciešama vienmēr, t.i.,~tās nepieciešamība
atkarīga no iepriekšējās pakāpes rezultāta.
Šajā darbā praktiskais piemērs ir, kur raksturpunkta \termTech{deskriptors}
jānosaka tikai attēla apgabaliem, kam nepieciešami vairāk datu, nekā pakāpei,
kura nosaka atbilstību raksturpunkta kritērijiem.
%TODO: Forward ref


