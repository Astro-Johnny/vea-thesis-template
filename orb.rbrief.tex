\subsubsection{rBRIEF deskriptors} \label{sec:rbrief-def}
rBRIEF ir rotācijas invarianta modifikācija
BRIEF deskriptoram (sk.~\ref{sec:brief}~nod.),
kura tiek izmantota par salāgošanas komponenti ORB
algoritmam. Idejas pamatā ir aprēķināt BRIEF deskriptoru izmantojot
rotētas salīdzināšanas pāru punktu koordinātes, rotācijas leņķi nosakot pēc
raksturpunkta virziena komponentes (sk.~\ref{sec:ofast}~nod.).

Apzīmēsim rBRIEF deskriptoru ar $D_{n_d}$, kuru mēs varam definēt
pievienojot BRIEF deskriptoram $B_{n_d}$ rotācijas komponenti ar leņķi $\theta$:
\begin{equation}\label{eq:rbrief}
	D_{n_d}(\hat{\vb{p}}, \theta) := \sum_{i=1}^{n_d} 2^{i-1}
		\tau\left(\hat{\vb{p}},
		          \vb{R}(\theta) \times \vb{a}_i,
		          \vb{R}(\theta) \times \vb{b}_i\right)
\end{equation}
kur $\vb{R}(\theta)$ ir rotācijas matrica leņķim $\theta$, kas ir definēta kā:
\[
	\vb{R}(\theta) = 
		\begin{pmatrix}
			\cos\theta & -\sin\theta\\
			\sin\theta & \cos\theta
		\end{pmatrix}
\]

Uzdotā definīcija \eqref{eq:rbrief} ir idejiski ekvivalenta bet 
pierakstīta ievērojami citādāk nekā \cite{ORB} avotā, jo 
Rublē~u.c. visai liberāli reinterpretē mainīgo nozīmi.

Jānorāda, ka nodrošinot rotācijas invarianci, šī informācija tiek
,,atmesta'' no deskriptora, kas samazina tā varianci un tādējādi arī
tā diskriminitāti. Rublē~u.c.\cite{ORB}, šī samazinājuma kompensēšanai,
izstrādā mašīnmācīšanās metodes, ar kuru palīdzību tiek izvēlēti
salīdzināšanas pāri ar augstāko varianci. Izvēlētie punkti attēloti
\ref{fig:pattern2}a.~attēlā un ir novērojama salīdzināšanu pāru tendence
orientēties raksturpunkta rotācijas virzienā (attēlā virzienā uz augšu).
\begin{figure}[tbh]
	\centering
	\def\svgwidth{0.7\linewidth}
	{\input{img/rBRIEF.pdf_tex}}
	\caption{Mašīnmācīšanās metožu izvēlēti salīdzināšanas pāri
		(attēloti 64 pāri)~\cite{ORB}.}
	\label{fig:pattern2}
\end{figure}
Bet ir arī novērojams, ka izvēlētie pāri ir izvietoti līdzīgi, kas palielina
to savstarpējo korelāciju
un tādējādi samazina to informatīvo vērtību. Rublē~u.c.\cite{ORB} izvēles
kritēriju papildina atlasot augstākās variances pārus, kuri atbilst
noteiktam korelācijas slieksnim (sk.~\ref{fig:pattern2}b.~att.), panākot
augstāku kopējo deskriptora diskriminitāti.
