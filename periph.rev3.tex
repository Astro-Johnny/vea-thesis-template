%\subsection{Papildinātā shēma (rev.~03)}

%Trešā sistēmas revīzija pievieno patstāvīgās atmiņas elementus sākotnējās
%programmas un datu uzglabāšanai. Šīs atmiņas ierīces piekārtotas kopējai
%adresācijas telpai, tādējādi rodas nepieciešamība pēc adreses dekodēšanas
%loģikas, kuru nodrošina MMU (\termEn{Memory map unit}).

\subsection{MMU — adrešu dekoderis} \label{sec:mmu}
	Adrešu dekoderis jeb MMU ir ierīce, kas atbild par adrešu telpas
	pārdalīšanu dažādām ierīcēm, kā arī nodrošina korektu komunikāciju
	ar šīm ierīcēm. MMU tādējādi var uzskatīt par mikrokontroliera kontroles
	iekārtu. MMU arī satur praktiski visu implementācijas un platformas
	specifisko kodu, kas tadējādi ir galvenā komponente kurā jāveic izmaiņas
	(vai pat jāimplementē no jauna) izmainot mikrokontroliera uzbūvi.
	
	MMU ar procesoru veido tādu pašu saskarni kā RAM,
	savukārt, piekārtojamo ierīču saskarne ir implementācijas definēta.
	Tādējādi MMU var uzskatīt par aparatūras līmeņa abstrakcijas slāni.
	
	\begin{figure}[thp]
		\centering
		\def\svgwidth{0.9\textwidth}
		{\ttfamily\small\input{img/remap.pdf_tex}}
		\caption{Adrešu telpas sadalījums.}
		\label{fig:memory-map}
	\end{figure}
	
	Adrešu telpas sadalījums (sk.~\ref{fig:memory-map}~att.) seko vienkāršiem pamatprincipiem:
	\begin{itemize}
		\item lielākie atmiņas bloki (konkrēti RAM) tiek novietoti augstākos
			atmiņas apgabalos, vienkāršojot dekodēšanas loģiku;
		\item patstāvīgo atmiņu (ROM) nepieciešams piekārtot ar sākuma
			adresi \texttt{0x0000}, jo tā saturēs sāknēšanas programmu,
			kuru nepieciešams izpildīt pēc mikrokontroliera atiestatīšanas
			(sk.~\ref{sec:rom}~nod.);
	\end{itemize}
	
	Papildus MMU arī satur konfigurācijas reģistru, ar kura palīdzību
	maināmi pieslēgto ierīču konfigurējamie parametri.
	\todo


\subsection{Patstāvīgā atmiņa} \label{sec:rom}
	Patstāvīgā atmiņa jeb ROM
	
	\termEn{Boot ROM} ir tikai nolasāmā atmiņa, kura paredzēta
	\termEn{bootstrap} procesa programmas uzglabāšanai.
	Šīs programmas uzdevums ir ielādēt reālo izpildes programmu no
	patstāvīgās atmiņas — šajā implementācijā konkrēti 
	no SPI \termEn{Flash} atmiņas.
	
	Par \termEn{Boot ROM} kalpos \termEn{Actel Fusion} piedāvātais
	\termEn{FlashROM} makross.\cite{FlashROM}
	Maksimāli pieejamais ROM garums uz izmantojamās platformas ir
	1 kilobits jeb 64 vārdi.\cite{FusionGuide}
	
	\todo
	
	\begin{figure}[thp]
		\centering
		\def\svgwidth{\textwidth}
		{\ttfamily\small\input{img/rom-access.pdf_tex}}
		\caption{Laika diagramma vārda nolasei no ROM.}
		\label{fig:rom-time-diag}
	\end{figure}

\subsection{Operatīvā atmiņa}
	\todo

\subsection{Adrešu telpai piekārtotā ievade/izvade}
	Adrešu telpā arī piekārtotas ievades un izvades komponentes,
	kuras nodrošina saskarni ar sistēmas perifēriju. Adrešu telpā šīm
	komponentēm izvietoti datu apmaiņas un konfigurācijas reģistri.
	
	\subsubsection{Vispārējas nozīmes ievades/izvades pieslēgvietas}
	\todo
	
	\subsubsection{SPI saskarne}
	SPI (\termEn{Serial Peripheral Interface Bus}) ir seriālās
	datu pārraides saskarne, ar vienpusēju plūsmas kontroli pēc
	\termEn{Master/Slave} modeļa.
	
	
	
	Uz doto brīdi (un kā redzams blokshēmā — \ref{fig:top-rev3}.~att.)
	paredzēta tikai SPI saskarne. Tam paredzēts viens pārraides un viens
	konfigurācijas reģistrs. Tam paredzēta iespēja nokonfigurēt datu
	apmaiņu pa baitam (8 biti) vai pa vārdam (16 biti). Datu pārraide tiek
	uzsākta uzreiz pēc jauno datu ierakstes reģistrā (ja SPI iespējots) un
	tiek pārraidīti bez procesora līdzdalības.
	
	\begin{figure}[thp]
		\centering
		%\def\svgwidth{7cm}
		\def\svgscale{1.25}
		{\ttfamily\scriptsize\input{img/sub-spi.pdf_tex}}
		\caption{SPI saskarnes ierīce.}
		\label{fig:spi}
	\end{figure}
	
	SPI saskarne nepieciešama komunikācijai ar SPI \termEn{Flash} atmiņu,
	kura pieejama uz izstrādes platformas,\cite[43.~lpp.]{FusionGuide}
	kur tiks glabāta galvenā izpildāmā programma, kura \termEn{boot} procesa
	laikā tiks pārvietota uz RAM.
