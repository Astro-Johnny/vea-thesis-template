%\subsection{Papildinātā shēma (rev.~03)}

%Trešā sistēmas revīzija pievieno patstāvīgās atmiņas elementus sākotnējās
%programmas un datu uzglabāšanai. Šīs atmiņas ierīces piekārtotas kopējai
%adresācijas telpai, tādējādi rodas nepieciešamība pēc adreses dekodēšanas
%loģikas, kuru nodrošina MMU (\termEn{Memory map unit}).

\subsection{MMU — adrešu dekoderis} \label{sec:mmu}
	Adrešu dekoderis jeb MMU ir ierīce, kas atbild par adrešu telpas
	pārdalīšanu dažādām ierīcēm, kā arī nodrošina korektu komunikāciju
	ar šīm ierīcēm. MMU tādējādi var uzskatīt par mikrokontroliera kontroles
	iekārtu. MMU arī satur praktiski visu implementācijas un platformas
	specifisko kodu, kas tadējādi ir galvenā komponente kurā jāveic izmaiņas
	(vai pat jāimplementē no jauna) izmainot mikrokontroliera uzbūvi.
	
	MMU ar procesoru veido tādu pašu saskarni kā RAM,
	savukārt, piekārtojamo ierīču saskarne ir implementācijas definēta.
	Tādējādi MMU var uzskatīt par aparatūras līmeņa abstrakcijas slāni.
	
	\begin{figure}[thp]
		\centering
		\def\svgwidth{0.9\textwidth}
		{\ttfamily\small\input{img/remap.pdf_tex}}
		\caption{Adrešu telpas sadalījums.}
		\label{fig:memory-map}
	\end{figure}
	
	Adrešu telpas sadalījums (sk.~\ref{fig:memory-map}~att.) seko vienkāršiem pamatprincipiem:
	\begin{itemize}
		\item lielākie atmiņas bloki (konkrēti RAM) tiek novietoti augstākos
			atmiņas apgabalos, vienkāršojot dekodēšanas loģiku;
		\item patstāvīgo atmiņu (ROM) nepieciešams piekārtot ar sākuma
			adresi \texttt{0x0000}, jo tā saturēs sāknēšanas programmu,
			kuru nepieciešams izpildīt pēc mikrokontroliera atiestatīšanas
			(sk.~\ref{sec:rom}~nod.);
	\end{itemize}
	
	Papildus MMU arī satur konfigurācijas reģistru,
	ar kura palīdzību maināmi pieslēgto ierīču konfigurējamie parametri.
	Tas nepieciešams lai lokalizētu šo parametru bitus \todo
	


\subsection{Sāknēšanas ROM} \label{sec:rom}
	Sāknēšanas ROM ir tikai nolasāmās, pastāvīgās atmiņas komponente, kura
	satur sāknēšanas (\termEn{boot}) programmu.
	Šīs programmas uzdevums ir ielādēt reālo izpildes programmu no
	ārēja datu avota. Parauga implementācijas testam tiek izmantota
	sāknēšanas programma, kas ielasa izpildāmo programmu no SPI \termEn{Flash}
	atmiņas (sk.~pielikumu~\ref{appx:boot}), kas atrodas uz izmantotās
	darba platformas (sk.~\ref{sec:devkit}~nod.).
	
	ROM realizēts ar \termEn{Actel Fusion} piedāvāto
	\texttt{FlashROM} makrosu, kas izmanto \termEn{Actel Fusion} FPGA
	speciālo \termEn{FlashROM} atmiņu \cite{FlashROM} (sk.~\ref{sec:devkit}~nod.).
	Maksimāli pieejamais ROM garums uz izmantojamās platformas ir
	1 kilobits jeb 64 vārdi \cite[12.~lpp.]{FusionGuide}. ROM mazā ietilpība ir
	galvenais iemesls kādēļ mikrokontrolieris realizēts ar sāknēšanu.
	
	\termEn{FlashROM} adresējamā datu vārda garums ir 8 biti (1 baits),
	tādēļ realizēta speciāla nolases loģikas saskarne, kas nolasa divus
	baitus un kombinē tos vienā 16 bitu mašīnvārdā pirms to nodot MMU.
	Šeit \termEn{FlashROM} glabā 16 bitu vārdus ar jaunāko baitu (LSB) 
	virspirms, bet nolases loģika šo niansi padara mazsvarīgu.
	
	Lai nodrošinātu korektu nolasi no \termEn{FlashROM}, bija nepieciešams
	modulēt \termEn{FlashROM} un nolases loģikas takts impulsa platumu.
	Šāda realizācija ir kompromiss, jo pagarina ROM nolases laiku, bet
	nodrošina pret nestabilitāti ,,skriešanās problēmas'' dēļ, kas bija
	novērojama mikrokontroliera darbības simulācijas laikā (pirms korekcijām).
	
	\begin{figure}[tbh]
		\centering
		\def\svgwidth{\textwidth}
		{\ttfamily\small\input{img/rom-access.pdf_tex}}
		\caption{Laika diagramma datu vārda nolasei no ROM.}
		\label{fig:rom-time-diag}
	\end{figure}
	
	\pagebreak[2]
	Nolases loģika datu nolasi no \termEn{FlashROM} veic vairākos soļos (sk.~\ref{fig:rom-time-diag}~att.):
	\begin{enumerate}[\hspace*{\parindent}1)]
		\item paceļot ROM izvēles signālu |rom_sel|, tiek iespējota nolases
			loģika un uzsākts nolases cikls;
		% FIXME: Is this right?
		\item paceļot nolases loģikas takti |fetch_clock| 
			tiek nodrošināta korekts adreses signāls;
		\item pie lecošās |rom_clk| frontes \termEn{FlashROM} nolasa pievadīto adresi,
			bet dati izejā nav definēti;
		\item \termEn{FlashROM} izvada datus pie krītošās |rom_clk|	frontes;
		\item pie |fetch_clock| krītošās frontes ROM dati tiek ielasīti
			pagaidu reģistrā --- tiek nolasīts vecākais baits, 
			2.--5.~solis tiek atkārtots arī jaunākā baita nolasei;
		\item nolasot jaunāko baitu tiek pacelts |ready| signāls
			(nav uzrādīts attēlā), pēc kura kodols (vai MMU) datus var
			nolasīt un atiestatīt |rom_sel| signālu.
	\end{enumerate}
	
	

\subsection{Operatīvā atmiņa} \label{sec:ram}
	Operatīvā atmiņa jeb RAM ir galvenā izpildes laika datu uzglabāšanas
	atmiņa. Tā kā kodols ir Fon~Neimaņa arhitektūras procesors
	(sk.~\ref{sec:cpu}~nod.), RAM tiek izmantots gan izpildāmās
	programmas mašīnkoda, gan datu uzglabāšanai.
	
	RAM realizēts ar \texttt{RAM4K9} makrosu, kas izmanto specializēto
	RAM, kas pieejams uz \termEn{Actel Fusion} FPGA \cite{RAM4K9}.
	Tādējādi tiek
	efektīvāk izmantoti FPGA resursi, jo nav nepieciešams izmantot
	programmējamos loģiskos blokus.
	
	\termEn{Actel Fusion} FPGA uzbūves un \texttt{RAM4K9} makrosa
	definīcijas dēļ, RAM ir sadalīts divos blokos \cite{RAM4K9}. 
	%Tā kā platums un garums
	%ir konfigurējami ir vairākas iespējas RAM sadalījumam.
	Katrs no abiem blokiem satur 6144 vārdus,
	kopā sastādot 12288 vārdu lielu RAM.
	
	\todo ?
	

\subsection{Adrešu telpai piekārtotā ievade/izvade}
	Adrešu telpā arī piekārtotas ievades un izvades komponentes,
	kuras nodrošina saskarni ar sistēmas perifēriju. Adrešu telpā šīm
	komponentēm izvietoti datu apmaiņas un konfigurācijas reģistri.
	
	\subsubsection{Vispārējas nozīmes ievades/izvades pieslēgvietas}
		Vispārējas nozīmes ievades/izvades pieslēgvietas jeb GPIO ir svarīga
		mikrokontroliera sastāvdaļa, kas ļauj saņemt un nodot signālus
		pieslēgvietām pievienotajām ierīcēm.
		
		GPIO pieslēgvietas apvieno kopās kuras sauc par portiem. Tipiski, bet
		ne obligāti, porta platums (atsevšķo pieslēgvietu skaits) ir vienāds
		ar mašīnvārda garumu (šeit~16). Porta platums nevar pārsniegt
		mašīnvārda garumu, jo tad, vienkārši, viss ports nav nolasāms, tādēļ ka dati
		neietilpst datu pārvades šinā. Tā vietā, mikrokontrolierī iestrādā
		vairākus GPIO portus.
		
		\begin{figure}[th]
			\centering
			\def\svgwidth{0.9\textwidth}
			{\ttfamily\footnotesize\input{img/gpio.pdf_tex}}\\
			\makebox[0.9\textwidth][r]{kur $N$ ir GPIO porta platums bitos}
			\caption{GPIO porta uzbūve.}
			\label{fig:gpio}
		\end{figure}
		
		Izstrādātā GPIO porta uzbūves pamatā ir
		bidirekcionālo buferu kopa, kura realizēta ar $N$ skaita
		\texttt{BIBUF} makrosa instancēm, kur $N$ ir porta platums,
		un divi $N$ garuma reģistri \texttt{DDRx} un \texttt{WRRx},
		no kuriem \texttt{DDRx} kontrolē pieslēgvietu darbības virzienu un
		\texttt{WRRx} uzglabā izvadāmo vārdu (pielēgvietām, kas ir izvades režīmā).
		
		Gan \texttt{DDRx}, gan \texttt{WRRx} ir lasāmi un pārrakstāmi.
		Pieslēgvietu satura nolasīšanai ir tikai lasāms ,,virtuālais reģistrs''%
		\footnote{Dati, kas nolasāmi kā reģistrs, kaut gan (mikrokontroliera iekšienē)
			neeksistē fizisks reģistrs datu uzglabāšanai.}
		\texttt{RDx}.
	
	Šajā mikrokontroliera paraugā realizēts viens 4 pieslēgvietu ports, kas
	pievienot uz darba platformas esošajām diodēm.
	
	\clearpage %\pagebreak[3]
	\subsubsection{SPI saskarne} \label{sec:spi}
		SPI (\termEn{Serial Peripheral Interface Bus}) ir sinhrona seriālās
		datu pārraides saskarne, ar vienpusēju plūsmas kontroli pēc
		\termEn{Master/Slave} modeļa [\todo{}]. SPI ir \termEn{de facto}%
		\footnote{Vispārēji pieņemts, bet neoficiāls.}
		standarts un ir liels skaits ierīču kas komunicē izmantojot SPI,
		tādējādi SPI ir vērtīga mikrokontroliera komponente.
		
		Darba ietvaros realizēta aparatūras kontrolēta paralēlā uz SPI saskarne.
		Šī saskarne implementē SPI \termEn{master} pusi, tādējādi mikrokontrolieris
		vada SPI komunikāciju un saskarnes paralēlajā pusē ir arī izvietoti
		plūsmas kontroles signāli (sk.~\ref{fig:spi}~att.).
		
		\begin{figure}[thp]
			\centering
			%\def\svgwidth{7cm}
			\def\svgscale{1.25}
			{\ttfamily\scriptsize\input{img/sub-spi.pdf_tex}}
			\caption{SPI saskarnes ierīce.}
			\label{fig:spi}
		\end{figure}
		
		SPI saskarne izstrādāta, lai programmatūras līmenī tā būtu vienkārši
		izmantojama. Datu pārraide viecama ar \mnem{ST} instrukciju ierakstot
		pārraidāmos datus speciālajā SPI datu reģistrā, kas tūlīt uzsāk datu
		apmaiņu. Šī pārraide notiek bez procesora kodola līdzdalības.
		Otras ierīces atgrieztos datus var atgūt vienkārši nolasot
		SPI datu reģistru (ar \mnem{LD} instrukciju). Pārraidāmo datu garumu
		ir iespējams konfigurēt datu apmaiņai pa baitam (8 biti) vai
		pa vārdam (16 biti).
		
		Realizētā SPI saskarnes iekārta izmanto divus 8 bitu bīdes reģistrus
		datu uzglabāšanai. Šo bīdes reģistru apvienojums atbilst SPI datu reģistram.
		Tipiski arī otra (\termEn{slave})
		ierīce līdzīgu bīdes reģistru, tādējādi tiek izveidots cilpveida
		bīdes reģistrs, kur atsevišķie bīdes reģistri, pēc vesela pārraides cilka,
		datus savā starpā apmaina. Pārraidīto bitu skaits tiek uzskaitīts
		ar iekšēju skaitītāju. 
		\begin{figure}[th]
			\centering
			\def\svgwidth{\textwidth}
			{\ttfamily\footnotesize\input{img/spi-tx.pdf_tex}}\\
			\caption{SPI pārraides laika diagramma.}
			\label{fig:spi-tx}
		\end{figure}
		Pārraide notiek vairākos soļos
		(sk.~\ref{fig:spi-tx}~att.):
		\begin{enumerate}[\hspace*{\parindent}1)]
			\item izpildot \mnem{ST} instrukciju, MMU paceļ |INIT|
				signalizējot, ka jānolasa jaunais pārraidāmais vārds, kurš
				tiek paralēlā režīmā ieraktīts bīdes reģistros;
			\item |INIT| kļūstot zemam, bīdes reģistri tiek pārslēgti
				seriālā režīmā un tiek uzsākta datu pārraide;
			\item pie lecošās takts signāla frontes gan \termEn{master},
				gan \termEn{slave} puses ierīce nolasa datu bitu;
			\item pie krītošās takts signāla frontes uz datu apmaiņas
				signālvadiem tiek uzlikts nākamais bits;
			\item tiek atkārtota 3.~un 4.~darbība pārējiem datu bitiem;
			\item apmainot pēdējo datu bitu, tiek pacelts |READY|
				signalizējot, ka datu vārda apmaiņa ir pabeigta, un no
				\termEn{slave} ierīces
				saņemtais vārds ir nolasāms no |PDO| pieslēgvietas;
		\end{enumerate}
		
		\todo ? % SPI mode 0
		
		SPI saskarne parauga mikrokonrolierī izmantota
		komunikācijai ar SPI \termEn{Flash} atmiņu,
		kura pieejama uz izstrādes platformas \cite[43.~lpp.]{FusionGuide},
		kur tiek glabāta galvenā izpildāmā programma, kura sāknēšanas procesa
		laikā tiks pārvietota uz RAM (sk.~\ref{sec:rom}~nod.~un pielikumu \ref{appx:boot}).
