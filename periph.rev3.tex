%\subsection{Papildinātā shēma (rev.~03)}
%Šī apakšnodaļa apraksta sistēmas papildinājumus funkcionālas, sintezējamas
%sistēmas realizācijai. Šīs revīzijas implementācija \textbf{nav pabeigta},
%tādēļ šī apakšnodaļa ir uzskatāma par \textbf{turpmākā darba dokumentāciju},
%nevis realizētu implementāciju.

%Trešā sistēmas revīzija pievieno patstāvīgās atmiņas elementus sākotnējās
%programmas un datu uzglabāšanai. Šīs atmiņas ierīces piekārtotas kopējai
%adresācijas telpai, tādējādi rodas nepieciešamība pēc adreses dekodēšanas
%loģikas, kuru nodrošina MMU (\termEn{Memory map unit}).



\subsection{Boot ROM}
	\termEn{Boot ROM} ir tikai nolasāmā atmiņa, kura paredzēta
	\termEn{bootstrap} procesa programmas uzglabāšanai.
	Šīs programmas uzdevums ir ielādēt reālo izpildes programmu no
	patstāvīgās atmiņas — šajā implementācijā konkrēti 
	no SPI \termEn{Flash} atmiņas.
	
	Par \termEn{Boot ROM} kalpos \termEn{Actel Fusion} piedāvātais
	\termEn{FlashROM} makross.\citeet{FlashROM}
	Maksimāli pieejamais ROM garums uz izmantojamās platformas ir
	1 kilobits jeb 64 vārdi.\citeet{FusionGuide}

\subsection{Adrešu telpai piekārtotā ievade/izvade}
	Papildus adresācijas telpā tiks piekārtoti speciālie reģistri dažādu
	perifēro ierīču datu apmaiņai un konfigurācijai.
	
	Uz doto brīdi (un kā redzams blokshēmā — \ref{fig:top-rev3}.~att.)
	paredzēta tikai SPI saskarne. Tam paredzēts viens pārraides un viens
	konfigurācijas reģistrs. Tam paredzēta iespēja nokonfigurēt datu
	apmaiņu pa baitam (8 biti) vai pa vārdam (16 biti). Datu pārraide tiek
	uzsākta uzreiz pēc jauno datu ierakstes reģistrā (ja SPI iespējots) un
	tiek pārraidīti bez procesora līdzdalības.
	
	\begin{figure}[thp]
		\centering
		%\def\svgwidth{7cm}
		\def\svgscale{1.25}
		{\ttfamily\scriptsize\input{img/sub-spi.pdf_tex}}
		\caption{SPI saskarnes ierīce.}
		\label{fig:spi}
	\end{figure}
	
	SPI saskarne nepieciešama komunikācijai ar SPI \termEn{Flash} atmiņu,
	kura pieejama uz izstrādes platformas,\cite[43.~lpp.]{FusionGuide}
	kur tiks glabāta galvenā izpildāmā programma, kura \termEn{boot} procesa
	laikā tiks pārvietota uz RAM.

\subsection{MMU — adrešu dekoderis} \label{sec:mmu}
	MMU jeb adrešu dekoderis ir ierīce, kas atbild par adrešu telpas
	pārdalīšanu dažādām ierīcēm, kā arī nodrošināt korektu komunikāciju
	ar šīm ierīcēm.
	
	\begin{figure}[thp]
		\centering
		\def\svgwidth{0.75\textwidth}
		{\ttfamily\small\input{img/remap.pdf_tex}}
		\caption{Adrešu telpas sadalījums.}
		\label{fig:memory-map}
	\end{figure}
	
	MMU tādējādi ar procesoru veido tādu pašu saskarni kā RAM un aparatūras
	līmenī ir pilnībā „caursīdīga”, savukārt piekārtojamo ierīču saskarne
	ir implementācijas definēta.
	
	\termEn{Boot ROM} nepieciešams piekārtot ar sākumu adresē \texttt{0x0000},
	lai pēc atiestatīšanas procesors sāktu izpildīt \termEn{boot} programmu.
