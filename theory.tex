\section{Mikrokontrolieri} %TODO: Normālu nosaukumu?
\todo

%Dzirdot vārdu ,,mikroprocesors'' un ,,datorsistēma'' vairums cilvēku 
%iztēlojas personālos datorus un klēpjdatorus, kas arī pēdējos gados ir teju
%vai katram. Tā gan ir tikai --- tēlaini izsakoties --- aisberga redzamā
%daļa. Mūsdienās mikroprocesori atrodami ļoti plaša klāsta dažādās sadzīves
%un industriālās ierīcēs, piem.~automašīnu elektronikā, ražošanas iekārtās,
%datortīklu maršutētājos, elektromēriekārtās, mobilajos telefonos,
%veļas mašīnu kontroles blokos un pat rotaļlietās.\cite[1.~lpp.]{Heath}

%% TODO: Šito mošk vajag pārfrāzēt
%Iekļautā sistēma arī ir datorsistēma, kuras primārā atšķirība no 
%vispārēja pielietojuma datorsistēmas ir darba specializācija,
%no kā seko praktiski visas realizācijas atšķirības. \todo

%TODO?: Procesors

\section{Aparatūras apraksta valodas}
Elekronikai attīstoties un paaugstinoties tirgus prasībām, izstrādājamās
sistēmas kļūst aizvien sarežģītākas. Jo sevišķi komplicētu ciparu shēmu, 
kā piemēram mikroprocesoru, iekšējo uzbūvi izstrādāt tradicionālā ceļā,
shematiski attēlojot komponentes un to savstarpējos savienojumus, kļūst
nepraktiski. Tieši šeit aparatūras apraksta valodas parāda savu efektifitāti.
% FIXME: Pārfrāzēt šito rindkopu

Aparatūras apraksta valoda jeb HDL
(no angļu \termEn{Hardware description language}),
ir valoda, ar kuras izteiksmēm, sekojot konkrētās 
valodas sintaktiskajiem nosa\-cī\-jumiem, ir iespējams aprakstīt
izstrādājamās shēmas uzbūvi un/vai darbību.\cite{HDL} %Šo aprakstu dēvē par ,,kodu''.
HDL gandrīz vienmēr (bet ne obligāti) apraksta ciparu shēmas.
Līdzīgi program\-mē\-šanas valodām, HDL piedāvā dažādas sintaktiskās 
konstrukcijas, kas ļauj strukturēt aprakstu un veikt abstrakcijas,
ar kurām iespējams kompakti aprakstīt sarežģītas shēmas.%
\cite[1.~lpp.]{Perry-VHDL}
Savukārt atšķirībā no (vairuma) programmēšanas valodu,
izpildāmās izteiksmes ir ,,konkurentas'', t.i.~tās tiek izpildītas paralēli, 
tādējādi shēmas definējošā apgabala izteiksmju secība ir maznozīmīga.%
\footnote{Izteiksmju secībai ir nozīme secīgajās konstrukcijās.
	(sk.~\ref{sec:hdl-styles}~nod.)}

Ļoti būtiska aparatūras apraksta valodu priekšrocība ir darbības simulācija.
Tā ir, ar simulatora palīdzību, programmatūras līmenī 
atveidota aprakstītās shēmas darbība, no kuras iegūstamas jebkuru
shēmas signālvadu laika oscilogrammas, atmiņas elementu saturs un to izmaiņas,
kā arī iespējams citi dati. Analizējot simulācijas rezultātus ir 
iespējams pārbaudīt aprakstītās shēmas darbības korektumu pirms
tās fiziskās realizācijas.

Bet galvenais iemesls HDL popularitātei un izmantošanai ir iespējai veikt 
shēmu sintēzi --- automatizētu fiziskās realizācijas ģenerēšanu pēc HDL apraksta.%
\cite{HDL}\cite{Perry-VHDL}\cite{Vahid-RTL}
Sintēzes nianses un izmantotās tehnoloģijas 
tiks sīkāk apskatīts \ref{sec:synth} nodaļā (\pageref{sec:synth}~lpp.) pēc
HDL valodu detalizētāka apraksta. \pagebreak[1]

\subsection{Aprakstu stili} \label{sec:hdl-styles}
HDL vienādas komponentes var aprakstīt dažādos veidos, bet
HDL aprakstus var sadalīt divos pieraksta stilos, pēc problēmas pieejas un
no valodas konstrukciju kopas kas šai pieejai raksturīga.
\begin{enumerate}
	\item \textbf{Strukturālais} pieraksta stils --- komponentes tiek
		izveidotas izmantojot vienkāršākas apakškomponentes un to savstarpējos
		savienojumus. Šis stils uzskatāms par tekstuālu analogu
		tradicionālajam, shematiskajam izstrādes veidam.
	\item \textbf{Funkcionālās} vai izturēšanās modelēšanas pieraksta
		stils --- komponentes tiek izveidotas aprakstot tās funkcionalitāti
		abstrahējoties no tās iespējamās uzbūves.
	\begin{itemize}
		\item \textbf{Secīgās izturēšanās} pieraksta stils --- apakškopa
			no funkcionālā stila, kurā izmantotas konstrukcijas, kas ļauj
			komponentes darbību aprakstīt ar secīgām izpildāmām izteiksmēm,
			līdzīgi imperatīvām programmēšanas valodām (kā C, Python, u.c.),
			pretstatā paralēli izpildāmām datu plūsmas izteiksmēm.
	\end{itemize}
\end{enumerate}

Šie stili ne tikai ietekmē koda (konkrētā shēmas apraksta teksta) pierakstu,
bet arī simulācijas un sintēzes procesus, kuriem šis pieraksts ir jāinterpretē.
Strukturālā pieraksta un tādu funkcionālā pieraksta primitīvu konstrukciju,
kā loģisko elementu (\texttt{UN}, \texttt{VAI}, utt.),
sintēze ir tieši translējama uz aparatūras komponentēm un to savienojumiem.
Savukārt sarežģītu funkcionālo konstrukciju, jo sevišķi secīgo
konstrukciju, sintēzei nepieciešams translēt funkcionalitātes aprakstu
uz konkrētu aparatūras implementāciju, t.i.~sintēzes rīkam nepieciešams
piemeklēt aparatūras komponentes kas realizē konkrēto funkcionalitāti.
Tā kā funkcionalitātes un implementācijas saistība bieži nav viennozīmīga,
dažādu sintēzes rīku interpretācija vienādam kodam var stipri atšķirties.

Praktiski visi sintēzes rīki funkcionālo pierakstu translē %uz aparatūras 
reģistru datu pārraides abstrakcijas līmenī jeb RTL
(no angļu \termEn{Register tranfer level}).\cite[2.~lpp.]{HDL}%
\cite[235.~lpp.]{Perry-VHDL}
RTL \todo


Citāda aina ir ar apraksta simulāciju. Tā kā simulācija notiek
programmatūras līmenī, vairumā gadījumu secīgās izturēšanās modeli
simulatoram ir vienkāršāk (un tādējādi arī ātrāk) simulēt nekā paralēli
izpildāmās konkurentās konstrukcijas.

 %\pagebreak[3]
\subsection{FPGA un ASIC tehnoloģijas}
\todo


%\subsection{Shēmu sintēze}
%\todo
		
%\subsubsection{FPGA, ASIC un to nozīme sintēzē}
%\todo

%TODO: HDL un FPGA


%TODO: SPI
