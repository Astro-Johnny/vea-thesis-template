\documentclass[12pt,a4paper]{article}
\usepackage{polyglossia}	% XeTeX multi-lingual support
\usepackage{fontspec}		% The XeTeX font spec package
\usepackage{xunicode}		% XeTeX Unicode character support
\usepackage{xltxtra}		% Umm something else XeTeX

\usepackage[landscape]{geometry}	% Make document landscape
\usepackage{fullpage}


%\usepackage{amsmath}	% Math stuff
%\usepackage{pifont}		% dingbats
%\usepackage{listings}	% Package for code formatting
%\usepackage[dvipsnames]{xcolor}
%\include{/home/johnlm/code/avr_assembler_listing_def.tex}
%\input{avr_assembler_listing_def.tex}
%\renewcommand{\lstlistingname}{Koda bloks}
%\usepackage{booktabs}	% Package for nicer tables
%\usepackage{wrapfig}
%\usepackage{rotating}	% Package for landscape figures

\begin{document}
	\begin{flushright}
		\fbox{\rule{1.8in}{0pt}\rule{0pt}{0.9in}}
	\end{flushright}

	\subsection*{Prakses dienasgrāmata}
	\begin{flushright}
		\rule[-1em]{15em}{1pt}\\
		\makebox[15em][c]{\scriptsize (studenta(-es) vārds, uzvārds)}
	\end{flushright}
	
	\centering
	\begin{tabular}{|c|c|c|c|}
		\hline
		\textbf{Datumi} & \textbf{Veicamais darbs} & 
			Apjoms (stundās) & Vadītāja paraksts\\
		(no\ldots līdz) & \rule{25em}{0pt} & & \\ \hline
		13.02 – 15.02 & Mikrokontroliera kodolā (rev.~02) izmantojamo komponenšu izstrāde VHDL & 5 & \\ \hline
		15.02 – 21.02 & Kodola (rev.~02) kontroles iekārtas izstrāde VHDL & 10 & \\ \hline
		29.02 & Revīzijas 02 kodola simulācija un darbības pārbaude & 2 & \\ \hline
		29.02 – 01.03 & Kodola VHDL koda izpēte un korekcijas darbības kļūdu novēršanai & 3 & \\ \hline
		02.03 & Revīzijas 02 kodola simulācija un darbības pārbaude & 2 & \\ \hline
		12.04 – 13.04 & Mikrokontroliera (rev.~03) ROM un RAM komponenšu izstrāde & 5 & \\ \hline
		23.04 – 25.04 & Mikrokontroliera (rev.~03) SPI saskarnes izstrāde & 5 & \\ \hline
	\end{tabular}
	
	
\end{document}
