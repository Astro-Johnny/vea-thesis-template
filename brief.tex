\section{BRIEF deskriptors} \label{sec:brief}
BRIEF ir $S \times S$ attēla apgabala deskriptors, kas satur informāciju par
apgabalu, un ir salīdzināma ar citu deskriptoru, lai noteiktu to līdzību.
Neapstrādāta pikseļu informācija nav tieši izmantojama, jo to salīdzinot
netiek nodrošināta dažādu attēla transformāciju invariance
(sk.~\ref{sec:matching}~nod.), tādēļ nepieciešams pikseļu informāciju
pārveidot.

Būtisks BRIEF uzlabojums pār SIFT deskriptoru ir tā informācijas blīvums.
BRIEF deskriptors, ar līdzvērtīgiem salāgojamības rādītājiem, uzglabājams
256~bitos atmiņas, turpretī SIFT deskriptoram nepieciešami 4096~biti
(128 peldošo komata skaitļu vektors)~\cite{BRIEF}. Papildus ieguvums ir,
ka BRIEF deskriptori ir salīdzināmi izmantojot Hemminga attālumu
(\termEn{Hamming distance}), pretstatā $L^2$ normai SIFT deskriptoriem.
Hemminga attālums ir algoritmiski vienkārši realizējams, sekmējot ātrdarbību,
it sevišķi CPU platformām, kuras aparatūras līmenī realizē
bitu skaitīšanas (\termEn{``popcount''}) instrukciju.
%TODO: Forward ref

\subsection{Definīcija} \label{sec:brief-def}
BRIEF deskriptors balstās attēla apgabala pikseļu pāru salīdzināšanu.
Veicot $n_d$ skaitu dažādu pikseļu pāru salīdzināšanu, iegūstam
bināro vektoru ar $n_d$ bitu garumu, kas arī ir deskriptors.
\cite{BRIEF}\cite{ORB}

Formāli, uzdodam apskatāmo $S \times S$ izmēra attēla gabalu par
$\vb{p}$, kas ir attēla $\vb{I}$ apakškopa:
$\vb{p} \subseteq \vb{I}$. Lai nodrošinātu lielāku salāgojamību nepieciešams
veikt attēla filtrēšanu, kam BRIEF izmanto Gausa (\textit{Gauß}) filtrēšanu
\cite{BRIEF}, definējot rezultējošo attēla gabalu kā:
\[
	\hat{\vb{p}} := \vb{p} \ast G(\sigma)
\]
kur $G(\sigma)$ ir diskretizēta Gausa matrica
(pie noteiktas $\sigma$ standartnovirzes).\\
Kalanders~u.c.\cite{BRIEF}~(\termEn{Calonder~et~al.})
rekomendē filtrēšanu veikt ar $\sigma \in [0; 3]$ un šajā darbā izmantots
$\sigma = 2$ Gausa filtrs.

Pikseļu intensitātes salīdzināšanu attēla gabalā $\hat{\vb{p}}$ definējam kā funkciju:
\begin{equation}
	\tau (\hat{\vb{p}}, \vb{a}, \vb{b}) := 
		\begin{cases}
			1\text{,} & \text{ja } \hat{\vb{p}}(\vb{a}) > \hat{\vb{p}}(\vb{b}) \\
			0\text{,} & \text{citos gadījumos}
		\end{cases}
\end{equation}
kur $\vb{a}$ un $\vb{b}$ ir punktu koordinātes
$\left(\begin{smallmatrix}x\\y\end{smallmatrix}\right)$
attēla gabalā $\hat{\vb{p}}$, starp kuriem tiek
veikta salīdzināšana~\cite{BRIEF}. 

Lai definētu deskriptoru ir nepieciešams uzdot $n_d$ skaitu punktu pārus,
starp kuriem tiks veikta salīdzināšana un
kurus apzīmēsim ar atsevišķiem vektoriem
$\vb{a}_1 \dots \vb{a}_{n_d}$ un $\vb{b}_1 \dots \vb{b}_{n_d}$.
Pašu deskriptoru, ko apzīmēsim ar $B_{n_d}$, definē kā~\cite{BRIEF}:
\begin{equation}
	B_{n_d}(\hat{\vb{p}}) := 
		\sum_{i=1}^{n_d} 2^{i-1} \tau\left(\hat{\vb{p}}, \vb{a}_i, \vb{b}_i\right)
\end{equation}

Konkrēto punktu izvēle būtiski ietekmē
deskriptoru diskriminitāti un salāgojamību. Kalanders~u.c.\cite{BRIEF}
ekperimentāli iegūst, ka koordinātu izvēle izmantojot gadījumskaitļus ar
isotropisku Gausa sadalījumu (par koordinātu sākumpunktu pieņemot
$\vb{p}$ centru) ar varianci $\sigma^2 = \frac{1}{25} S^2$, kā ilustrēts
\ref{fig:pattern1}~attēlā,
\begin{figure}[tbh]
	\centering
	\includegraphics[width=0.25\linewidth]{brief-pattern}
	\caption{Salīdzināšanas pāri (parādīti 128 pāri)~\cite{BRIEF}.}
	\label{fig:pattern1}
\end{figure}
dod labus rezultātus.

BRIEF deskriptora aprēķināšanai, salīdzināmu deskriptoru kopas ietvaros
izmanto tos pašus salīdzināšanas pārus (matricas $\vb{a}$ un $\vb{b}$).
Tā kā šo pāru koordinātu punkti ir piesaistīti $\vb{p}$, un līdz ar to arī
attēla orientācijai, var viegli secināt, ka BRIEF nav rotācijas invariants.
%~ Šo problēmu risina rBRIEF, kas apskatīts sekojošā apakšnodaļā.
Šo probēmu risina Rublē~u.c.\cite{ORB} izstrādātā modifikācija rBRIEF.
Pie tam, lai kompensētu variances samazinājumu rBRIEF
deskriptoram, Rublē~u.c.\cite{ORB} izmanto mašīnmācīšanās metodes, atlasot
salīdzināšanas pārus ar lielāko vērtības varianci un zemāko savstarpējo
korelāciju. rBRIEF plašāk apskatīts \ref{sec:rbrief-def}~nodaļā.

\subsection{BRIEF FPGA implementācijas modelis} \label{sec:brief-fpga}
Principā,
BRIEF deskriptora noteikšana ir īpaši piemērota implementēšanai FPGA, jo
tiek izmantota vienkārša salīdzināšana un 
FPGA atsevišķu, konkrētu, iepriekš definētu elementu izgūšana
no masīva ir ļoti efektīva, jo tiek realizēts kā vienkāršs loģisko elementu 
savienojums (vads).
\begin{figure}[tbh]
	\centering
	%\def\svgwidth{\linewidth}
	\def\svgscale{1.1}
	{\small\input{img/brief-fpga.pdf_tex}}
	\caption{Vienkāršota uzbūves shēma BRIEF aprēķina vienībai.}
	\label{fig:brief-fpga}
\end{figure}
Pēc modeļa, kas parādīts \ref{fig:brief-fpga}~attēlā,
izmantojot $n_d$ salīdzinātājus, var noteikt BRIEF deskriptoru
vienā takts ciklā, pieņemot ka visu punktu vērtības no attēla gabala ir 
izgūstamas vienlaikus.

Tomēr, lielākais veiktspējas ierobežojums ir $\hat{\vb{p}}$ iegūšana
filtrējot attēla gabalu $\vb{p}$ ar Gausa filtru. Attēla gabala filtrēšanai
nepieciešams realizēt konvolūciju, kas prasa lielu skaitu reizināšanas un
saskaitīšanas operācijas. Pie tam Gausa matricas $G$ elementi ir skaitļi
ar decimāldaļu, nevis veseli (\termEn{integer}) skaitļi, kam nepieciešams
izmantot peldošā vai fiksētā komata skaitļu formātu.

\phantomsection\label{C:selective-gauss}
Zināma resursu ekonomiju dod tas, ka ne visi attēla gabala $\hat{\vb{p}}$
punkti tiek izmantoti BRIEF noteikšanai, tādēļ var implementēt daļēju Gausa
filtru, aprēķinot tikai nepieciešamo $2n_d$ skaita punktu vērtības.
Bet, neskatoties uz to,
pieņemot, ka $n_d = 256$ un izmantotā matrica $G$ ir $5 \times 5$ pikseļu
liela, nepieciešamo reizināšanas un saskaitīšanas operāciju skaits ir
$2\cdot 256 \cdot 5^2 = 12800$.

Autors secina, ka iepriekšminētā modeļa nodrošināšana ar datiem
($\hat{\vb{p}}$) pietiekamā ātrumā (vienā takts ciklā) ir tuvu neiespējama,
vai vismaz ļoti nepraktiska. Tādēļ, tiek piedāvāts alternatīvs modelis,
izdalot atsevišķas apstrādes vienības pikseļu pāra salīdzināšanai, iekļaujot
tajā arī Gausa filtrēšanu, kā ilustrēts \ref{fig:gauss+brief}~attēlā, kur
ar $w_G$ apzīmē Gausa matricas platumu un garumu.
\begin{figure}[tbh]
	\centering
	%\def\svgwidth{\linewidth}
	\def\svgscale{1.3}
	{\small\input{img/gauss+brief.pdf_tex}}
	\caption{Uzbūves shēma pikseļu pāra salīdzināšanas vienībai ar Gausa filtru.}
	\label{fig:gauss+brief}
\end{figure}



