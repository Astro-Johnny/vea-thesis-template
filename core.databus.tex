\subsection{Iekšējās datu šinas uzbūve} \label{sec:databus}
Izveidotajam procesoram ir vienvirziena, sazarota, 
cirkulāra datu apmaiņas šina.
Šādai šinas uzbūvei ir vairākas priekšrocības:
\begin{itemize}
	\item \textbf{Nav trīsstāvokļu loģika}, tādējādi zūd nepieciešamība
		izmantot trīsstāvokļu komponentes procesora iekšējā uzbūvē,
		kas šajā gadījumā samazina realizācijas kompleksitāti.
	\item Tā kā šinā esošo komponentēm izejas savstarpēji nevar tikt
		savienotas, ir nodrošināta zināma \textbf{elektriskā drošība},
		likvidējot iespējamo kaitējumu arbitācijas kļūdu rezultātā.
	\item Zūd nepieciešamība pēc trīsstāvokļu komponenšu iespējošanas un
		datu apmaiņas virziena signāliem. Tātad kontroles iekārtai ir
		\textbf{mazāk vadības signālu}, jo vienvirziena datu
		apmaiņai nepieciešams pievadīt tikai (pār)rakstīšanas vadības signālus%
		\footnote{Lai uzsvērtu šo vadības signālu nozīmi reģistru satura
			atjaunošanā, to nosaukumiem VHDL kodā pievienots -\texttt{Updt} piedēklis.}%
		, vai pat nevienu signālu kombinacionālo komponenšu gadījumā.
	\item Iespējama datu apmaiņas \textbf{paralelitāte}, jo dati dažādos
		šinas posmos ir pārraidāmi vienlaicīgi. Šī īpašība izmantota
		\mnem{BRxx} instrukciju izpildē 
		(sk.~\ref{sec:branching}~nod.~un kodu
		pielikumā \ref{appx:control}).
\end{itemize}

Viens no šādas šinas uzbūves trūkumiem var tikt uzskatīta nepieciešamība
multipleksēt signālus, bet pēc visu iespējamo datu apmaiņas scenāriju analīzes
implementējamās instrukciju kopas ietvaros tika noskaidrots, ka pat ne tuvu 
katrai komponentei nepieciešama datu apmaiņa ar visām pārējām.
Konkrēti, tikai piecām komponentēm nepieciešama datu nolase no vairāk
nekā vienas citas. Šīs komponentes uzrādītas \ref{tbl:muxes}~tabulā.
\begin{table}[thb]
	\centering
	\caption{Komponentes kurām nepieciešama ieejas multipleksēšana.}
	\label{tbl:muxes}
	\begin{tabular}{ll}
		\toprule
		Komponente & Ienākošie (multipleksējamie) dati\\ 
		\midrule
		Programmskaitītājs & Atmiņas ielase; ALU signālceļa dati\\
		Reģistru masīvs & Atmiņas ielase; ALU signālceļa dati\\
		ALU (pirmais operands) & Programmskaitītājs; Reģistru masīvs\\
		Komparators (pirmais operands) & Programmskaitītājs; Reģistru masīvs\\
		Adresācijas reģistrs & ALU signālceļa dati; Reģistru masīvs\\
		\bottomrule
	\end{tabular}
\end{table}
Apvienojot vienādos multipleksorus, iegūst optimizētu datu apmaiņas šinu
ar tikai trijiem \texttt{32>16} multipleksoriem (sk.~\ref{fig:cpu-rev3}~att.)
un, pēc autora domām, kontroles iekārtas vadības signālu samazināšana vairākkārt
atsver multipleksoru pievienoto kompleksitāti.

Otrs, iespējams trūkums šādai datu šinai ir tās nepārskatāmība shematiskajā
zīmējumā (sk.~\ref{fig:cpu-rev3}~att.), bet šī ir tikai vizualizācijas
problēma un neatstāj nekādu iespaidu uz realizāciju.


%Galvenais šāda risinājuma iemesls
%ir izmantojamās \termEn{Actel} FPGA darba platformas trīs stāvokļu
%loģikas atbalsta \mbox{trūkums.\cite[18.~lpp.]{FusionFAQ}}
