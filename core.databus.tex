\subsection{Iekšējās datu šinas uzbūve} \label{sec:databus}
Izveidotajam procesoram ir sazarota, cirkulāra datu apmaiņas šina atšķirībā no kopējas,
trīsstāvokļu arbitējamas datu šinas. Šādai šinas uzbūvei ir vairākas
priekšrocības:
\begin{description}
	\item[Nav trīsstāvokļu loģika] \hfill \\
		Trīsstāvokļu loģikas trūkums atbrīvo no nepieciešamības izmantot
		trīsstāvokļu komponentes procesora iekšējā uzbūvē,
		samazinot realizācijas kompleksitāti.
	\item[Elektriskā drošība] \hfill \\ %TODO: Šitam citādāku nosaukumu?
		Nevienai no komponentēm pie šinas nav savienotas izejas,
		likvidējot iespējamo kaitējumu arbitācijas kļūdu rezultātā.
	\item[Mazāk vadības signālu] \hfill \\
		Šādai implementācijai nav nepieciešami nolases un rakstīšanas
		vadības signālu pāri katrai pievienotai ierīcei. Vienvirziena datu
		apmaiņa ļauj tikai pievadīt rakstīšanas vadības signālu%
		\footnote{Lai uzsvērtu šo vadības signālu nozīmi reģistru satura
			atjaunošanā, to nosaukumiem\\ pievienots -\texttt{Updt} piedēklis.}%
		, vai pat nevienu signālu kombinacionālo komponenšu gadījumā.
\end{description}

Viens no šādas šinas uzbūves trūkumiem var tikt uzskatīta nepieciešamība
multipleksēt signālus, bet pēc visu iespējamo datu apmaiņas scenāriju analīzes,
implementējamās instrukciju kopas ietvaros, tika noskaidrots ka --- 
pat ne tuvu --- katrai komponentei nepieciešama datu apmaiņa ar visām pārējām.
Konkrēti, tikai piecām komponentēm nepieciešama datu nolase no vairāk
nekā vienas citas.

%\todo % TODO: Tabula ar nepieciešamajiem savienojumiem
\begin{table}[thb]
	\centering
	\caption{Komponentes kurām nepieciešama ieejas multipleksēšana.}
	\label{tbl:muxes}
	\begin{tabular}{ll}
		\toprule
		Komponente & Ienākošie (multipleksējamie) dati\\ 
		\midrule
		Programmskaitītājs & Atmiņas ielase; ALU signālceļa dati\\
		Reģistru masīvs & Atmiņas ielase; ALU signālceļa dati\\
		ALU (pirmais operands) & Programmskaitītājs; Reģistru masīvs\\
		Komparators (pirmais operands) & Programmskaitītājs; Reģistru masīvs\\
		Adresācijas reģistrs & ALU signālceļa dati; Reģistru masīvs\\
		\bottomrule
	\end{tabular}
\end{table}

Apvienojot vienādos multipleksorus iegūstam optimizētu datu apmaiņas šinu
ar tikai trijiem \texttt{32>16} multipleksoriem. (sk.~\ref{fig:cpu-rev3}~att.)

%Galvenais šāda risinājuma iemesls
%ir izmantojamās \termEn{Actel} FPGA darba platformas trīs stāvokļu
%loģikas atbalsta \mbox{trūkums.\cite[18.~lpp.]{FusionFAQ}}
