\section{Skaitļošanas platformas un resursi} \label{sec:proc}
Dažādu video apstrādes algoritmu izpildei ir nepieciešama
augsta skaitļošanas jauda, jo sevišķi apstrādei reālā laikā, kas pieprasa
lielu datu apjoma apstrādi ļoti ierobežotā laikā. Tipiski, video dati
tiek uzņemti ar frekvenci līdz 30~kadriem sekundē, kas, tādējādi, atvēl ap
33~milisekunžu laika apstrādei derīgās informācijas izgūšanai.

Atšķirībā no citādiem video apstrādes uzdevumiem, mašīnredzes uzdevumi 
visbiežāk jāveic ar dažādām iekļautajām sistēmām (\termEn{embedded systems}),
kurām mobilitātes, gabarītu, cenas vai citu apsvērumu dēļ ir ierobežoti
skaitļošanas resursi un kuru aparatūra var būt vai nebūt specializēta
konkrētajam uzdevumam.

Turpmākajās apakšnodaļās apskatīti pieejamo resursu veidi, to arhitektūras
īpatnības un šo īpatnību veiktspējas ietekme un ierobežojumi
algoritmu implementācijām.

\subsection{Centrālais procesors} \label{sec:cpu}
Centrālais procesors jeb CPU (no angļu \termEn{central processing unit}), ir
praktiski visās datorsistēmās --- gan personālajos datoros (PC), gan serveros,
gan iekļautajās sistēmās. CPU kalpo par galveno vadības komponenti un, vairumā
gadījumu, arī kā galvenais skaitļošanas resurss.

CPU arhitektūra ir vēsturiski attīstīta kopš 20.~gadsimta vidus~%
\cite{Flynn-arch}\cite{von-Neumann}.
Klasiskas arhitektūras CPU raksturo ātra secīgu instrukciju izpilde, 
bet kam nepiemīt ne datu, ne uzdevumu paralelitāte~\cite{Owens-GPU}.
Šādi CPU aritmētiskās instrukcijas izpilda ar vienu datu vienību --- 
operandu vai operandu pāri (piem.,~divu skaitļu saskaitīšanu)~\cite{Flynn-arch},
kā tas vienkāršoti parādīts \ref{fig:cpu-arch}~attēlā.
Līdz 21.~gadsimtam veiktspējas palielināšanai pamatā bija 
instrukcijas izpildes laika saīsināšana vienkārši 
palielinot takts signāla frekvenci
un ieviešot izpildes signāltraktu ar ,,konveijera principu''(\termEn{pipelining})~\cite{Flynn-arch}.
20.~gadsimta 90-o~gadu beigās plaši pieejamo CPU arhitektūrā tika ieviestas
vektoru jeb SIMD (\termEn{single instruction, multiple data}) 
instrukcijas, kas ieviesa zināmu datu paralelitāti jo ar šo instrukciju
palīdzību, varēja izdarīt darbības ar vairāk datu vienībām (skaitļu vektoru)
vienlaikus~\cite{SIMD}.
Savukārt, sākoties 21.~gadsimtam, \newTerm{vairāku kodolu} procesori
sāka kļūt komerciāli pieejami, kas nodrošināja arī uzdevumu paralelitāti.

\begin{figure}[tbh]
	\centering
	\def\svgscale{1.2}
	{\input{img/CPU-arch.pdf_tex}}
	\caption{Skaitļošanas resursi CPU arhitektūrā.}
	\label{fig:cpu-arch}
\end{figure}

\phantomsection\label{sec:cache}
Palielinot CPU instrukciju izpildes ātrumu par
,,vājāko ķēdes posmu'' kļuva datu atgūšana no operatīvās atmiņas (RAM) jeb
tās latentums.
Šo problēmu jau 60-ajos gados risināja radot
\termTech{kešatmiņu} (angļu \termEn{cache})~\cite[473.~lpp.]{Patterson},
kuras pamatprincips ir mazākas ietilpības, bet zemāka latentuma (ātrākas)
atmiņas izmantošana, lai uzglabātu datu apakškopas kopiju no
RAM ar kuru, potenciāli, tūlītēji tiks veiktas darbības.
\cite{Flynn-arch}\cite{Patterson2}\cite{Patterson}\cite{Cache}

\termTech{Kešatmiņa} būtiski uzlabo CPU aprīkotas sistēmas vispārējo
ātrdarbību, bet tās trūkumi sistēmas paralelitātē kļuva acīmredzami
izstrādājot vairāku kodolu CPU~\cite{Fatahalian}\cite{Owens-GPU}\cite{Cache}.
Vairāku procesoru vai vairāku procesora kodolu%
\footnote{Turpmāk nodaļas tekstā minēti kā atsevišķi procesori.}
sistēmas nodrošina
\termTech{kešatmiņas} \newTerm{koherenci}
(angļu \termEn{cache coherence}), t.i.,~visām vienas datu vienības
kopijām \termTech{kešatmiņā(s)} pēc izmaiņas vienādi jāatspoguļo tās
jaunākā vērtība, kā arī jānodrošina, ka šo vērtību vienlaicīgi 
izmainīt drīkst tikai viens no procesoriem.
\begin{figure}[tbh]
	\centering
	\def\svgscale{1.2}
	{\input{img/snoop-cache-bottleneck.pdf_tex}}
	\caption{\termTech{Kešatmiņas} koherence vairāku procesoru sistēmā.}
	\label{fig:snoop-bottleneck}
\end{figure}
Šādā sistēmā, koherences nodrošināšanai, nepieciešama papildus loģika.
Pie tam, tipiskā implementācijā, šī komunikācija
jānodrošina procesoram ,,katram ar katru''.
Tas nozīmē, ka pie $N$ skaita
procesoru nepieciešama komunikācija $\frac{N^2-N}{2}$ skaitam procesoru pāru
(sk.~\ref{fig:snoop-bottleneck}~att.), tādējādi koherences nodrošināšana ir
kavējošais faktors liela skaita procesoru sistēmās.
Šādu koherences nodrošināšanas protokola modeli dēvē ,,okšķērējošo''
(\termEn{snooping}) modeli, kur izmaiņas par datu vienības vērtības izmaiņu
tiek apraidītas visu procesoru \termTech{kešatmiņas}.
\cite{Cache}

Eksistē arī alternatīvs modelis --- direktorija (\termEn{directory}) modelis,
kur centrāli tiek \termTech{kešatmiņas} datu vienībām piedēvēts ,,īpašnieks''
un komunikācija notiek tikai starp šo ,,īpašnieku'' un procesoru, kurš
pieprasa pieeju datu vienībai. Šādi tiek samazināts komunikācijas apjoms pie
liela procesoru skaita, bet modelis ir komplicētāks, kas prasa papildus
atmiņu direktorijam un atmiņas transakcijas izpildes laiks ir garāks.
\cite{Cache}

Savukārt, būtiski citādāka pieeja ir izmantota grafiskā procesora (GPU)
uzbūvē. GPU definē citādu atmiņas izmantošanas
modeli, kas padara \termTech{kešatmiņas} koherenci mazsvarīgu,
atbrīvojot arhitektūras uzbūvi no koherences nodrošināšanas loģikas un
sekmējot paralelitāti. GPU arhitektūra plašāk apskatīta \ref{sec:gpu}~nodaļā.





\subsection{Grafiskais procesors} \label{sec:gpu}
Grafiskais procesors jeb GPU (no angļu \termEn{graphics processing unit})
ir specializēta skaitļošanas iekārta, kura izstrādāta
un attīstīta divdimensiju un trīsdimensiju attēlu atveidošanai un apstrādei
to izvadei uz displeja.
Grafiskos procesorus izvieto:
\begin{itemize}
	\item uz \newTerm{video kartēm} --- kopā ar tam speciāli paredzētu
		atmiņu (VRAM) --- kuras var pieslēgt PC \newTerm{mātes platei};
	\item tieši uz mātes plates, kur GPU var izmantot speciāli paredzētu
		atmiņu un/vai koplietot (ar CPU) datora operatīvo atmiņu (RAM);
	\item iestrādājot vienā mikroshēmā ar CPU (tipiski jaunos klēpjdatoros);
	\item arvien biežāk, iekļautajās sistēmās iestrādājot SoC
		(angļu \termEn{system-on-chip}) mikroshēmās.
\end{itemize}

Lai gan GPU idejiski nav izstrādāts, lai veiktu vispārējus skaitļošanas
uzdevumus, GPU arhitektūras attīstības tendences pavēra šādu iespēju un GPU
kļuva nozīmīga skaitļošanas platforma augstās veiktspējas dēļ, ko,
galvenokārt, nodrošina GPU arhitektūras izteiktā paralelitāte.

Sākotnēji GPU arhitektūras pamatā bija vairāku pakāpju signāltrakts, kur
katra pakāpe veica fiksētu funkciju ar lielu apjomu datu. Katra pakāpe
signāltraktā varēja darboties vienlaicīgi, tādējādi GPU arhitektūrai
piemita gan uzdevumu, gan datu paralelitāte no tās pirmsākumiem.
Programmējamība GPU arhitektūrā parādījās programmējamas \newTerm{ēnotāju}
(\termEn{pixel shader} un \termEn{vertex shader}) pakāpes,
kuras iepriekš arī bija fiksētas funkcijas. Šādam signāltraktam bija būtiska
problēma ar slodzes sadalīšanu, jo slodze dažādās pakāpēs bija atkarīga
no datiem un ēnotāju programmējuma. Šo problēmu risināja izstrādājot
,,vienotu ēnotāju arhitektūru'' (\termEn{unified shader architecture}),
kuras pamatā ir liels skaits programmējami, paralēli
\newTerm{straumes procesori} (\termEn{stream processors}), kuru lomu
signāltraktā, kurš tagad vairs nav fiksēts aparatūras līmeni, var mainīt.
GPU straumes procesori izmanto SIMD instrukcijas darbībām ar skaitļu
vektoriem, kā ilustrēts \ref{fig:gpu-arch}~attēlā. Ņemot vērā lielo skaitu%
\footnote{AMD Radeon HD7990 ir 4096 straumes procesori.
	\url{http://www.amd.com/en-us/products/graphics/desktop/7000/7990}}
šādu SIMD straumes procesoru, GPU var attīstīt ļoti lielu datu caurlaidspēju
(\termEn{throughput}).
\cite{Fatahalian}\cite{Owens-GPU}

Būtiska nianse, kas uzliek ierobežojumus algoritmu implementācijām, ir tas,
ka straumes procesori nav pilnībā neatkarīgi. Tie ir apvienoti grupās, kuras
sauc par ,,straumes multiprocesoriem'', ,,pavedienu procesoriem'' vai
vienkāršāk (bet ne visai korekti) par GPU ,,kodoliem''. Katra šāda grupa
koplieto vienu instrukciju atmiņu, kas nozīmē, visi grupas straumes
procesori izpilda to pašu instrukciju, bet ar dažādiem datiem.
Šo izpildes modeli sauc par SPMD (\termEn{single program, multiple data})
modeli.

Šis izpildes modelis uzliek ierobežojumus uz zarošanos.
Situāciju, ja algoritms izmanto
zarošanos (piem.,~\texttt{if} konstrukciju),
bet dažādiem grupas straumes procesoriem zarošanās nosacījums neizpildās
vienādi, sauc par ,,nekoherentu zarošanos''. Ņemot vērā, ka visai grupai
jāizpilda tās pašas instrukcijas, GPU ir spiests izpildīt
abus (vai visus) instrukciju secības variantus. Pēc abu (visu) variantu
izpildes, korektais reģistru saturs katram straumes procesoram tiek
atjaunots ar datu maskas palīdzību, kas atspoguļo kuru secību
konkrētajam straumes procesoram būtu jāizpilda pēc algoritma.
\cite{Owens-GPU}

\begin{figure}[tbh]
	\centering
	\def\svgscale{1.2}
	{\input{img/GPU-arch.pdf_tex}}
	\caption{Skaitļošanas resursi GPU arhitektūrā.}
	\label{fig:gpu-arch}
\end{figure}

GPU atmiņas izmantošanas modeli arī definē tā pamatuzdevums --- atveidot jeb
\termTech{rasterizēt} attēlus vadoties pēc trīsdimensiju objektu datiem.
Šie dati signāltraktā tiek transformēti un apstrādāti attēla iegūšanai,
bet ieejas dati \termTech{rasterizējot} netiek modificēti
un iegūtais attēls (vairumā gadījumu) pēc tā izvades uz displeja neietekmē
nākamo attēlu, respektīvi, esošie, koplietojamie dati netiek pārrakstīti,
bet tiek radīti jauni dati no tiem. Tas atbrīvo GPU no atmiņas koherences
problēmas, kāda ir CPU arhitektūrā (sk.~\pageref{sec:cache}~lpp.).

Vēl viens būtisks paralelitātes aspekts atmiņas izmantošanas modelī ir
uzsvars uz datu caurlaidspēju, nevis zemu atmiņas latentumu.
Tādēļ kešatmiņas loma GPU arhitektūrā ir novirzīt slodzi no
pamatatmiņas, un pie tam šī kešatmiņa, galvenokārt, ir tikai lasāmā
(\termEn{read-only}) kešatmiņa%
\footnote{GPU arhitektūrā speciālām datu grupām izmanto arī
	rakstāmu (\termEn{read/write}) kešatmiņu~\cite{Owens-GPU}.}
(no izpildes procesoru puses)~\cite{Fatahalian}.
Augsto atmiņas latentumu GPU kompensē
ar \newTerm{vairākpavedienošanu} aparatūras līmenī.
Katra straumes procesoru grupa uzglabā informāciju
par vairākiem%
\footnote{NVIDIA GeForce GTX 280 atbalsta līdz 128 pavedieniem uz katru
	straumes procesoru grupu~\cite{Fatahalian}.}
izpildes pavedieniem (\termEn{threads}), kur aktīvais pavediens, kas izdara
pieprasījumu no atmiņas un ir spiests gaidīt, tiek aizvietots ar
citu pavedienu, kurš ir tūlītēji izpildāms~\cite{Fatahalian}.
Tādējādi tiek samazināts gaidīšanas laiks un tiek efektīvāk
izmantoti skaitļošanas resursi.

\subsection{FPGA} \label{sec:fpga}
FPGA (angļu \termEn{field-programmable gate array}) jeb
pārprogrammējams loģisko elementu masīvs ir integrētā shēma, kura sastāv no
liela skaita programmējamiem loģiskajiem blokiem. Šie bloki ir pārprogrammējami
izmantojot aparatūras apraksta valodas (HDL) un ražotāja programmatūru,
ļaujot lietotājam ar FPGA realizēt vēlamo funkcionalitāti.
FPGA loģisko bloku un starpsavienojumu tīkla (\termEn{the interconnect})
uzbūve, loģisko bloku un citu resursu skaits ir ražotāja un
FPGA modeļa specifiska.\cite{JIS}

FPGA, kā skaitļošanas resurss, ir unikāls ar to, ka tā arhitektūras definēšana
ir ļoti elastīga un, vairumā gadījumu, arhitektūru pielāgo veicamajam
uzdevumam vai algoritmam, nevis otrādi, kā iepriekš apskatītajiem CPU un GPU.

\begin{figure}[tbh]
	\centering
	\def\svgscale{1.2}
	{\input{img/FPGA-arch.pdf_tex}}
	\caption{Fiksēta signāltrakta arhitektūra.}
	\label{fig:fpga-arch}
\end{figure}
Autorprāt, efektīvs un visai intuitīvs arhitektūras uzbūves modelis ir
fiksēta signāltrakta modelis. Kā ilustrēts \ref{fig:fpga-arch}~attēlā, šādā
modelī datu vienība --- tipiski, matrica vai vektors --- tiek virzīta caur
signāltraktam, kur tā tiek transformēta lai iegūtu rezultātu. Atšķirībā no
CPU un GPU, kur skaitļošanas algoritmus realizē ar secīgu instrukciju izpildi
starprezultātus no ALU (aritmētiski loģiskās ierīces)
atgriežot uzglabāšanai koplietojamos reģistros vai RAM,
FPGA, pēc signāltrakta modeļa, RAM vai koplietojamie reģistri nav
nepieciešami, jo starprezultāti tiek nākamajām pakāpēm nodoti tieši.
Šādi tiek efektīvi izmantoti FPGA resursi un sekmēta ātrdarbība.
%~ Nav nepieciešams dekodēt instrukcijas...

Pēc CPU analoģijas, var uzskatīt, ka šādā modelī FPGA, veic vienu
īpaši augstas kompleksitātes instrukciju kura tiek izpildīta noteikta,
iespējams mainīga, skaita takts ciklu laikā.

Ja ir pieejami FPGA resursi, tad var iegūt datu paralelitāti replicējot 
vairākus paralēlus signāltraktus, palielinot datu caurlaides spēju. Kā arī, ja ir
jāapstrādā datu straume, ievērojamu datu caurlaides spējas uzlabojumu var
iegūt ar ,,konveijera principu'' --- virzot jaunu datu vienību
signāltrakta pakāpē pirms iepriekšējā datu vienība ir šķērsojusi visu
signāltraktu, tādējādi signāltraktā, dažādās pakāpēs, vienlaikus
var atrasties vairākas datu vienības, līdzīgi CPU instrukciju ,,konveijeram''
(\termEn{instruction pipeline})~\cite{Flynn-arch}.


\subsection{Salīdzinājums} \label{sec:proc-cmp}
\TODO
