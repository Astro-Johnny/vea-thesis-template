\subsection{Eksperimenta protokols: FAST implementāciju salīdzinājums}\label{appx:test1}
Eksperimenta mērķis ir salīdzināt
eksistējošo FAST algoritma implementāciju ātrdarbību CPU platformai, 
dodot iespēju izvērtēt implementācijas detaļas vadoties 
pēc kvantitatīva rādītāja. Iegūtie rezultāti arī uzstāda ātrdarbības 
,,etalonu'' GPU un FPGA platformu implementācijām.

Eksperiments tika veikts uz vairākiem datoriem, kuru aprīkojums norādīts
\ref{tbl:test1-dev}~tabulā. Izmantotie datori nosedz vairākas procesoru
sēriju paaudzes, kā arī reprezentē dažādas veiktspējas ,,spektra'' daļas.
\begin{table}[hb]\footnotesize
	\centering
	\caption{Izmantotās iekārtas (datori).}
	\label{tbl:test1-dev}
	\vspace{4pt}
	\begin{tabular}{cllll}
		\toprule
		\textbf{Nr.} & \textbf{Procesors} & \textbf{Takts frekvence} & 
			\textbf{Operētājsistēma} & \textbf{Arhitektūra}\\
		\midrule
		I & AMD Phenom 9950 & 2.60 GHz & Linux 3.10.7 (Gentoo) & \texttt{x86\_64}\\
		II & Intel Pentium M 740 & 1.73 GHz & Linux 3.10.17 (Gentoo) & \texttt{x86}\\
		III & Intel Core i5-2430M & 2.40 GHz & Windows 7 (+cygwin) & \texttt{x86\_64}\\
		\bottomrule
	\end{tabular}
\end{table}

\begin{table}[hb]\small
	\centering
	\caption{Ātrdarbības rezultāti.}
	\label{tbl:test1-data}
	\vspace{4pt}
	\begin{tabular}{clcccr}
		\toprule
		\input{results1-t1.tbl_tex}
		\bottomrule
	\end{tabular}
\end{table}
\footnotetext{Kadri sekundē (angļu \termEn{frames per second}).}

Par ieejas datiem tika izmantota ,,\termEn{bas-relief}'' attēlu kopa%
	\footnote{Pieejama no \url{http://www.edwardrosten.com/work/junk.tar}},
kuru ātrdarbības pārbaudēm izmantoja arī Rostens~un~Dramonds\cite{FAST}.
Uzstādītais jutības slieksnis $t$ visos testos bija 25, un katra 
attēla, iekārtas un implementācijas testa permutācija tika atkārtota 10 reizes.
Lielā datu apjoma dēļ (600 ieraksti) visa kopa netiek atspoguļota. Datu
apkopojums redzams \ref{tbl:test1-data}~tabulā, kur ,,Vid.~max'' kolonna
atspoguļo augstāko vērtību no kopas attēlu vidējiem rādītājiem
(t.i., ,,grūtākā'' kopas attēla vidējais apstrādes laiks 10 mērījumos).
,,Min'' un ,,Max'' kolonnas atspoguļo attiecīgi absolūto minimumu un absolūto
maksimumu no (80) uzņemto mērījumu sērijas. \TODO
