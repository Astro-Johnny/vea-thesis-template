\section*{Secinājumi un priekšlikumi}%
\addcontentsline{toc}{section}{Secinājumi un priekšlikumi}
%FIXME: Izdalīt atsevišķu nodaļu kopsavilkumam?
%~ Autors, šī maģistra darba ietvaros ir veicis gan teorētisku,
%~ gan empīrisku pētījumu par attēlu raksturpunktu noteikšanas algoritmiem,
%~ to komponentēm un skaitļošanas platformām, kurām analizētie 
%~ algoritmi ir implementējami. Darbā veikto var izdalīt sekojošos punktos:
%~ \begin{itemize}
	%~ \item CPU, GPU un FPGA skaitļošanas platformu apskats un salīdzinājums;
	%~ \item ORB algoritma literatūras analīze;
	%~ \item ORB ātrdarbības salīdzinājums vairākām platformām;
	%~ \item FAST FPGA implementācijas izstrāde; %(sk.~\ref{sec:fast-fpga}~nod. un pielikumu~\ref{appx:fast-fpga});
	%~ \item BRIEF FPGA implementācijas modeļa izveide; %(sk.~\ref{sec:fast-fpga}~nod.);
%~ \end{itemize}

Izstrādājot šo darbu, autors izdarījis vairākus secinājumus un atzinumus,
tajā skaitā:
\begin{itemize}
	\item Eksistē liels skaits stūru detektoru, bet trūkst literatūras to
		kvalitātes salīdzinājumam, jo ir apgrūtināta salīdzināšanas
		kritēriju definēšana.
	\item ORB algoritms sastāv no vairākām komponentēm, no kurām
		Gausa filtrēšana un Harisa mērs ir komponentes ar lielāko skaitu
		matemātisko operāciju. Lielāko iespaidu uz kopējo ātrdarbību dos
		uzlabojumi tieši šajās komponentēs.
	\item BRIEF (un ORB) ātrdarbība uzlabojama veicot ,,selektīvu'' Gausa
		filtrēšanu tikai izmantotajiem pikseļiem, samazinot operāciju
		skaitu pat 4 reizes\footnote{4 reizes tipiskā gadījumā. Uzlabojums atkarīgs no BRIEF parametriem.}
		(sk.~\pageref{C:selective-gauss}~lpp.).
	\item Teorētiskie skaitļošanas jaudas rādītāji (piem.~FLOPS) nav
		objektīvs platformu salīdzināšanas kritērijs, jo neņem vērā
		arhitektūras struktūras atšķirības.
	\item Liels skaits ātrdarbības ietekmējošo faktoru nosaka, ka 
		\emph{tikai empīriska analīze} dod objektīvus rezultātus
		algoritmu ātrdarbības salīdzināšanai.
	\item Konkrēta algoritma augsta veiktspēja vienam platformas tipam (piem.~CPU),
		nenosaka ,,automātisku'' tā augstu veiktspēju visiem skaitļošanas
		platformu tipiem. Kā arī, viens algoritms var sasniegt augstāko
		veiktspēju vienā platformā,
		kamēr cits algoritms --- citā platformā.
	\item Algoritma ātrdarbības uzlabojumi var dot atšķirīgus rezultātus
		pat vienas platformas arhitektūras sērijas (piem.~\texttt{x86}) 
		dažādu paaudžu ierīcēm, arhitektūras attīstības dēļ
		(sk.~\ref{sec:fast-compare}~nod).
	\item Algoritmu \emph{ātrdarbību}, tiešā veidā, \emph{nosaka} tā
		\emph{implementācijas struktūras pielāgojamība} konkrētai
		skaitļošanas \emph{platformai} un tās aparatūras definētai struktūrai.
	\item Datorredzes algoritmu implementēšanai iegultajās sistēmās, FPGA
		platformai ir vislielākais veiktspējas potenciāls, pētījumā uzrādot
		augstākos rezultātus, pateicoties tā arhitektūras elastībai,
		pielāgojot to algoritmam aparatūras līmenī.
\end{itemize}

Darba sākumā izvirzītie mērķi ir sasniegti --- ir veikts pētījums par
raksturpunktu noteikšanas algoritmu ātrdarbību, iegūti empīriski rezultāti un
izstrādāti risinājumi ātrdarbības uzlabošanai ar FPGA.
