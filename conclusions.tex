\section*{Secinājumi un priekšlikumi}%
\addcontentsline{toc}{section}{Secinājumi un priekšlikumi}
Izstrādājot šo darbu, autors izdarījis vairākus secinājumus un atzinumus.
\begin{itemize}
	\item Lai arī abas ir aparatūras apraksta valodas, VHDL un Verilog,
		savas uzbūves un atšķirīgas problēmu pieejas rezultātā piemērotas
		dažādam pielietojumam, VHDL vairāk piemērota sintezējamu, augstas
		abstrakcijas līmeņu shēmu aprakstam, savukārt, Verilog piemērota
		vidēja un zema abstrakcijas līmeņa komponenšu 
		simulācijas modeļu aprakstam (sk.~\ref{sec:hdl-comparison}~nod.).
	\item FPGA uzbūves rezultātā kopēja trīsstāvokļu kopne nav realizējama
		FPGA, bet eksistē alternatīvas, kā, piem.,~darbā realizētā sazarotās
		ķēdes veida kopne (sk.~\ref{sec:perry-bus}~un \ref{sec:databus}~nod.).
	\item Unikālo mikrooperāciju skaitu kontroles iekārtā iespējams samazināt
		līdzīgo instrukciju operāciju koda fragmentam piešķirot
		mērķa komponentes interpretējamu nozīmi, rezultātā izveidojot šablonveida
		mikroinstrukcijas un deleģējot daļu operācijas koda dekodēšanu ārējai
		komponentei (sk.~\ref{sec:perry-instr}~un \ref{sec:AR}~nod.).
\end{itemize}

Darba izstrādes laikā, pētot un salīdzinot FPGA arhitektūras, 
izvirzīta hipotēze, ka
 FPGA loģisko bloku granularitāte ietekmē tajā implementetās
loģikas veiktspēju. Respektīvi, rupjas granularitātes 
arhitektūra potenciāli ir ar lielāku veiktspēju kombinacionālo shēmu realizācijā,
salīdzinājumā ar smalkas granularitātes arhitektūru (sk.~\ref{sec:fpga}~nod.).
Šī hipotēze šajā darbā netiek pierādīta un
tās pārbaudei būtu nepieciešams papildus pētījums.

Izstrādes laikā tika apzināti
vairāki potenciāli uzlabojumi:
\begin{itemize}
	\item aparatūras līmeņa apakšprogrammu izsaukšanas atbalsts
		sekmētu efektīvāku un modulārāku izpildes programmas kodu,
	\item ieviešot pārtraukumu mehānismu, kodols varētu tūlītēji reaģēt
		uz ārējiem notikumiem 
		izpildot speciālu, lietotāja definējamu pārtraukuma apstrādes apakšprogrammu.
\end{itemize}

Šie uzlabojumi netika implementēti līdz darba nodošanas brīdim,
lai varētu sasniegt darba sākumā izvirzīto funkcionalitāti.
Papildus izstrādāts asemblerkoda translators, kas nodrošina asemblerkodā
rakstīto testa programmu translēšanu mašīnkodā tā ielādei VHDL simulatorā
procesora kodola un mikrokontroliera testēšanai.

Darba izpildes sākumā izvirzītie uzdevumi ir izpildīti un mērķi sasniegti
--- ir izstrādāts modulārs mikrokontroliera kodols un tā implementācijas
paraugs mikrokontrolierī.



% 1: FPGA ir kruts jo iespeejams sintezeet HDL aprakstus

% Vaig RJMP
