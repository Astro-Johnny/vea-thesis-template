\section*{Secinājumi un priekšlikumi}%
\addcontentsline{toc}{section}{Secinājumi un priekšlikumi}
Izstrādājot šo darbu, autors izdarījis vairākus secinājumus un atzinumus.
\begin{itemize}
	\item VHDL un Verilog, lai arī abas ir aparatūras apraksta valodas,
		savas uzbūves un atšķirīgas problēmu pieejas rezultātā, piemērotas
		dažādam pielietojumam. VHDL vairāk piemērota sintezējamu, augstas
		abstrakcijas līmeņu shēmu aprakstam, savukārt, Verilog piemērota
		vidēja un zema abstrakcijas līmeņa komponenšu 
		simulācijas modeļu aprakstam (sk.~\ref{sec:hdl-comparison}~nod.).
	\item (FPGA arhitektūras granularitāte \todo ) 
		(sk.~\ref{sec:fpga}~nod.).
	\item FPGA uzbūves rezultātā, kopēja trīsstāvokļu šina nav realizējama
		FPGA, bet eksistē alternatīvas, kā piem.,~darbā realizētā sazarotās
		ķēdes veida šina (sk.~\ref{sec:perry-bus}~un \ref{sec:databus}~nod.).
	\item Unikālo mikrooperāciju skaitu kontroles iekārtā iespējams samazināt,
		līdzīgo instrukciju operāciju koda fragmentam piešķirot
		mērķa komponentes interpretējamu nozīmi, rezultātā izveidojot šablonveida
		mikroinstrukcijas un deleģējot daļu operācijas koda dekodēšanu ārējai
		komponentei (sk.~\ref{sec:perry-instr}~un \ref{sec:AR}~nod.).
\end{itemize}

Izvirzītais darba mērķis tika sasniegts, un izstrādes laikā tika apzināti
vairāki potenciāli uzlabojumi.
\begin{itemize}
	\item Ieviešot pārtraukumu mehānismu, kodols varētu tūlītēji reaģēt
		uz ārējiem notikumiem ,,preemptīvi''\footnote{%
			Pārtraucot un atliekot pamatrprogrammas izpildi.}
		izpildot speciālu, lietotāja definējamu pārtraukuma apakšprogrammu.
	\item Aparatūras līmeņa apakšprogrammu izsaukšanas atbalsts,
		sekmētu efektīvāku un modulārāku izpildes programmas kodu.
\end{itemize}
Šie uzlabojumi netika uzreiz implementēti,
tādēļ, ka tie nav fundamentāli mērķa sasniegšanai, un lai būtu iespējams 
noslēgt pilnu iteratīvu izstrādes ciklu, t.i.~jaunas modifikācijas netiek
ieviestas pirms iesāktā izstrādes revīzija tiek finalizēta.



% 1: FPGA ir kruts jo iespeejams sintezeet HDL aprakstus

% Vaig RJMP
