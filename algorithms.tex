\section{Attēlu raksturpunkti} \label{sec:algo}
Attēla \newTerm{raksturpunkti} (\termEn{keypoints} vai \termEn{feature points})
ir attēla punkti (pikseļi), kuru apkārtne attēlā 
satur (vai potenciāli satur) noderīgu informāciju mašīnredzes algoritmiem.
Raksturpunktus ļoti bieži (t.i.~praktiski vienmēr)
nosaka ar ,,stūru meklēšanas'' algoritmiem
(sk.~\ref{sec:corners}~nod.,~\pageref{sec:corners}~lpp.),
tāpēc literatūrā vārdi ,,raksturpunkts'' (\termEn{keypoint}) un ,,stūris''
(\termEn{corner}) bieži tiek izmantoti kā sinonīmi.

Šis darbs apskata attēlu raksturpunktu pāru noteikšanu, un šādiem
algoritmiem ir
nozīmīga raksturpunktu apkārtnes unikalitāte,
lai raksturpunktus būtu iespējams salāgot ar pēc iespējas mazāk kļūdām,
nevis to strikta atbilstība kādam attēlā redzamā objekta stūrim.
Raksturpunktu pāru noteikšanas algoritmus var iedalīt divos etapos:
\begin{enumerate}
	\item raksturpunktu detektēšana;
	\item atbilstošo raksturpunktu atrašana vairākos attēlos.
\end{enumerate}
Turpmākās apakšnodaļas apskata šos etapus (sākot no pēdējā),
un to algoritmus plašāk.

\subsection{Attēlu raksturpunktu salāgošanas algoritmi} \label{sec:matching}
\begin{figure}[tbh]
	\centering
	\includegraphics[width=0.8\textwidth]{orb-match}
	\caption{Attēlu pāra raksturpunktu salāgošana ar ORB~\cite{ORB}.}
	\label{fig:orb}
\end{figure}

Raksturpunktu salāgošana ir punktu pāru (vai kopu) atrašana divos
(vai vairākos) attēlos, kuri atbilst vienam un tam pašam attēlā redzamam
objektam. Salāgošanas piemērs redzams \ref{fig:orb}~attēlā, kur ar zaļām līnijām
savienoti salāgotie raksturpunktu pāri un ar sarkanu apzīmēti raksturpunkti,
kuriem pāris nav atrasts.

Raksturpunktu salāgošana tiek izmantota tādos mašīnredzes pielietojumos, kā
% TODO: citēt pielietojumus
objektu atpazīšanā un sekošanā, attēlu ,,sašūšanā'', 
telpiskās informācijas rekonstrukcijā (kartēšanā) no attēliem,
vienlaicīgā pašlokalizācijā un kartēšanā (SLAM), u.c..

Raksturpunktu salāgošanas pamatā ir šo punktu ,,aprakstīšana'' izveidojot
raksturpunkta \termTech{deskriptoru}, kas satur informāciju,
kas ir pietiekami unikāla, lai nesakristu ar citiem raksturpunktiem attēlā
un ir noturīga pret sagaidāmām attēla īpašību izmaiņām.
\termEn{Deskriptoram} tātad nepieciešamas vai vēlamas šādas īpašības:
\begin{itemize}
	\item \newTerm{\emph{diskriminitāte}} --- spēja izšķirt raksturpunktus
		pēc to \termTech{deskriptoriem};
	\item \emph{pozīcijas invariance} --- spēja salāgot raksturpunktus
		neatkarīgi no tā koordinātēm attēlā;
	\item \emph{rotācijas invariance} --- spēja salāgot raksturpunktus
		neatkarīgi no attēla rotācijas leņķa;
	\item \emph{intensitātes nobīdes invariance} --- spēja salāgot raksturpunktus
		neatkarīgi no globālas intensitātes izmaiņas;
	\item \emph{mēroga invariance} --- spēja salāgot raksturpunktus
		neatkarīgi no mēroga (attēla izmēru un/vai raksturpunkta objekta attāluma izmaiņa);
	\item \emph{perspektīvas invariance} --- spēja salāgot raksturpunktus
		neatkarīgi no perspektīvas (attēla uzņemšanas pozīcijas izmaiņa);
	\item \emph{trokšņu noturība} --- spēja salāgot raksturpunktus
		attēlā (iespējami) neatkarīgi no ,,trokšņa'' līmeņa attēlā;
\end{itemize}
Ir norādītas vairākas attēla izmaiņu invariances īpašības,
kas nozīmē ka tās nevar izmantot lai aprakstītu raksturpunktu.
Variance ir nepieciešama lai nodrošināti diskriminitāti un, praktiski visos
algoritmos, tiek izmantota lokālas intensitātes izmaiņas attēlā 
rakturpunkta tuvējā apkārtnē.

Eksistē vairāki algoritmi \termTech{deskriptoru} izgūšanai un
salīdzināšanai, t.sk.,~%
SIFT, % FIXME: Vai SIFT arī apzīmē deskriptoru?
SURF, BRIEF un tā varianti, ORB, u.c..
Algoritmus var izvērtēt pēc jau uzdotajām īpašībām, 
bet jāņem vērā arī to veiktspējas īpašības:
\begin{itemize}
	\item \emph{skaitļošanas kompleksitāte}, kas tieši ietekmē algoritma
		ātrdarbību;
	\item \emph{informācijas blīvums}, kas atspoguļo cik liela ir katra
		informācijas bita variance, kas ļauj sasniegt lielāku diskriminitāti
		pie mazāka bitu skaita.
\end{itemize}
% TODO?: Izvērst dažādo algoritmu teorētisko aprakstu?

Netiešs, bet būtisks, salāgošanu ietekmējošais faktors ir raksturpunktu
atlase no attēla, jo salāgošana tiek veikta tikai starp atlasītajiem
punktiem. Raksturpunktu (un to atlases algoritmu) galvenās īpašības
ir to variance, kas nosaka cik aprakstoši ir šie punkti,
un \newTerm{atkārtojamība}, kas ir ieviests termins un izsaka
atbilstošo (salāgojamo) punktu pārus pret kopējo raksturpunktu skaitu~%
\cite{FAST}\cite{SIFT-FPGA}.
Raksturpunktu atlases algoritmi un to īpašības plašāk apskatītas
\ref{sec:corners}~apakšnodaļā.

Šajā darbā, turpmākai algoritma implementāciju veiktspējas izvērtēšanai
un salīdzināšanai izvēlēts, ORB algoritms, vai konkrētāk tā salāgošanas
komponente --- rBRIEF.
% TODO: Forward ref
Šāda izvēle pamatota ar to, ka rBRIEF raksturpunktu salāgošanas
spēja ir līdzīga vai labāka nekā SIFT, bet ir par divām kārtām ātrāks nekā
SIFT un par kārtu ātrāks nekā SURF \cite{ORB}.
rBRIEF \termTech{deskriptors} ir arī noturīgāks pret troksni nekā
SIFT un tā rotācijas invariance līdzvērtīga SIFT \cite{ORB}.



\subsection{Attēlu raksturpunktu atlases algoritmi} \label{sec:corners}
Attēlu raksturpunktu atlase ir punktu apakškopas atlase pēc noteiktiem
kritērijiem. Kā jau minēts nodaļas ievadā (\pageref{sec:algo}~lpp.),
raksturpunktu atlasei bieži tiek izmantoti ,,stūru'' meklēšanas algoritmi,
kādi ir arī visi šajā darbā apskatītie raksturpunktu atlases algoritmi.

% TODO: Algoritmu aprakstošās īpašības

Eksistē vairāki stūru atrašanas algoritmi: \termEn{Harris} stūru detektors,
SUSAN, SIFT, KFT, FAST, u.c.. 
Šajā darbā apskatītais ORB algoritms, rakturpunktu atlasei izmanto oFAST, kas
ir FAST modifikācija, kas nosaka arī virziena informāciju.

