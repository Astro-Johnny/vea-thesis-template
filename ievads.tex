\section*{Ievads} \addcontentsline{toc}{section}{Ievads}
Vai roboti varēs redzēt? Vai automašīnas varēs autonomi braukt?
Šie ir tikai divi piemēri attēlu apstrādes iespējamiem pielietojumiem un
izpētes virzieniem. Objektu atpazīšana un izsekošana,
telpas rekonstrukcija no attēliem --- tas viss balstās uz dažādiem 
attēlu apstrādes algoritmiem.

Attēlu apstrādes pielietojumiem saistībā ar mašīnredzi, gandrīz vienmēr, ir
nepieciešamība apstrādi veikt reālā laikā, 
kas, pie ierobežotiem skaitļošanas resursiem, ļoti būtiski ietekmē 
izmantojamo algoritmu nosacījumus veiktspējai. Lai to nodrošinātu,
bieži vien ir jādefinē jauni algoritmi vai jāoptimizē esošie algoritmi,
pēc iespējas lietderīgākai pieejamo skaitļošanas resursu izmantošanai.

\TODO
