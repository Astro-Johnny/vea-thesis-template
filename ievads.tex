\section*{Ievads} \addcontentsline{toc}{section}{Ievads}
% FIXME: Should I do the "screaming" first sentence?
Zinātne nestāv uz vietas! Nepārtraukti tiek uzsākti jauni zinātniskie projekti,
un arvien biežāk tajos nepieciešami specializēti elektroniskie risinājumi.
Šie risinājumi ir projekta specifiski, bet tajos bieži ir nepieciešamas
kontroles iekārtas, datu formēšanas un pārraides iekārtas, un
šim pielietojumam var izmantot specializētus mikrokontrolierus.
Šādu mikrokontrolieru izstrāde tiek krietni vienkāršota, ja pieejams
modulārs mikrokontroliera kodols ar vienkāršu un adaptējamu saskarni.

Sevišķas prasības specializētu mikrokontrolieru izstrādei ir kosmosa
tehnoloģijās, kur jaunas ierīces un komponentes pieprasa īpaši rūpīgu
pārbaudi, uzticama mikrokontroliera kodola universalitāte ir sevišķi
vērtīga.

Šī bakalaura darba mērķis ir, pirmkārt, izstrādāt modulāru mikrokontroliera kodolu,
kurš, bez vai ar minimālām modifikācijām, būtu izmantojams,
vispārēja un specializēta pielietojuma, mikrokontrolieru implemetācijās,
sintēzei FPGA (\termEn{{field-programmable} gate array}), un otrkārt,
izstrādāt vienkāršu parauga mikrokontrolieri, kas šo kodolu izmanto,
demonstrējot izstrādātā kodola izmantošanu šim nolūkam.
Mērķa sasniegšanai ir izvirzīti vairāki uzdevumi.
\begin{enumerate}
	\item Kodola un perifērijas saskarnes definēšana,
		kurai jābūt pietiekami vienkāršai, lai atbalstītu plašu spektru
		komponenšu.
	\item Kodola izmantojamās instrukciju kopas definēšana. Tai jābūt
		vienkāršai un jānodrošina pamata funkcionalitāti.
	\item Kodola arhitektūras definēšana un izstrāde. Tas galvenokārt nozīmē
		visu kodola apakškomponenšu izstrādi un komplektēšanu.
	\item Parauga mikrokontroliera uzbūves definēšana un komponenšu izstrāde.
\end{enumerate}

Darba izstrādes galvenais instruments ir VHDL, kas ir
aparatūras apraksta valoda (HDL) ar kuru var aprakstīt shēmas darbību un uzbūvi.
HDL nozīmīgākā īpašība ir shēmas aprakstu sintezējamība fiziskā realizācijā,
galvenokārt FPGA mikroshēmā, kas ir darba mērķa platforma.
FPGA ir integrētā shēma, kuru vienkāršoti var 
uzskatīt par ,,tukšu'', ar HDL palīdzību, programmējamu mikroshēmu.

\todo

