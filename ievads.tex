\section*{Ievads} \addcontentsline{toc}{section}{Ievads}
	Šis darbs apraksta procesora un tiešās sistēmas perifērijas izveidi
	VHDL apraksta valodā. Izklāstītā informācija balstīta vairāk uz
	praktiskā darba rezultātiem un tādējādi arī darba gaitā pieņemtajiem
	lēmumiem sastapto problēmu risinājumam.
	
	Apskatīta procesora uzbūve bez kopējās datu šinas un reducētu instruk\-ciju
	kopu, realizācijai uz FPGA čipa platformas — konkrēti,
	\termEn{Actel Fusion Embedded Development Kit} ar \texttt{M1AFS1500}
	FPGA čipu. Darba mērķi tātad ir:
	\begin{itemize}
		\item Funkcionāla procesora izstrāde VHDL;
		\item Perifērijas izstrāde sintezējamai sistēmai;
		\item Sistēmas sintēze un pārbaude.
	\end{itemize}
	
	Darba gaitā pieņemtie lēmumiem, kas ievērojami izmaina sistēmas
	imple\-men\-tā\-ciju un ir izstrādes pagrieziena punkts, nosauktas par
	„Revīzijām”.
	Lai gan pamatā darbs atspoguļo galējo rezultātu, nodaļu
	beigās pievienots pārskats par Revīziju vēsturi — to iespaids uz
	izstrādi, kā arī šo izmaiņu pieņemšanas iemesli.
