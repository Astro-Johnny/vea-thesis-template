\subsection{BRIEF FPGA implementācijas modelis} \label{sec:brief-fpga}
Principā,
BRIEF deskriptora noteikšana ir īpaši piemērota implementēšanai FPGA, jo
tiek izmantota vienkārša salīdzināšana un 
FPGA atsevišķu, konkrētu, iepriekš definētu elementu izgūšana
no masīva ir ļoti efektīva, jo tiek realizēts kā vienkāršs loģisko elementu 
savienojums (vads).
\begin{figure}[tbh]
	\centering
	%\def\svgwidth{\linewidth}
	\def\svgscale{1.1}
	{\small\input{img/brief-fpga.pdf_tex}}
	\caption{Vienkāršota uzbūves shēma BRIEF aprēķina vienībai.}
	\label{fig:brief-fpga}
\end{figure}
Pēc modeļa, kas parādīts \ref{fig:brief-fpga}~attēlā,
izmantojot $n_d$ salīdzinātājus, var noteikt BRIEF deskriptoru
vienā takts ciklā, pieņemot ka visu punktu vērtības no attēla gabala ir 
izgūstamas vienlaikus.

Tomēr, lielākais veiktspējas ierobežojums ir $\hat{\vb{p}}$ iegūšana
filtrējot attēla gabalu $\vb{p}$ ar Gausa filtru. Attēla gabala filtrēšanai
nepieciešams realizēt konvolūciju, kas prasa lielu skaitu reizināšanas un
saskaitīšanas operācijas. Pie tam Gausa matricas $G$ elementi ir skaitļi
ar decimāldaļu, nevis veseli (\termEn{integer}) skaitļi, kam nepieciešams
izmantot peldošā vai fiksētā komata skaitļu formātu.

\phantomsection\label{C:selective-gauss}
Zināma resursu ekonomiju dod tas, ka ne visi attēla gabala $\hat{\vb{p}}$
punkti tiek izmantoti BRIEF noteikšanai, tādēļ var implementēt daļēju Gausa
filtru, aprēķinot tikai nepieciešamo $2n_d$ skaita punktu vērtības.
Bet, neskatoties uz to,
pieņemot, ka $n_d = 256$ un izmantotā matrica $G$ ir $5 \times 5$ pikseļu
liela, nepieciešamo reizināšanas un saskaitīšanas operāciju skaits ir
$2\cdot 256 \cdot 5^2 = 12800$.

Autors secina, ka iepriekšminētā modeļa nodrošināšana ar datiem
($\hat{\vb{p}}$) pietiekamā ātrumā (vienā takts ciklā) ir tuvu neiespējama,
vai vismaz ļoti nepraktiska. Tādēļ, tiek piedāvāts alternatīvs modelis,
izdalot atsevišķas apstrādes vienības pikseļu pāra salīdzināšanai, iekļaujot
tajā arī Gausa filtrēšanu, kā ilustrēts \ref{fig:gauss+brief}~attēlā, kur
ar $w_G$ apzīmē Gausa matricas platumu un garumu.
\begin{figure}[tbh]
	\centering
	%\def\svgwidth{\linewidth}
	\def\svgscale{1.3}
	{\small\input{img/gauss+brief.pdf_tex}}
	\caption{Uzbūves shēma pikseļu pāra salīdzināšanas vienībai ar Gausa filtru.}
	\label{fig:gauss+brief}
\end{figure}

