\subsubsection{OpenCV FAST impementācija} \label{sec:fast-ocv}
OpenCV\footnote{\url{http://opencv.org/}} piedāvā alternatīvu FAST9 algoritma
implementāciju (FAST ar citām $n$ vērtībām netiek implementētas)~\cite{OpenCV-src}.
Tās pamatā ir segmenta tests ar papildinātu
priekšpārbaudi\cite{Rosten-tracking}\cite{FAST}, kas var klasificēt vairumu
gadījumu, kad punkts $\vec{p}$ nav ,,stūris''. Priekšpārbaudi iztur
punkti kas ir ,,stūri'', jeb visi $\vec{p} \in F_9(\vb{A}, t)$,
un noteikta apakškopa punktu, kuri neatbilst $F_9(\vb{A}, t)$ un kuru
apkārtne ir ļoti specifiska un reālos attēlos ir statistiski reti.
Neskatoties
uz to, jebkuram punktam, kas iztur priekšpārbaudi, tiek veikts segmenta
tests, lai nodrošinātu pilnīgu klasifikāciju.
Var tikt veikts arī daļējs segmenta tests, jo priekšpārbaude var,
vairumā gadījumu, papildus noteikt vai ir meklējams ,,tumšs'' segments, vai
,,gaišs'' segments.

Šāda implementācija, ,,skalāra~koda\footnote{Kods, kas neizmanto SIMD
	instrukcijas, pretstatā ,,vektoriskam'' kodam.}''
izpildījumā, dod nelielu uzlabojumu. OpenCV implementācijas galvenā
priekšrocība ir SSE2 SIMD instrukciju izmantojums, kas ļauj vienlaikus
apstrādāt 16 kandidātu punktus (pikseļus). Empīriski novērotais
uzlabojums ir līdz 4 reizēm ātrāk (sk.~pielikumu~\ref{appx:test1})
par neapmācītu mašīnmācāmo implementāciju.

\subsubsection*{OpenCL versija}
OpenCV arī piedāvā OpenCL FAST implementāciju izpildei GPU, bet tās
ātrdarbība testos nepārsniedza CPU implementāciju ar CPU
(sk.~pielikumu~\ref{appx:test2}).

\TODO
